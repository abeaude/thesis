\documentclass[../main.tex]{subfiles}
\usepackage{silence}
\WarningFilter{glossaries}{No \printglossary or \printglossaries found}

\begin{document}
\ifSubfilesClassLoaded{%
	\graphicspath{{figures/3-SOTA/}}%
	\setcounter{chapter}{2}%
}{
	\graphicspath{{../figures/3-SOTA/}}%
}
\chapter{Deep-learning for omics data analysis}\label{chap:sota}
\minitocpagecentered

\section{Prediction from single omics data}
	With the application of deep-learning to omics data multiple tasks are possible: predict gene expression from the DNA sequence~\cite{Avsec2021}, predict the impact of a sequence variation on \glsxtrlong{dnam}~\cite{Zeng2017}, annotate genome variants~\cite{Quang2014}, etc\@.
	In this document, we will focus on tasks predicting a phenotype: does a patient have a cancer or not, what type of cancer a patient has, or survival-related tasks.

	Omics data suffers from a curse of dimensionality: they are high-dimensional but the number of available examples remains limited.
	To overcome the limitation of training examples, various strategies have been used: self-supervised training, feature engineering, transfer learning.
	\Glspl{ae} or \glspl{vae} are used to learn a compact representation of the input; the latent representation should contain relevant biological features.
	The latent representation can be fed to a downstream classifier such as \gls{svm}~\cite{Zhang2020}, \gls{xgboost} or k-NN classifier~\cite{Arafa2023} logistic regression~\cite{Wang2018} or used to train a neural network classifier~\cite{Karim2019}.
	The encoder part of the \gls{ae} can be fine-tuned on a predictive task through transfer learning~\cite{Levy2020,Kaczmarek2022}.
	\citeauthor{Hanczar2018} pretrain each layer of an \gls{mlp} with a denoising \gls{ae}~\cite{Hanczar2018}.
	Feature engineering involves the creation, selection, and transformation of features to enhance the model's ability to capture essential patterns or signals hidden within the omics data.
	Many works use feature selection to reduce the input dimension of neural networks.
	The selections process can be based on features with the highest median absolute deviation, based on a student t-test selection~\cite{Liu2019}, based on knowledge on important features~\cite{Kaczmarek2022} or based on another models such as \glspl{rf}~\cite{Wojewodzic2021,Liu2019} or LASSO~\cite{Liu2019} or PCA~\cite{Yu2019}.
	Other approaches use the available knowledge to transform raw features to annotated biologically relevant features.
	DeepCC~\cite{gaoDeepCCNovelDeep2019} applies an \gls{mlp} on biologically informed features by transforming gene expression data into a functional spectrum, \ie{}a list of enrichment scores calculated by gene set enrichment analysis.
	DeepTRIAGE~\cite{beykikhoshkDeepTRIAGEInterpretableIndividualised2020a} transforms gene counts to pathways counts by summing all gene counts belonging to a gene ontology biological process pathway or a KEGG pathway.
	\citeauthor{Zhang2020} use a gated \gls{mlp} to identify a set of important features, those features are then used to train an \gls{ae}~\cite{Zhang2020}.

	Using multiple models to select features and perform predictive tasks results in increased complexity, and the process of selecting features before constructing the predictive models limits the capacity to consider all feature interactions.
	On the contrary, deep learning architectures can adapt to any number of features and select the most relevant features through their architectures.
	They allow modeling complex, non-linear relationships and have been shown to outperform classical machine learning approaches with training sets of a few thousand examples~\cite{Hanczar2022}.

	\Gls{mlp} are well studied for predicting outcomes from omics data.
	\citeauthor{yuArchitecturesAccuracyArtificial2019b}, explored different \gls{mlp} architectures by varying the number of neurons in each layer and the number of layers~\cite{yuArchitecturesAccuracyArtificial2019b}.
	They showed that wider networks perform better than deeper ones.
	Cancer types are accurately predicted from gene expression data with an \gls{mlp}~\cite{Divate2022}.
	Adding inductive bias can help the model better generalize when training data is limited.
	Inductive bias corresponds to the assumptions and prior knowledge incorporated into the model to better generalize.
	Introducing biological knowledge into the architecture is a robust inductive bias that should help the model better generalize.
	Biological system and their knowledge representation are hierarchical, making this kind of inductive bias suitable for hierarchical architecture such as \gls{mlp}~\cite{bourgeaisDeepGONetSelfexplainable2021,haoPASNetPathwayassociatedSparse2018}.
	In DeepGONet~\cite{bourgeaisDeepGONetSelfexplainable2021}, authors constrain the \gls{mlp} architecture with the GO biological processes hierarchy.
	In PASNet~\cite{haoPASNetPathwayassociatedSparse2018,Hao2018}, authors use the REACTOME hierarchy to constrain their \gls{mlp} architecture.
	With such architectures, each neuron represents a known pathway, and only neurons with known relations are connected.
	For the input layer, neurons represent genes, and they are connected to neurons of the first hidden layers if they are involved in the corresponding pathway.
	In practice, this is achieved by masking the unknown relationships in a fully connected layer:
	\begin{equation}
		\symbf{a}^{l} = f\left( \symbf{a}^{l-1} \cdot \left( W^l \odot M^l\right) + \symbf{b}^{l} \right) \label{eq:mlp_sparse_matrix}
	\end{equation}
	where \(M^l\) is a binary mask restricting connections between neurons to known biological relations and \(\odot\) is the Hadamard product or element-wise product: \({\left(A\odot B\right)}_{ij} = {\left(A\right)}_{ij}{\left(B\right)}_{ij}\).

	Earlier, we presented approaches that trained an \gls{ae} and used the latent representation as input to another classification model.
	However, deep learning is a flexible approach and a supervision of the latent space from a classifier can be added during the \gls{ae} training.
	Such approach allow for an end-to-end training of the model and constrains the latent space to learn relevant features for the classification~\cite{goreCancerNetUnifiedDeep2022}.
	For such models, the training loss correspond to the sum of the unsupervised loss and the supervised loss: \(\symcal{L}_{AE} + \symcal{L}_{CE}\).

	Promising results of \gls{cnn} in computer vision inspired its application in precision medicine.
	However, expression profile are not 2D like images but 1D.
	To apply 2D convolutions onto expression profile a transformation step to go from a 1D representation to a 2D representation: \(\symcal{T}: \symbb{R}^{d} \mapsto \symbb{R}^{h\times w}\).
	For such transformation, a zero-padding of the 1D vector is required for the transformation to be applied.
	The simplest transformation often corresponds to a naive reshaping~\cite{Wang2021,deGuia2019,Elbashir2019,Chatterjee2018}\@. \citeauthor{deGuia2019}, transform the gene expression vector to a \(102\times 102\) image after a feature selection process~\cite{deGuia2019}.
	Similarly, \citeauthor{Elbashir2019} reshape the gene expression vector to a \(127 \times 114\) image~\cite{Elbashir2019}.
	Some approaches hypothetize that chromosical adjacent genes are more likely to interact.
	Based on that assumptions, features are reordered based on their poistion in the genome before reshaping the expression to a 2D image~\cite{Mostavi2020,Lyu2018,Yin2022}.
	Instead of naive reshaping, knowledge-based transformation have been depicted.
	Hierarchical tree structure such KEGG BRITE are used to transform features, then the tree structure is represented as treemap with a tiling algorithm~\cite{LpezGarca2020,maOmicsMapNetTransformingOmics2019}.
	The chosen algorithm ensures that all samples share the same spatial arrangements by ordering genes based on their mean level of expression across the dataset.
	The treemap image are then colorized based on the gene expression abundances.
	Instead of forcing a 2D representation of expression profile, 1D convolution can be directly used.
	Similarly to 2D \gls{cnn}, features were reordered based on their genomic position~\cite{Mostavi2020,Zhao2020,Yin2022} or expression profile were leave untouched~\cite{Mohammed2021}.
	While many approaches around convolutions were developped they are not suited for omics data as they introduce wrong inductive biases.
	Convolutions are more suited for structured data, and even the incorporation of a structure in the data based on the chromosomal location or some knowledge is a constraint that limits the range of possible interactions.
	For instance such architectures do not consider interactions between genes that are far away from each other in the genome.

	% GNN 
	\Gls{gnn} are architecture used to process graphs structures, that impose constraints on relationships between node entities during the learning process.
	In biology, many interactions between molecular entities are known and are represented as graphs.
	\Gls{gnn} are well suited to study omics data by introducing relationships inductive bias in the learning process.
	\citeauthor{Ramirez2020} used a \gls{gcn} based on a \gls{ppi} graph or a co-expression graph to predict cancer types from gene expression data~\cite{Ramirez2020}.
	The co-expression graph is obtained by computing a correlation matrix between genes, then a threshold is applied on the correlation values to construct the adjacency matrix.
	All patients share the same graphs, predictions are done at the graph level, and nodes represents genes.
	The expression profile of each patient is mapped onto the corresponding nodes.
	Information from neighboring nodes is propagated with \gls{chebconv}~\cite{ChebConv}.
	The thresholding of the correlation matrix, or the use of knowledge based graphs might introduce singletons nodes, \ie{}nodes that are not connected to any nodes.
	The authors tested the impact of the singleton nodes on the accuracy.
	The presence of singleton nodes only increase the accurcy with the \gls{ppi} graph.
	Authors extended their work to survival prediction~\cite{Ramirez2021} with a \gls{gcn} based on a co-expression graph, a gene interaction graph from GeneMania~\cite{WardeFarley2010} or a graph combining the two previous graphs.
	The inclusion of clinical information in the last layer improved the performances of their model.
	\Glspl{gat} have also been used to propagate information from neighboring nodes~\cite{Xing2021}.
	\Gls{sag} have been used in combination of \glspl{gcn} to extract differentially expressed methylation sites when predicting cancer types~\cite{Jiang2023}.
	Node level predictions have also been tested with patient similarity graphs~\cite{Baul2022}.
	Each node, \ie{}patient, is represented by its gene expression profile and nodes are connected based on their similarity.

	\ifSubfilesClassLoaded{%
		\begin{longtblr}[
	caption = {examples single omics},
	entry = {short caption},
	note{a} = {Used for treatment recommendation},
	]{
	colspec = {Q[c,m]Q[l,m]Q[c,m]Q[c,m, wd=2.5cm]Q[c,m,wd=2.2cm]Q[l,m, wd=3.2cm]},%
	hline{1,Z} = {2pt},%
			hline{2} = {1pt},%
			row{2-Z} = {font=\small},%
			rowhead = 1, %
			rowfoot = 0%
		}
	Ref.                                                         & Omics & Task                 & Arch.                                 & Graph      & Feature engineering                        \\ % chktex 2
	\cite{Arafa2023}                                             & mRNA  & sample               & \glsxtrshort{ae}                      & \xmark     & Feature selection                          \\ % chktex 2
	\cite{Wang2018}                                              & DNAm  & subtype              & \glsxtrshort{vae}                     & \xmark     & Feature selection (MAD)                    \\ % chktex 2
	\cite{Karim2019}                                             & CNV   & subtype              & \glsxtrshort{ae} + \glsxtrshort{cnn}  & \xmark     & Feature selection (oncogenes)              \\ % chktex 2
	\cite{Levy2020}                                              & DNAm  & cancer               & \glsxtrshort{vae} + \glsxtrshort{mlp} & \xmark     & Feature selection (MAD)                    \\ % chktex 2
	\cite{Kaczmarek2022}                                         & miRNA & sample               & \glsxtrshort{ae} + \glsxtrshort{mlp}  & \xmark     & \xmark                                     \\ % chktex 2
	\cite{Hanczar2018}                                           & mRNA  & binary               & \glsxtrshort{ae} + \glsxtrshort{mlp}  & \xmark     & \xmark                                     \\ % chktex 2
	\cite{Wojewodzic2021}                                        & DNAm  & sample               & \glsxtrshort{mlp}                     & \xmark     & Feature selection                          \\ % chktex 2
	\cite{Liu2019}                                               & DNAm  & sample               & \glsxtrshort{mlp}                     & \xmark     & Feature selection                          \\ % chktex 2
	\cite{gaoDeepCCNovelDeep2019}                                & mRNA  & subtype              & \glsxtrshort{mlp}                     & \xmark     & Feature selection (Knowledge)              \\ % chktex 2
	\cite{beykikhoshkDeepTRIAGEInterpretableIndividualised2020a} & mRNA  & subtype              & attention-based                       & \xmark     & Feature selection (Knowledge)              \\ % chktex 2
	\cite{Zhang2020}                                             & DNAm  & sample               & attention-based + \gls{ae}            & \xmark     & model-based selection                      \\ % chktex 2
	\cite{yuArchitecturesAccuracyArtificial2019b}                & mRNA  & {sample                                                                                                                \\ grade}        &   {\gls{mlp} \\ \gls{cnn}}                                    & \xmark     &  {LASSO \\ PCA}                                          \\ % chktex 2
	\cite{Divate2022}                                            & mRNA  & cancer               & \gls{mlp}                             & \xmark     & \xmark                                     \\ % chktex 2
	\cite{Elbashir2019}                                          & mRNA  & cancer               & \gls{cnn}                             & \xmark     & Coding genes + Feature selection           \\ % chktex 2
	\cite{deGuia2019}                                            & mRNA  & cancer               & \gls{cnn}                             & \xmark     & Feature selection (threshold)              \\ % chktex 2
	\cite{Wang2021}                                              & mRNA  & cancer               & \gls{cnn}                             & \xmark     & Feature selection (Mutual information)     \\ % chktex 2
	\cite{Mostavi2020}                                           & mRNA  & cancer               & \gls{cnn}                             & \xmark     & Feature selection (threshold)              \\ % chktex 2
	\cite{Lyu2018}                                               & mRNA  & cancer               & \gls{cnn}                             & \xmark     & Feature selection (threshold)              \\ % chktex 2
	\cite{LpezGarca2020}                                         & mRNA  & survival             & \gls{cnn}                             & \xmark     & Feature selection (MAD)                    \\ % chktex 2
	\cite{maOmicsMapNetTransformingOmics2019}                    & mRNA  & grade                & \gls{cnn}                             & \xmark     & Feature selection (threshold)              \\ % chktex 2
	\cite{Hao2018}                                               & mRNA  & survival             & \gls{mlp}                             & \xmark     & \xmark                                     \\ % chktex 2
	\cite{Chatterjee2018}                                        & DNAm  & cancer               & \gls{cnn}                             & \xmark     & \xmark                                     \\ % chktex 2
	\cite{Zhao2020}                                              & mRNA  & cancer               & \gls{cnn}                             & \xmark     & Feature selection (DE)                     \\ % chktex 2
	\cite{Mohammed2021}                                          & mRNA  & cancer               & \gls{cnn}                             & \xmark     & Feature selection (threshold + DE + LASSO) \\ % chktex 2
	\cite{Yin2022}                                               & mRNA  & survival             & \gls{cnn}                             & \xmark     & Feature selection (threshold + CWx)        \\ % chktex 2
	\cite{Yu2019}                                                & mRNA  & cancer               & {\gls{mlp}                                                                                      \\ \gls{cnn}} & \xmark & PCA \\  % chktex 2
	\cite{Ramirez2020}                                           & mRNA  & cancer               & \gls{gcn}                             & {PPI                                                    \\ co-exp} & Feature selection (threshold) \\  % chktex 2
	\cite{Ramirez2021}                                           & mRNA  & survival             & \gls{gcn}                             & {GeneMania                                              \\ co-exp} & Feature selection (threshold) \\ % chktex 2
	\cite{Xing2021}                                              & mRNA  & grade                & \gls{gcn}                             & co-exp     & Feature selection (DE)                     \\ % chktex 2
	\cite{Jiang2023}                                             & DNAm  & cancer               & \gls{gcn}                             & co-exp     & Feature selection (threshold)              \\ % chktex 2
	\cite{Baul2022}                                              & mRNA  & subtype              & \gls{gat}                             & Patient    & Feature selection (threshold)              \\ % chktex 2
	\cite{levyMethylSPWNetMethylCapsNetBiologically2021a}        & DNAm  & subtype              & \glsxtrshort{capsnet}                 & \xmark     & Feature selection                          \\ % chktex 2
	\cite{Khan2023}                                              & mRNA  & subtype              & Transformer                           & \xmark     & \xmark                                     \\ % chktex 2
	\cite{Lee2022}                                               & mRNA  & survival             & \gls{mlp} + attention                 & \xmark     & ?                                          \\ % chktex 2 chktex 26
	\cite{bourgeaisDeepGONetSelfexplainable2021}                 & mRNA  & cancer               & \gls{mlp}                             & \xmark     & Feature selection (GO)                     \\ % chktex 2
	\cite{haoPASNetPathwayassociatedSparse2018}                  & mRNA  & {LTS                                                                                                                   \\ non LTS} & \gls{mlp} & \xmark & Feature selection (REACTOME) \\ %chktex 2
	\cite{goreCancerNetUnifiedDeep2022}                          & DNAm  & cancer               & \gls{vae}                             & \xmark     & Average \(\beta\)-value in a 100 bp window \\ % chktex 2
	\cite{Ching2018}                                             & mRNA  & survival             & \gls{mlp}                             & \xmark     & \xmark                                     \\ %chktex 2
	\cite{katzmanDeepSurvPersonalizedTreatment2018}              & mRNA  & survival\TblrNote{a} & \gls{mlp}                             & \xmark     & \xmark                                     \\ %chktex 2
\end{longtblr}
%
	}{
		\begin{longtblr}[
	caption = {examples single omics},
	entry = {short caption},
	note{a} = {Used for treatment recommendation},
	]{
	colspec = {Q[c,m]Q[l,m]Q[c,m]Q[c,m, wd=2.5cm]Q[c,m,wd=2.2cm]Q[l,m, wd=3.2cm]},%
	hline{1,Z} = {2pt},%
			hline{2} = {1pt},%
			row{2-Z} = {font=\small},%
			rowhead = 1, %
			rowfoot = 0%
		}
	Ref.                                                         & Omics & Task                 & Arch.                                 & Graph      & Feature engineering                        \\ % chktex 2
	\cite{Arafa2023}                                             & mRNA  & sample               & \glsxtrshort{ae}                      & \xmark     & Feature selection                          \\ % chktex 2
	\cite{Wang2018}                                              & DNAm  & subtype              & \glsxtrshort{vae}                     & \xmark     & Feature selection (MAD)                    \\ % chktex 2
	\cite{Karim2019}                                             & CNV   & subtype              & \glsxtrshort{ae} + \glsxtrshort{cnn}  & \xmark     & Feature selection (oncogenes)              \\ % chktex 2
	\cite{Levy2020}                                              & DNAm  & cancer               & \glsxtrshort{vae} + \glsxtrshort{mlp} & \xmark     & Feature selection (MAD)                    \\ % chktex 2
	\cite{Kaczmarek2022}                                         & miRNA & sample               & \glsxtrshort{ae} + \glsxtrshort{mlp}  & \xmark     & \xmark                                     \\ % chktex 2
	\cite{Hanczar2018}                                           & mRNA  & binary               & \glsxtrshort{ae} + \glsxtrshort{mlp}  & \xmark     & \xmark                                     \\ % chktex 2
	\cite{Wojewodzic2021}                                        & DNAm  & sample               & \glsxtrshort{mlp}                     & \xmark     & Feature selection                          \\ % chktex 2
	\cite{Liu2019}                                               & DNAm  & sample               & \glsxtrshort{mlp}                     & \xmark     & Feature selection                          \\ % chktex 2
	\cite{gaoDeepCCNovelDeep2019}                                & mRNA  & subtype              & \glsxtrshort{mlp}                     & \xmark     & Feature selection (Knowledge)              \\ % chktex 2
	\cite{beykikhoshkDeepTRIAGEInterpretableIndividualised2020a} & mRNA  & subtype              & attention-based                       & \xmark     & Feature selection (Knowledge)              \\ % chktex 2
	\cite{Zhang2020}                                             & DNAm  & sample               & attention-based + \gls{ae}            & \xmark     & model-based selection                      \\ % chktex 2
	\cite{yuArchitecturesAccuracyArtificial2019b}                & mRNA  & {sample                                                                                                                \\ grade}        &   {\gls{mlp} \\ \gls{cnn}}                                    & \xmark     &  {LASSO \\ PCA}                                          \\ % chktex 2
	\cite{Divate2022}                                            & mRNA  & cancer               & \gls{mlp}                             & \xmark     & \xmark                                     \\ % chktex 2
	\cite{Elbashir2019}                                          & mRNA  & cancer               & \gls{cnn}                             & \xmark     & Coding genes + Feature selection           \\ % chktex 2
	\cite{deGuia2019}                                            & mRNA  & cancer               & \gls{cnn}                             & \xmark     & Feature selection (threshold)              \\ % chktex 2
	\cite{Wang2021}                                              & mRNA  & cancer               & \gls{cnn}                             & \xmark     & Feature selection (Mutual information)     \\ % chktex 2
	\cite{Mostavi2020}                                           & mRNA  & cancer               & \gls{cnn}                             & \xmark     & Feature selection (threshold)              \\ % chktex 2
	\cite{Lyu2018}                                               & mRNA  & cancer               & \gls{cnn}                             & \xmark     & Feature selection (threshold)              \\ % chktex 2
	\cite{LpezGarca2020}                                         & mRNA  & survival             & \gls{cnn}                             & \xmark     & Feature selection (MAD)                    \\ % chktex 2
	\cite{maOmicsMapNetTransformingOmics2019}                    & mRNA  & grade                & \gls{cnn}                             & \xmark     & Feature selection (threshold)              \\ % chktex 2
	\cite{Hao2018}                                               & mRNA  & survival             & \gls{mlp}                             & \xmark     & \xmark                                     \\ % chktex 2
	\cite{Chatterjee2018}                                        & DNAm  & cancer               & \gls{cnn}                             & \xmark     & \xmark                                     \\ % chktex 2
	\cite{Zhao2020}                                              & mRNA  & cancer               & \gls{cnn}                             & \xmark     & Feature selection (DE)                     \\ % chktex 2
	\cite{Mohammed2021}                                          & mRNA  & cancer               & \gls{cnn}                             & \xmark     & Feature selection (threshold + DE + LASSO) \\ % chktex 2
	\cite{Yin2022}                                               & mRNA  & survival             & \gls{cnn}                             & \xmark     & Feature selection (threshold + CWx)        \\ % chktex 2
	\cite{Yu2019}                                                & mRNA  & cancer               & {\gls{mlp}                                                                                      \\ \gls{cnn}} & \xmark & PCA \\  % chktex 2
	\cite{Ramirez2020}                                           & mRNA  & cancer               & \gls{gcn}                             & {PPI                                                    \\ co-exp} & Feature selection (threshold) \\  % chktex 2
	\cite{Ramirez2021}                                           & mRNA  & survival             & \gls{gcn}                             & {GeneMania                                              \\ co-exp} & Feature selection (threshold) \\ % chktex 2
	\cite{Xing2021}                                              & mRNA  & grade                & \gls{gcn}                             & co-exp     & Feature selection (DE)                     \\ % chktex 2
	\cite{Jiang2023}                                             & DNAm  & cancer               & \gls{gcn}                             & co-exp     & Feature selection (threshold)              \\ % chktex 2
	\cite{Baul2022}                                              & mRNA  & subtype              & \gls{gat}                             & Patient    & Feature selection (threshold)              \\ % chktex 2
	\cite{levyMethylSPWNetMethylCapsNetBiologically2021a}        & DNAm  & subtype              & \glsxtrshort{capsnet}                 & \xmark     & Feature selection                          \\ % chktex 2
	\cite{Khan2023}                                              & mRNA  & subtype              & Transformer                           & \xmark     & \xmark                                     \\ % chktex 2
	\cite{Lee2022}                                               & mRNA  & survival             & \gls{mlp} + attention                 & \xmark     & ?                                          \\ % chktex 2 chktex 26
	\cite{bourgeaisDeepGONetSelfexplainable2021}                 & mRNA  & cancer               & \gls{mlp}                             & \xmark     & Feature selection (GO)                     \\ % chktex 2
	\cite{haoPASNetPathwayassociatedSparse2018}                  & mRNA  & {LTS                                                                                                                   \\ non LTS} & \gls{mlp} & \xmark & Feature selection (REACTOME) \\ %chktex 2
	\cite{goreCancerNetUnifiedDeep2022}                          & DNAm  & cancer               & \gls{vae}                             & \xmark     & Average \(\beta\)-value in a 100 bp window \\ % chktex 2
	\cite{Ching2018}                                             & mRNA  & survival             & \gls{mlp}                             & \xmark     & \xmark                                     \\ %chktex 2
	\cite{katzmanDeepSurvPersonalizedTreatment2018}              & mRNA  & survival\TblrNote{a} & \gls{mlp}                             & \xmark     & \xmark                                     \\ %chktex 2
\end{longtblr}
%
	}

	With MethylSPWNet, \citeauthor{levyMethylSPWNetMethylCapsNetBiologically2021a} propose an architecture based on \gls{capsnet}, another architecture borrowed from the computer vision domain~\cite{levyMethylSPWNetMethylCapsNetBiologically2021a}.
	\Glspl{capsnet} are a type of architecture used to model hierarchical relationships, and a capsule correspond to a set of neurons whose activation vector represents a property of an object~\cite{CapsNet}.
	In the context of its application to \gls{dnam}, a capsules represents the set of CpGs that are link to a set of manually curated genes.
	Each context capsules is dynamically routed to one output capsule.

	For the application to omics data, deep learning architectures using attention mechanisms have been little explored.
	Attention mechanisms are a simple weighting of the data, various way of obtaining this weight have been proposed.
	\Glspl{fcn} can be used to score each feature~\cite{Lee2022,beykikhoshkDeepTRIAGEInterpretableIndividualised2020a}, the attention mechanism works more like a soft gate where only a part of the signal is transmitted to subsequent layers.
	This type of attention is different from the scaled-dot product attention that enrich the input vector with information from other features.
	The Gene transformer~\cite{Khan2023} was the first architecture to apply self-attention to \gls{mrna} data.
	The authors proposed to use 1D convolution layers combined with maximum pooling to reduce the dimension of the gene expression vector.
	Using a pooling layer is equivalent to a dimension reduction that does not consider all possible feature interactions.

	Precision medicine is not only interested in correctly diagnosing a patient but also in estimating its prognosis.
	The prognosis is estimated with survival analysis.
	Survival analysis focuses on analyzing and modeling the time it takes for an event of interest to occur.
	The event of interest can be death, relapse, or recovery, among others.
	A standard survival model is the Cox proportional hazards model.
	This standard model has been extended to consider non-linear survival data with deep learning models~\cite{katzmanDeepSurvPersonalizedTreatment2018,Ching2018}.
	Cox-nnet~\cite{Ching2018} uses a log-likelihood loss (\cref{eq:cox_loss}) to predict prognosis from gene expression data.
	Deep-surv~\cite{katzmanDeepSurvPersonalizedTreatment2018} uses the same principles to recommend the best treatment option for a patient by computing the risk ratio.
	\citeauthor{Lee2022} model the survival task as a regression on the survival days; this approach does not consider the particular structure of survival data and completely ignores censoring data.

\section{Prediction from multi-omics data}
	In the previous section, we saw that cancer types can be predicted from single omics; going to multi-omics data allows for a holistic view, which is necessary for a better understanding of the roles played by the various omics layers.
	In both cases, a sphere or a vertical cylinder viewed from the top looks like a circle.
	However, there is insufficient information to infer the 3D shape correctly; more information is needed to select from the possible hypothesis.
	Each view of a 3D object carries different information; they can be complementary or redundant and help improve predictions regarding performance and robustness.
	This is precisely what multimodal machine learning is: combining multiple views or modalities to make better predictions.
	In precision medicine and the realm of omics, without prior knowledge, a mutation's impact cannot be known without \textit{looking} at the protein.
	Furthermore, gene expression is a highly regulated phenomenon involving various molecular actors~(\cref{subsec:gene_regulation}).

	The following subsection will present multimodal machine learning, its properties, challenges, and methods to fuse multiple modalities.
	This description will be based on the work of \citeauthor{MML_morency}: \citetitle{MML_morency}~\cite{MML_morency}.
	Then, describe some machine learning techniques used to integrate multi-omics data.
	Finally, it presents the deep learning architectures used to integrate multi-omics data.

	\subsection{Multimodal machine learning}
		Multimodal machine learning corresponds to machine learning methods developed to learn from multiple modalities.
		Theoretical justification established that multimodal deep learning is better than unimodal~\cite{NEURIPS2021_5aa3405a}.
		A multimodal model can correspond to a model where the input is from one modality and the output from another modality (translation) or models that jointly learn from multiple modalities, the joint representation can later be used to predict an entire modality.
		A modality represents the way a phenomenon is expressed~\cite{MML_morency}.
		If an instance in one modality is explicitly linked to a corresponding instance in another modality, the modalities are paired.
		On the contrary, unpaired modalities correspond to situations without one-to-one correspondence between instances across modalities.
		Multi-omics data obtained by analyzing a sample from one patient are paired, but methods jointly measuring multiple modalities from the same cell are rare.
		Multi-omics single-cell data are often unpaired; this data type is obtained by measuring different cells from the same condition.
		In the context of this manuscript, a modality corresponds to an omics data type.
		Each modality has its own specific characteristics and distributions.
		Indeed, each omics is obtained using different techniques and processed with different bioinformatics tools, leading to data with different dimensionalities and distributions.
		The modality heterogeneity results from biological, as they measure different molecular components and technical reasons.
		Making the integration of these diverse data types a significant challenge.
		While heterogeneous, modalities are connected as they share complementary information.
		For instance, it should be noted that not all mRNAs are translated to proteins.
		Both mRNAs and proteins provide a complementary view of the underlying biological process.
		They also provide the same information due to their redundancy, as protein levels confirm the translation of mRNAs.
		Integrating multiple modalities allows interactions between modalities to happen and exploit redundant and complementary information.
		Multimodal models aim to learn a representation that reflects those cross-modal interactions.
		To construct a multimodal representation different fusion strategies are employed:
		\begin{description}[%
				style=multiline,
				leftmargin=!,
				labelwidth=2.5cm,
			]
			\item[Early fusion]
				The early fusion method combines the different omics in the data space; the resulting vector is analyzed like an unimodal input.
				This approach is well-studied due to its simplicity.
				However, it does not fully exploit the complementarity between omics and is known to be sensitive to the differences in distributions across omics.
			\item[Late fusion]
				For the late fusion approach, the fusion occurs in the prediction space.
				Each omics is processed separately with modality specific models, and the predictions are combined, similar to ensemble methods.
				Errors between modalities should not be correlated to ensure they have complementary effects.
				This approach does not necessarily capture the complex interactions between modalities.
			\item[Intermediate fusion]
				With intermediate fusion, a modality-specific encoder first processes each omics separately.
				Then, the features learned from each modality are combined in the latent space with specific fusion blocks to learn a joint representation before being fed to a multimodal classifier.
				The unimodal encoders can be jointly learned or pretrained.
		\end{description}
		Instead of fusing the latent representations of each modality, it is also possible to coordinate them.
		Coordination aims to ensure that the representations learned from each modality are compatible by constraining latent representations with representations of other modalities.
		The constraints are obtained by imposing similarity constraints, finding linear combinations of variables in each set that are maximally correlated with each other, or through contrastive learning.
		Contrastive learning forces the model to encode similar items closer together in the representation space while pushing dissimilar items further apart.

		The constructed multimodal representation must account for the interactions between modalities, \ie{}how to correctly align the different modalities.
		The alignment of modalities corresponds to knowing what features from one modality have an impact or are connected to another feature from a different modality.
		There are multiple ways of modeling this with multimodal machine learning models.
		Contrastive learning is a good way to find an alignment between multiple modalities by forcing similar concepts expressed in different modalities to match.
		Optimal transport is another way to align multiple modalities in a shared latent space by solving a divergence minimization problem, \ie{}minimizing the cost of transporting one distribution to another.
		Learning better multimodal representation can be achieved by contextualizing a modality's representation with the representation of another.
		The contextualization aims to capture the context in which the data appears by modeling all interactions between different modalities.
		For instance, a multimodal model dealing with gene expression data and \gls{dnam} data would capture how features relate to each other, \ie{}how a methylation site affect the expression of a gene.
		Such approach allows to capture more nuanced and complex interactions.
		Aligment layers based on the scaled-dot product attention are used to capture the alignment.
		The attention can be used in various ways to align a modality pair:
		\begin{description}[%
				style=multiline,
				leftmargin=!,
				labelwidth=3cm,
				%format=\labelsinglespace
			]
			\item[Early concatenation]
				The representation of two modalities \(X_A \in \symbb{R}^{L_A \times d}\) and \(X_B \in \symbb{R}^{L_B \times d}\) are concatenated and passed to a self-attention layer: \(\operatorname{Attention}\left(\symbfup{C}\left(X_A,X_B\right)\right)\), where \(\symbfup{C}\) is the concatenation operation.
				This early fusion technique learns undirected connections between modalities: all modality elements are connected to all other modality elements.
				This fusion approach considers both intra-modality and inter-modality interactions.
				While this fusion method can be adapted to any number of modalities, increasing the number of modalities increases the input dimension \(L'=\sum_i L_i\) of the attention layer, therefore increasing the computational complexity.
			\item[Hierarchical attention]
				Each modality is individually encoded with an attention based encoder, the resulting representation are concatenated and passed to another attention layer:
				\begingroup
				\setlength{\abovedisplayskip}{2pt}
				\setlength{\belowdisplayskip}{2pt}
				\setlength{\abovedisplayshortskip}{0pt}
				\setlength{\belowdisplayshortskip}{0pt}
				\begin{align*}
					Z_A & = \operatorname{Attention}\left(X_A\right)                              \\
					Z_B & = \operatorname{Attention}\left(X_B\right)                              \\
					Z   & = \operatorname{Attention}\left(\symbfup{C}\left(Z_A,Z_B\right)\right)
				\end{align*}
				Decoupling the learning of the intra-modality interactions from the inter-modality ones can help construct better individual representation, and modality encoders could be pre-trained.
				This approach can also be adapted to any number of modalities but faces the same limitations as early concatenation.
				\endgroup
			\item[Cross attention]
				This fusion approach is used to compute asymmetric interactions, a target modality is reinforced in a directed manner by a source modality.
				Usually, this methods is done in a bidirectional manner resulting in two contextualized representation accounting for the assymetric interactions: 
				\begingroup
				\setlength{\abovedisplayskip}{2pt}
				\setlength{\belowdisplayskip}{2pt}
				\setlength{\abovedisplayshortskip}{0pt}
				\setlength{\belowdisplayshortskip}{0pt}
				\begin{align*}
					Z_{A \rightarrow B} & = \operatorname{CA}\left(X_A, X_B \right) \\
					Z_{B \rightarrow A} & = \operatorname{CA}\left(X_B, X_A \right)
				\end{align*} % add a ref to equation
				\endgroup
				This fusion techniques can be adapted to any number \(m\) of modalities, but it requires to compute \(m\left(m-1\right)\) different cross-attentions.
				While this approach can capture cross-modal interactions between pairs of modality, it is not able to capture the whole multimodal context. 
				It has been proposed to concatenate the outputs of the different cross-attention and process the resulting embedding by another attention layer to capture the global context. % vilbert
		\end{description}
		The different approaches can be combined in a single architecture. 
		For instance, in the TriBERT architecture, three cross-attention are used to integrate three modalities. 
		For each cross-attention the query is obtained from one modality, and the key and value results from the concatenation of the two others modalities. 
		% https://www.biorxiv.org/content/10.1101/2024.02.26.582051v1.full.pdf
		% https://openaccess.thecvf.com/content/CVPR2023/papers/Wang_Multi-Modal_Learning_With_Missing_Modality_via_Shared-Specific_Feature_Modelling_CVPR_2023_paper.pdf
		% https://people.csail.mit.edu/khosla/papers/icml2011_ngiam.pdf

		In biology, connections between modalities are already known: an mRNA transcribed from a coding gene will likely be translated into a protein, and the impact of promoter methylation on the mRNA expression level is known.
		Moreover, the connections between individual features from different modalities are also known.
		For instance, connections from a coding gene and the corresponding protein are known, or CpGs methylation can be mapped to a gene based on their position in the genome.
		Adding this knowledge into the integration process could help construct better multimodal representation.
		Restricting the integration to features with known regulatory links would leave out many features, as a large part of \gls{ncrna} roles still need to be discovered.
		Even though their role is unknown, they can still play an essential role in gene expression regulation, particularly in cancer development.
		Using data-driven approaches, such as contextualized representations based on the attention mechanism, can help identify new connections and generate new knowledge.

		Access to paired multimodal data can be difficult during inference.
		Only some modalities will be generated to reduce costs, or a patient may refuse some analysis.
		A solution would be to use methods that exploit paired multi-omics datasets during training, and inference is done using only one omics. 
		With co-learning, all available modalities are used to construct a joint or coordinated multimodal representation. 
		Adding more modalities during the training allows for the transfer of information and enriches the multimodal representation.
		While paired multimodal datasets are the ideal scenario, the availability of such datasets remains limited.
		Many multi-omics datasets contain samples where one or more modalities are missing, and with single-cell, many of the available multi-omics datasets are unpaired. 
		New methods that can adapt to missing modalities, or perform diagonal integration of unpaired datasets are required to exploit the currently available datasets. 

		An important challenge not specific to multimodal machine learning is the interpretability aspects of the model predictions.
		A better understanding of multimodal models is critical to gain insights into multimodal learning and help improve the model design.
		In a multimodal setting, modality and multimodal-level interpretability are essential, in addition to feature-level interpretability. 
		At the modality level, it is interesting to know how modalities are used to make a prediction and their individual contribution to the prediction.
		Indeed, in multimodal training, modalities compete with each other as they learn at different rates, only a subset of modalities is explored by the network~\cite{pmlr-v162-huang22e}. 
		Detecting modalities that hinder model performances is essential as adding more modalities does not necessarily mean adding more information; sometimes, it adds noise and reduces performance.
		With multiple modalities, one important interest consists in knowing the type of interaction between modalities: additive, \( w_A\cdot g_A\left(X_A\right) + w_B\cdot g_B\left(X_B\right)\),  or multiplicative, \(w\cdot  g\left( X_A \times X_B\right)\), where \(g\) is a model function or the identity function.
		Models using additive interactions do not use cross-modal information but only the unimodal information. 
		While complex models are able to learn complex multiplicative interactions, it is essential to ensure that a complex models is not learning simpler additive interactions. 
		To detect models not using multiplicative fusion, \citeauthor{EMAP} proposed to approximate a multimodal model \(g\left(X_A,X_B\right)\) by a multimodal additive model: \(\hat{g}\left(X_A,X_B\right) = \hat{g}_A\left(X_A\right) + \hat{g}_B\left(X_B\right) - \hat{\mu}\).
		Perfomance degradation from the additive model \(\hat{g}\) indicates that the model \(g\) did learn cross-modal interactions. 
		Approaches to understanding multimodal deep learning models can be categorized into approaches producing post-hoc explanations or interpretable models by design.
		Post-hoc explainability of multimodal models uses attribution methods, such as Grad-CAM~\cite{Chandrasekaran2018DoEM} or LRP~\cite{Ellis2021}, to understand modality importance. % todo add examples of usage
		Model weights can be directly used to detect feature interactions~\cite{tsang2017detecting}.
		In DIME~\cite{DIME}, authors propose to estimate the different contributions with LIME, a widely used interpretation approach~\cite{lime}.
		They decompose a model \(g\) into a sum of two terms: unimodal contributions \(\operatorname{UC}\) and multimodal interactions \(\operatorname{MI}\). 
		Then, LIME is applied on this two newly defined models perturbing only one modality at a time to obtain the different explanation: the unimodal contribution, \(\operatorname{UC}_A\) and \(\operatorname{UC}_B\), and the contribution of modality \(A\) to the multimodal interactions, \(\operatorname{MI}_A\) and reciprocally the contribution of modality \(B\) to the multimodal interactions \(\operatorname{MI}_A\).
		Perceptual scores~\cite{PerceptualScores} have been proposed as a way to detect on which modality a model relies on for the predicitons.  
		The perceptual score towards a modality \(A\), \(\symcal{P}_A\), is the normalized expectation of sample perceptual scores: \(\symcal{P}_A = \frac{1}{nz}\sum_{i=1}^{N}\symcal{P}_{x_i}\), where the sample perceptual score \(\symcal{P}_{x_i}\) correspond to the difference of accuracy between the model using all modalities and the model the model that do not use the modality \(A\) and \(z\) is the normalization factor.
		In~\cite{SHAPE}, authors propose to use game theory and Shapley values, \(\phi_i\), to estimate the marginal contribution and the cooperation of individual modalities.
		In the Shapley framework, individuals modalities are considered as players. 
		The marignal contirbution of a modality \(m\) corresponds to its scaled Shapley value, \(\symbfcal{S}_m = \frac{1}{z}\phi_m\) and the modality cooperation of a set of modalities \(\symbfcal{A}\),  corresponds to the Shapley value of the set of modalities where the individual modality contributions are substracted, \(\symbfcal{C}_{\symbfcal{A}} = \phi_{\symbfcal{A}} - \sum_{m \in \symbfcal{A}}\phi_m\).
		A new approach based on information decomposition quantifies the degree of redundancy, uniqueness, and synergy between input modalities and a specific task~\cite{liang2023multimodal,liang2023quantifying}. 
		Post-hoc approaches, while model-agnostic, are computationally expensive as they estimate the different contributions by substituting part of the input.
		The cost to run those methods increases with the number of modalities.
		Models interpretable by design exploit new blocks, such as attention, to quantify modality importance or cross-modal interactions.
		The PJ-X architecture~\cite{8579013} uses a block to produce an attention map that highlights the important parts of an image in a visual-question-answering model.
		Graph-based fusion explicitly models modality fusion as a directed graph where each node represents a modality or a fusion of modalities. 
		A learnable weight is associated with each edge representing the modality importance in the fused representation~\cite{bagher-zadeh-etal-2018-multimodal}.

		%file:///home/aurel/T%C3%A9l%C3%A9chargements/s41467-022-31104-x.pdf
		%file:///home/aurel/T%C3%A9l%C3%A9chargements/s41592-024-02391-7.pdf
		%https://www.nature.com/articles/s41467-023-43019-2
		%https://cdn.aaai.org/ojs/16330/16330-13-19824-1-2-20210518.pdf


	\subsection{Multi-omics data integration}
		different type of integration N vs P
		An N−integration type of question
		(several data sets, supervised or unsupervised)
		A P−integration type of question (several studies of the same omic type, supervised or unsupervised)

		Canonical Correlation Analysis (CCA) for 2 omics only
		DIABLO more omics ?
		unsupervised vs supervised integration ?
		different methods NMF, bayesian, network step

		omics challenges
		Different omics rely on different laboratory techniques and data extraction platforms, resulting in data sets of different formats, complexity, dimensionalities, information content, and scale, and may be processed using different bioinformatics tools. Therefore, data heterogeneity arises from biological and technical reasons and is the main analytical challenge to overcome.

		Matrix factorisation techniques, where large data sets are decomposed into smaller sub-matrices to summarise information. These techniques use algebra and analysis to optimise specific statistical criteria and integrate different levels of information. Methods in mixOmics fit into this category and will be detailed in Chapter 3 and subsequent chapters, • Bayesian methods, which use assumptions of prior distributions for each omics type to find correlations between data layers and infer posterior distributions, • Network-based approaches, which use visual and symbolic representations of biological systems, with nodes representing molecules and edges as correlations between molecules, if they exist. Network-based methods are mostly applied for detecting significant genes within pathways, discovering sub-clusters, or finding co-expression network modules, • Multiple-step approaches that first analyse each single omics data set individually before combining the results based on their overlap (e.g. at the gene level of a molecular signature) or correlation. This type of approach technically deviates from data integration but is commonly used.
		% https://www.nature.com/articles/s41467-020-20430-7
		show some example on omics data

		motivate use of deep learning methods
		multimodal DL is able to model nonlinear withinand cross-modality relationships
	\subsection{Deep learning models for multi-omics data}
		\subsubsection{Early fusion} This kind of architectures
		\subsubsection{Late fusion}
		\subsubsection{Intermediate fusion}
			organise by type of integration
			start with early fusion, simplest
			go to late fusion
			finish with intermediate fusion



\section{Model interpretability}
	% general about interpretability/explanability
	% fix the vocabulary that is used
	% present different methods, advantage drawbacks

	% way to introduce counterfactuals

	% remind what are counterfactuals in plain english: what if
	% show equation
	% desired property of CFs + construct the optimization problem
	% how to solve it: gradient descent show algo
	% discuss lambda
	% show some methods

	% need to talk about adversarial examples, details in annex?
	% how to have a model robust to adv attacks, defenses, adv training

\end{document}
