\documentclass[../main.tex]{subfiles}
\usepackage{silence}
\WarningFilter{glossaries}{No \printglossary or \printglossaries found}

\begin{document}
\ifSubfilesClassLoaded{%
	\graphicspath{{figures/3-SOTA/}}%
	\setcounter{chapter}{2}%
}{
	\graphicspath{{../figures/3-SOTA/}}%
}
\chapter{Deep-learning for omics data analysis}\label{chap:sota}
\minitocpagecentered

\section{Prediction from single omics data}
	With the application of deep-learning to omics data multiple tasks are possible: predict gene expression from the DNA sequence~\cite{Avsec2021}, predict the impact of a sequence variation on \glsxtrlong{dnam}~\cite{Zeng2017}, annotate genome variants~\cite{Quang2014}, etc\@.
	In this document, we will focus on tasks predicting a phenotype: does a patient have a cancer or not, what type of cancer a patient has, or survival-related tasks.

	Omics data suffers from a curse of dimensionality: they are high-dimensional but the number of available examples remains limited.
	To overcome the limitation of training examples, various strategies have been used: self-supervised training, feature engineering, transfer learning.
	\Glspl{ae} or \glspl{vae} are used to learn a compact representation of the input; the latent representation should contain relevant biological features.
	The latent representation can be fed to a downstream classifier such as \gls{svm}~\cite{Zhang2020}, \gls{xgboost} or k-NN classifier~\cite{Arafa2023} logistic regression~\cite{Wang2018} or used to train a neural network classifier~\cite{Karim2019}.
	The encoder part of the \gls{ae} can be fine-tuned on a predictive task through transfer learning~\cite{Levy2020,Kaczmarek2022}.
	\citeauthor{Hanczar2018} pretrain each layer of an \gls{mlp} with a denoising \gls{ae}~\cite{Hanczar2018}.
	Feature engineering involves the creation, selection, and transformation of features to enhance the model's ability to capture essential patterns or signals hidden within the omics data.
	Many works use feature selection to reduce the input dimension of neural networks.
	The selections process can be based on features with the highest median absolute deviation, based on a student t-test selection~\cite{Liu2019}, based on knowledge on important features~\cite{Kaczmarek2022} or based on another models such as \glspl{rf}~\cite{Wojewodzic2021,Liu2019} or LASSO~\cite{Liu2019} or PCA~\cite{Yu2019}.
	Other approaches use the available knowledge to transform raw features to annotated biologically relevant features.
	DeepCC~\cite{gaoDeepCCNovelDeep2019} applies an \gls{mlp} on biologically informed features by transforming gene expression data into a functional spectrum, \ie{}a list of enrichment scores calculated by gene set enrichment analysis.
	DeepTRIAGE~\cite{beykikhoshkDeepTRIAGEInterpretableIndividualised2020a} transforms gene counts to pathways counts by summing all gene counts belonging to a gene ontology biological process pathway or a KEGG pathway.
	\citeauthor{Zhang2020} use a gated \gls{mlp} to identify a set of important features, those features are then used to train an \gls{ae}~\cite{Zhang2020}.

	Using multiple models to select features and perform predictive tasks results in increased complexity, and the process of selecting features before constructing the predictive models limits the capacity to consider all feature interactions.
	On the contrary, deep learning architectures can adapt to any number of features and select the most relevant features through their architectures.
	They allow modeling complex, non-linear relationships and have been shown to outperform classical machine learning approaches with training sets of a few thousand examples~\cite{Hanczar2022}.

	\Gls{mlp} are well studied for predicting outcomes from omics data.
	\citeauthor{yuArchitecturesAccuracyArtificial2019b}, explored different \gls{mlp} architectures by varying the number of neurons in each layer and the number of layers~\cite{yuArchitecturesAccuracyArtificial2019b}.
	They showed that wider networks perform better than deeper ones.
	Cancer types are accurately predicted from gene expression data with an \gls{mlp}~\cite{Divate2022}.
	Adding inductive bias can help the model better generalize when training data is limited.
	Inductive bias corresponds to the assumptions and prior knowledge incorporated into the model to better generalize.
	Introducing biological knowledge into the architecture is a robust inductive bias that should help the model better generalize.
	Biological system and their knowledge representation are hierarchical, making this kind of inductive bias suitable for hierarchical architecture such as \gls{mlp}~\cite{bourgeaisDeepGONetSelfexplainable2021,haoPASNetPathwayassociatedSparse2018}.
	In DeepGONet~\cite{bourgeaisDeepGONetSelfexplainable2021}, authors constrain the \gls{mlp} architecture with the GO biological processes hierarchy.
	In PASNet~\cite{haoPASNetPathwayassociatedSparse2018,Hao2018}, authors use the REACTOME hierarchy to constrain their \gls{mlp} architecture.
	With such architectures, each neuron of a hidden layer represents a known pathway, and only neurons with known relations are connected.
	For the input layer, neurons represent genes, and they are connected to neurons of the first hidden layers if they are involved in the corresponding pathway.
	In practice, this is achieved by masking the unknown relationships in a fully connected layer:
	\begin{equation}
		\symbf{a}^{l} = f\left( \symbf{a}^{l-1} \cdot \left( W^l \odot M^l\right) + \symbf{b}^{l} \right) \label{eq:mlp_sparse_matrix}
	\end{equation}
	where \(M^l\) is a binary mask restricting connections between neurons to known biological relations and \(\odot\) is the Hadamard product or element-wise product: \({\left(A\odot B\right)}_{ij} = {\left(A\right)}_{ij}{\left(B\right)}_{ij}\).

	Earlier, we presented approaches that trained an \gls{ae} and used the latent representation as input to another classification model.
	However, deep learning is a flexible approach and a supervision of the latent space from a classifier can be added during the \gls{ae} training.
	Such approach allow for an end-to-end training of the model and constrains the latent space to learn relevant features for the classification~\cite{goreCancerNetUnifiedDeep2022}.
	For such models, the training loss correspond to the sum of the unsupervised loss and the supervised loss: \(\symcal{L}_{AE} + \symcal{L}_{CE}\).

	Promising results of \gls{cnn} in computer vision inspired its application in precision medicine.
	However, expression profile are not 2D like images but 1D.
	To apply 2D convolutions onto expression profile a transformation step to go from a 1D representation to a 2D representation: \(\symcal{T}: \symbb{R}^{d} \mapsto \symbb{R}^{h\times w}\).
	For such transformation, a zero-padding of the 1D vector is required for the transformation to be applied.
	The simplest transformation often corresponds to a naive reshaping~\cite{Wang2021,deGuia2019,Elbashir2019,Chatterjee2018}\@. \citeauthor{deGuia2019}, transform the gene expression vector to a \(102\times 102\) image after a feature selection process~\cite{deGuia2019}.
	Similarly, \citeauthor{Elbashir2019} reshape the gene expression vector to a \(127 \times 114\) image~\cite{Elbashir2019}.
	Some approaches hypothetize that chromosical adjacent genes are more likely to interact.
	Based on that assumptions, features are reordered based on their poistion in the genome before reshaping the expression to a 2D image~\cite{Mostavi2020,Lyu2018,Yin2022}.
	Instead of naive reshaping, knowledge-based transformation have been depicted.
	Hierarchical tree structure such KEGG BRITE are used to transform features, then the tree structure is represented as treemap with a tiling algorithm~\cite{LpezGarca2020,maOmicsMapNetTransformingOmics2019}.
	The chosen algorithm ensures that all samples share the same spatial arrangements by ordering genes based on their mean level of expression across the dataset.
	The treemap image are then colorized based on the gene expression abundances.
	Instead of forcing a 2D representation of expression profile, 1D convolution can be directly used.
	Similarly to 2D \gls{cnn}, features were reordered based on their genomic position~\cite{Mostavi2020,Zhao2020,Yin2022} or expression profile were leave untouched~\cite{Mohammed2021}.
	While many approaches around convolutions were developped they are not suited for omics data as they introduce wrong inductive biases.
	Convolutions are more suited for structured data, and even the incorporation of a structure in the data based on the chromosomal location or some knowledge is a constraint that limits the range of possible interactions.
	For instance such architectures do not consider interactions between genes that are far away from each other in the genome.

	% GNN
	\Gls{gnn} are architecture used to process graphs structures, that impose constraints on relationships between node entities during the learning process.
	In biology, many interactions between molecular entities are known and are represented as graphs.
	\Gls{gnn} are well suited to study omics data by introducing relationships inductive bias in the learning process.
	\citeauthor{Ramirez2020} used a \gls{gcn} based on a \gls{ppi} graph or a co-expression graph to predict cancer types from gene expression data~\cite{Ramirez2020}.
	The co-expression graph is obtained by computing a correlation matrix between genes, then a threshold is applied on the correlation values to construct the adjacency matrix.
	All patients share the same graphs, predictions are done at the graph level, and nodes represents genes.
	The expression profile of each patient is mapped onto the corresponding nodes.
	Information from neighboring nodes is propagated with \gls{chebconv}~\cite{ChebConv}.
	The thresholding of the correlation matrix, or the use of knowledge based graphs might introduce singletons nodes, \ie{}nodes that are not connected to any nodes.
	The authors tested the impact of the singleton nodes on the accuracy.
	The presence of singleton nodes only increase the accurcy with the \gls{ppi} graph.
	Authors extended their work to survival prediction~\cite{Ramirez2021} with a \gls{gcn} based on a co-expression graph, a gene interaction graph from GeneMania~\cite{WardeFarley2010} or a graph combining the two previous graphs.
	The inclusion of clinical information in the last layer improved the performances of their model.
	\Glspl{gat} have also been used to propagate information from neighboring nodes~\cite{Xing2021}.
	\Gls{sag} have been used in combination of \glspl{gcn} to extract differentially expressed methylation sites when predicting cancer types~\cite{Jiang2023}.
	Node level predictions have also been tested with patient similarity graphs~\cite{Baul2022}.
	Each node, \ie{}patient, is represented by its gene expression profile and nodes are connected based on their similarity.

	\ifSubfilesClassLoaded{%
		\begin{longtblr}[
	caption = {examples single omics},
	entry = {short caption},
	note{a} = {Used for treatment recommendation},
	]{
	colspec = {Q[c,m]Q[l,m]Q[c,m]Q[c,m, wd=2.5cm]Q[c,m,wd=2.2cm]Q[l,m, wd=3.2cm]},%
	hline{1,Z} = {2pt},%
			hline{2} = {1pt},%
			row{2-Z} = {font=\small},%
			rowhead = 1, %
			rowfoot = 0%
		}
	Ref.                                                         & Omics & Task                 & Arch.                                 & Graph      & Feature engineering                        \\ % chktex 2
	\cite{Arafa2023}                                             & mRNA  & sample               & \glsxtrshort{ae}                      & \xmark     & Feature selection                          \\ % chktex 2
	\cite{Wang2018}                                              & DNAm  & subtype              & \glsxtrshort{vae}                     & \xmark     & Feature selection (MAD)                    \\ % chktex 2
	\cite{Karim2019}                                             & CNV   & subtype              & \glsxtrshort{ae} + \glsxtrshort{cnn}  & \xmark     & Feature selection (oncogenes)              \\ % chktex 2
	\cite{Levy2020}                                              & DNAm  & cancer               & \glsxtrshort{vae} + \glsxtrshort{mlp} & \xmark     & Feature selection (MAD)                    \\ % chktex 2
	\cite{Kaczmarek2022}                                         & miRNA & sample               & \glsxtrshort{ae} + \glsxtrshort{mlp}  & \xmark     & \xmark                                     \\ % chktex 2
	\cite{Hanczar2018}                                           & mRNA  & binary               & \glsxtrshort{ae} + \glsxtrshort{mlp}  & \xmark     & \xmark                                     \\ % chktex 2
	\cite{Wojewodzic2021}                                        & DNAm  & sample               & \glsxtrshort{mlp}                     & \xmark     & Feature selection                          \\ % chktex 2
	\cite{Liu2019}                                               & DNAm  & sample               & \glsxtrshort{mlp}                     & \xmark     & Feature selection                          \\ % chktex 2
	\cite{gaoDeepCCNovelDeep2019}                                & mRNA  & subtype              & \glsxtrshort{mlp}                     & \xmark     & Feature selection (Knowledge)              \\ % chktex 2
	\cite{beykikhoshkDeepTRIAGEInterpretableIndividualised2020a} & mRNA  & subtype              & attention-based                       & \xmark     & Feature selection (Knowledge)              \\ % chktex 2
	\cite{Zhang2020}                                             & DNAm  & sample               & attention-based + \gls{ae}            & \xmark     & model-based selection                      \\ % chktex 2
	\cite{yuArchitecturesAccuracyArtificial2019b}                & mRNA  & {sample                                                                                                                \\ grade}        &   {\gls{mlp} \\ \gls{cnn}}                                    & \xmark     &  {LASSO \\ PCA}                                          \\ % chktex 2
	\cite{Divate2022}                                            & mRNA  & cancer               & \gls{mlp}                             & \xmark     & \xmark                                     \\ % chktex 2
	\cite{Elbashir2019}                                          & mRNA  & cancer               & \gls{cnn}                             & \xmark     & Coding genes + Feature selection           \\ % chktex 2
	\cite{deGuia2019}                                            & mRNA  & cancer               & \gls{cnn}                             & \xmark     & Feature selection (threshold)              \\ % chktex 2
	\cite{Wang2021}                                              & mRNA  & cancer               & \gls{cnn}                             & \xmark     & Feature selection (Mutual information)     \\ % chktex 2
	\cite{Mostavi2020}                                           & mRNA  & cancer               & \gls{cnn}                             & \xmark     & Feature selection (threshold)              \\ % chktex 2
	\cite{Lyu2018}                                               & mRNA  & cancer               & \gls{cnn}                             & \xmark     & Feature selection (threshold)              \\ % chktex 2
	\cite{LpezGarca2020}                                         & mRNA  & survival             & \gls{cnn}                             & \xmark     & Feature selection (MAD)                    \\ % chktex 2
	\cite{maOmicsMapNetTransformingOmics2019}                    & mRNA  & grade                & \gls{cnn}                             & \xmark     & Feature selection (threshold)              \\ % chktex 2
	\cite{Hao2018}                                               & mRNA  & survival             & \gls{mlp}                             & \xmark     & \xmark                                     \\ % chktex 2
	\cite{Chatterjee2018}                                        & DNAm  & cancer               & \gls{cnn}                             & \xmark     & \xmark                                     \\ % chktex 2
	\cite{Zhao2020}                                              & mRNA  & cancer               & \gls{cnn}                             & \xmark     & Feature selection (DE)                     \\ % chktex 2
	\cite{Mohammed2021}                                          & mRNA  & cancer               & \gls{cnn}                             & \xmark     & Feature selection (threshold + DE + LASSO) \\ % chktex 2
	\cite{Yin2022}                                               & mRNA  & survival             & \gls{cnn}                             & \xmark     & Feature selection (threshold + CWx)        \\ % chktex 2
	\cite{Yu2019}                                                & mRNA  & cancer               & {\gls{mlp}                                                                                      \\ \gls{cnn}} & \xmark & PCA \\  % chktex 2
	\cite{Ramirez2020}                                           & mRNA  & cancer               & \gls{gcn}                             & {PPI                                                    \\ co-exp} & Feature selection (threshold) \\  % chktex 2
	\cite{Ramirez2021}                                           & mRNA  & survival             & \gls{gcn}                             & {GeneMania                                              \\ co-exp} & Feature selection (threshold) \\ % chktex 2
	\cite{Xing2021}                                              & mRNA  & grade                & \gls{gcn}                             & co-exp     & Feature selection (DE)                     \\ % chktex 2
	\cite{Jiang2023}                                             & DNAm  & cancer               & \gls{gcn}                             & co-exp     & Feature selection (threshold)              \\ % chktex 2
	\cite{Baul2022}                                              & mRNA  & subtype              & \gls{gat}                             & Patient    & Feature selection (threshold)              \\ % chktex 2
	\cite{levyMethylSPWNetMethylCapsNetBiologically2021a}        & DNAm  & subtype              & \glsxtrshort{capsnet}                 & \xmark     & Feature selection                          \\ % chktex 2
	\cite{Khan2023}                                              & mRNA  & subtype              & Transformer                           & \xmark     & \xmark                                     \\ % chktex 2
	\cite{Lee2022}                                               & mRNA  & survival             & \gls{mlp} + attention                 & \xmark     & ?                                          \\ % chktex 2 chktex 26
	\cite{bourgeaisDeepGONetSelfexplainable2021}                 & mRNA  & cancer               & \gls{mlp}                             & \xmark     & Feature selection (GO)                     \\ % chktex 2
	\cite{haoPASNetPathwayassociatedSparse2018}                  & mRNA  & {LTS                                                                                                                   \\ non LTS} & \gls{mlp} & \xmark & Feature selection (REACTOME) \\ %chktex 2
	\cite{goreCancerNetUnifiedDeep2022}                          & DNAm  & cancer               & \gls{vae}                             & \xmark     & Average \(\beta\)-value in a 100 bp window \\ % chktex 2
	\cite{Ching2018}                                             & mRNA  & survival             & \gls{mlp}                             & \xmark     & \xmark                                     \\ %chktex 2
	\cite{katzmanDeepSurvPersonalizedTreatment2018}              & mRNA  & survival\TblrNote{a} & \gls{mlp}                             & \xmark     & \xmark                                     \\ %chktex 2
\end{longtblr}
%
	}{
		\begin{longtblr}[
	caption = {examples single omics},
	entry = {short caption},
	note{a} = {Used for treatment recommendation},
	]{
	colspec = {Q[c,m]Q[l,m]Q[c,m]Q[c,m, wd=2.5cm]Q[c,m,wd=2.2cm]Q[l,m, wd=3.2cm]},%
	hline{1,Z} = {2pt},%
			hline{2} = {1pt},%
			row{2-Z} = {font=\small},%
			rowhead = 1, %
			rowfoot = 0%
		}
	Ref.                                                         & Omics & Task                 & Arch.                                 & Graph      & Feature engineering                        \\ % chktex 2
	\cite{Arafa2023}                                             & mRNA  & sample               & \glsxtrshort{ae}                      & \xmark     & Feature selection                          \\ % chktex 2
	\cite{Wang2018}                                              & DNAm  & subtype              & \glsxtrshort{vae}                     & \xmark     & Feature selection (MAD)                    \\ % chktex 2
	\cite{Karim2019}                                             & CNV   & subtype              & \glsxtrshort{ae} + \glsxtrshort{cnn}  & \xmark     & Feature selection (oncogenes)              \\ % chktex 2
	\cite{Levy2020}                                              & DNAm  & cancer               & \glsxtrshort{vae} + \glsxtrshort{mlp} & \xmark     & Feature selection (MAD)                    \\ % chktex 2
	\cite{Kaczmarek2022}                                         & miRNA & sample               & \glsxtrshort{ae} + \glsxtrshort{mlp}  & \xmark     & \xmark                                     \\ % chktex 2
	\cite{Hanczar2018}                                           & mRNA  & binary               & \glsxtrshort{ae} + \glsxtrshort{mlp}  & \xmark     & \xmark                                     \\ % chktex 2
	\cite{Wojewodzic2021}                                        & DNAm  & sample               & \glsxtrshort{mlp}                     & \xmark     & Feature selection                          \\ % chktex 2
	\cite{Liu2019}                                               & DNAm  & sample               & \glsxtrshort{mlp}                     & \xmark     & Feature selection                          \\ % chktex 2
	\cite{gaoDeepCCNovelDeep2019}                                & mRNA  & subtype              & \glsxtrshort{mlp}                     & \xmark     & Feature selection (Knowledge)              \\ % chktex 2
	\cite{beykikhoshkDeepTRIAGEInterpretableIndividualised2020a} & mRNA  & subtype              & attention-based                       & \xmark     & Feature selection (Knowledge)              \\ % chktex 2
	\cite{Zhang2020}                                             & DNAm  & sample               & attention-based + \gls{ae}            & \xmark     & model-based selection                      \\ % chktex 2
	\cite{yuArchitecturesAccuracyArtificial2019b}                & mRNA  & {sample                                                                                                                \\ grade}        &   {\gls{mlp} \\ \gls{cnn}}                                    & \xmark     &  {LASSO \\ PCA}                                          \\ % chktex 2
	\cite{Divate2022}                                            & mRNA  & cancer               & \gls{mlp}                             & \xmark     & \xmark                                     \\ % chktex 2
	\cite{Elbashir2019}                                          & mRNA  & cancer               & \gls{cnn}                             & \xmark     & Coding genes + Feature selection           \\ % chktex 2
	\cite{deGuia2019}                                            & mRNA  & cancer               & \gls{cnn}                             & \xmark     & Feature selection (threshold)              \\ % chktex 2
	\cite{Wang2021}                                              & mRNA  & cancer               & \gls{cnn}                             & \xmark     & Feature selection (Mutual information)     \\ % chktex 2
	\cite{Mostavi2020}                                           & mRNA  & cancer               & \gls{cnn}                             & \xmark     & Feature selection (threshold)              \\ % chktex 2
	\cite{Lyu2018}                                               & mRNA  & cancer               & \gls{cnn}                             & \xmark     & Feature selection (threshold)              \\ % chktex 2
	\cite{LpezGarca2020}                                         & mRNA  & survival             & \gls{cnn}                             & \xmark     & Feature selection (MAD)                    \\ % chktex 2
	\cite{maOmicsMapNetTransformingOmics2019}                    & mRNA  & grade                & \gls{cnn}                             & \xmark     & Feature selection (threshold)              \\ % chktex 2
	\cite{Hao2018}                                               & mRNA  & survival             & \gls{mlp}                             & \xmark     & \xmark                                     \\ % chktex 2
	\cite{Chatterjee2018}                                        & DNAm  & cancer               & \gls{cnn}                             & \xmark     & \xmark                                     \\ % chktex 2
	\cite{Zhao2020}                                              & mRNA  & cancer               & \gls{cnn}                             & \xmark     & Feature selection (DE)                     \\ % chktex 2
	\cite{Mohammed2021}                                          & mRNA  & cancer               & \gls{cnn}                             & \xmark     & Feature selection (threshold + DE + LASSO) \\ % chktex 2
	\cite{Yin2022}                                               & mRNA  & survival             & \gls{cnn}                             & \xmark     & Feature selection (threshold + CWx)        \\ % chktex 2
	\cite{Yu2019}                                                & mRNA  & cancer               & {\gls{mlp}                                                                                      \\ \gls{cnn}} & \xmark & PCA \\  % chktex 2
	\cite{Ramirez2020}                                           & mRNA  & cancer               & \gls{gcn}                             & {PPI                                                    \\ co-exp} & Feature selection (threshold) \\  % chktex 2
	\cite{Ramirez2021}                                           & mRNA  & survival             & \gls{gcn}                             & {GeneMania                                              \\ co-exp} & Feature selection (threshold) \\ % chktex 2
	\cite{Xing2021}                                              & mRNA  & grade                & \gls{gcn}                             & co-exp     & Feature selection (DE)                     \\ % chktex 2
	\cite{Jiang2023}                                             & DNAm  & cancer               & \gls{gcn}                             & co-exp     & Feature selection (threshold)              \\ % chktex 2
	\cite{Baul2022}                                              & mRNA  & subtype              & \gls{gat}                             & Patient    & Feature selection (threshold)              \\ % chktex 2
	\cite{levyMethylSPWNetMethylCapsNetBiologically2021a}        & DNAm  & subtype              & \glsxtrshort{capsnet}                 & \xmark     & Feature selection                          \\ % chktex 2
	\cite{Khan2023}                                              & mRNA  & subtype              & Transformer                           & \xmark     & \xmark                                     \\ % chktex 2
	\cite{Lee2022}                                               & mRNA  & survival             & \gls{mlp} + attention                 & \xmark     & ?                                          \\ % chktex 2 chktex 26
	\cite{bourgeaisDeepGONetSelfexplainable2021}                 & mRNA  & cancer               & \gls{mlp}                             & \xmark     & Feature selection (GO)                     \\ % chktex 2
	\cite{haoPASNetPathwayassociatedSparse2018}                  & mRNA  & {LTS                                                                                                                   \\ non LTS} & \gls{mlp} & \xmark & Feature selection (REACTOME) \\ %chktex 2
	\cite{goreCancerNetUnifiedDeep2022}                          & DNAm  & cancer               & \gls{vae}                             & \xmark     & Average \(\beta\)-value in a 100 bp window \\ % chktex 2
	\cite{Ching2018}                                             & mRNA  & survival             & \gls{mlp}                             & \xmark     & \xmark                                     \\ %chktex 2
	\cite{katzmanDeepSurvPersonalizedTreatment2018}              & mRNA  & survival\TblrNote{a} & \gls{mlp}                             & \xmark     & \xmark                                     \\ %chktex 2
\end{longtblr}
%
	}

	With MethylSPWNet, \citeauthor{levyMethylSPWNetMethylCapsNetBiologically2021a} propose an architecture based on \gls{capsnet}, another architecture borrowed from the computer vision domain~\cite{levyMethylSPWNetMethylCapsNetBiologically2021a}.
	\Glspl{capsnet} are a type of architecture used to model hierarchical relationships, and a capsule correspond to a set of neurons whose activation vector represents a property of an object~\cite{CapsNet}.
	In the context of its application to \gls{dnam}, a capsules represents the set of CpGs that are link to a set of manually curated genes.
	Each context capsules is dynamically routed to one output capsule.

	For the application to omics data, deep learning architectures using attention mechanisms have been little explored.
	Attention mechanisms are a simple weighting of the data, various way of obtaining this weight have been proposed.
	\Glspl{fcn} can be used to score each feature~\cite{Lee2022,beykikhoshkDeepTRIAGEInterpretableIndividualised2020a}, the attention mechanism works more like a soft gate where only a part of the signal is transmitted to subsequent layers.
	This type of attention is different from the scaled-dot product attention that enrich the input vector with information from other features.
	The Gene transformer~\cite{Khan2023} was the first architecture to apply self-attention to \gls{mrna} data.
	The authors proposed to use 1D convolution layers combined with maximum pooling to reduce the dimension of the gene expression vector.
	Using a pooling layer is equivalent to a dimension reduction that does not consider all possible feature interactions.

	Precision medicine is not only interested in correctly diagnosing a patient but also in estimating its prognosis.
	The prognosis is estimated with survival analysis.
	Survival analysis focuses on analyzing and modeling the time it takes for an event of interest to occur.
	The event of interest can be death, relapse, or recovery, among others.
	A standard survival model is the Cox proportional hazards model.
	This standard model has been extended to consider non-linear survival data with deep learning models~\cite{katzmanDeepSurvPersonalizedTreatment2018,Ching2018}.
	Cox-nnet~\cite{Ching2018} uses a log-likelihood loss (\cref{eq:cox_loss}) to predict prognosis from gene expression data.
	Deep-surv~\cite{katzmanDeepSurvPersonalizedTreatment2018} uses the same principles to recommend the best treatment option for a patient by computing the risk ratio.
	\citeauthor{Lee2022} model the survival task as a regression on the survival days; this approach does not consider the particular structure of survival data and completely ignores censoring data.

\section{Prediction from multi-omics data}
	In the previous section, we saw that cancer types can be predicted from single omics; going to multi-omics data allows for a holistic view, which is necessary for a better understanding of the roles played by the various omics layers.
	In both cases, a sphere or a vertical cylinder viewed from the top looks like a circle.
	However, there is insufficient information to infer the 3D shape correctly; more information is needed to select from the possible hypothesis.
	Each view of a 3D object carries different information; they can be complementary or redundant and help improve predictions regarding performance and robustness.
	This is precisely what multimodal machine learning is: combining multiple views or modalities to make better predictions.
	In precision medicine and the realm of omics, without prior knowledge, a mutation's impact cannot be known without \textit{looking} at the protein.
	Furthermore, gene expression is a highly regulated phenomenon involving various molecular actors~(\cref{subsec:gene_regulation}).

	The following subsection will present multimodal machine learning, its properties, challenges, and methods to fuse multiple modalities.
	This description will be based on the work of \citeauthor{MML_morency}: \citetitle{MML_morency}~\cite{MML_morency}.
	Then, describe some machine learning techniques used to integrate multi-omics data.
	Finally, it presents the deep learning architectures used to integrate multi-omics data.

	\subsection{Multimodal machine learning}
		Multimodal machine learning corresponds to machine learning methods developed to learn from multiple modalities.
		Theoretical justification established that multimodal deep learning is better than unimodal~\cite{NEURIPS2021_5aa3405a}.
		A multimodal model can correspond to a model where the input is from one modality and the output from another modality (translation) or models that jointly learn from multiple modalities, the joint representation can later be used to predict an entire modality.
		A modality represents the expression of a phenomenon~\cite{MML_morency}.
		If an instance in one modality is explicitly linked to a corresponding instance in another modality, the modalities are paired.
		On the contrary, unpaired modalities correspond to situations without one-to-one correspondence between instances across modalities.
		Multi-omics data obtained by analyzing a sample from one patient are paired, but methods jointly measuring multiple modalities from the same cell are rare.
		Multi-omics single-cell data are often unpaired; this data type is obtained by measuring different cells from the same condition.
		In the context of this manuscript, a modality corresponds to an omics data type.
		Each modality has its own specific characteristics and distributions.
		Indeed, each omics is obtained using different techniques and processed with different bioinformatics tools, leading to data with different dimensionalities and distributions.
		The modality heterogeneity results from biological, as they measure different molecular components and technical reasons.
		Making the integration of these diverse data types a significant challenge.
		While heterogeneous, modalities are connected as they share complementary information.
		For instance, it should be noted that not all mRNAs are translated to proteins.
		Both mRNAs and proteins provide a complementary view of the underlying biological process.
		They also provide the same information due to their redundancy, as protein levels confirm the translation of mRNAs.
		Integrating multiple modalities allows interactions between modalities to happen and exploit redundant and complementary information.
		Multimodal models aim to learn a representation that reflects those cross-modal interactions.
		To construct a multimodal representation different fusion strategies are employed:
		\begin{description}[%
				style=multiline,
				leftmargin=!,
				labelwidth=2.5cm,
			]
			\item[Early fusion]
				The early fusion method combines the different omics in the data space; the resulting vector is analyzed like an unimodal input.
				This approach is well-studied due to its simplicity.
				However, it does not fully exploit the complementarity between omics and is known to be sensitive to the differences in distributions across omics.
			\item[Late fusion]
				For the late fusion approach, the fusion occurs in the prediction space.
				Each omics is processed separately with modality specific models, and the predictions are combined, similar to ensemble methods.
				Errors between modalities should not be correlated to ensure they have complementary effects.
				This approach does not necessarily capture the complex interactions between modalities.
			\item[Intermediate fusion]
				With intermediate fusion, a modality-specific encoder first processes each omics separately.
				Then, the features learned from each modality are combined in the latent space with specific fusion blocks to learn a joint representation before being fed to a multimodal classifier.
				The unimodal encoders can be jointly learned or pretrained.
		\end{description}
		Instead of fusing the latent representations of each modality, it is also possible to coordinate them.
		Coordination aims to ensure that the representations learned from each modality are compatible by constraining latent representations with representations of other modalities.
		The constraints are obtained by imposing similarity constraints, finding linear combinations of variables in each set that are maximally correlated with each other, or through contrastive learning.
		Contrastive learning forces the model to encode similar items closer together in the representation space while pushing dissimilar items further apart.

		The constructed multimodal representation must account for the interactions between modalities, \ie{}how to correctly align the different modalities.
		The alignment of modalities corresponds to knowing what features from one modality have an impact or are connected to another feature from a different modality.
		There are multiple ways of modeling this with multimodal machine learning models.
		Contrastive learning is a good way to find an alignment between multiple modalities by forcing similar concepts expressed in different modalities to match.
		Optimal transport is another way to align multiple modalities in a shared latent space by solving a divergence minimization problem, \ie{}minimizing the cost of transporting one distribution to another.
		Learning better multimodal representation can be achieved by contextualizing a modality's representation with the representation of another.
		The contextualization aims to capture the context in which the data appears by modeling all interactions between different modalities.
		For instance, a multimodal model dealing with gene expression data and \gls{dnam} data would capture how features relate to each other, \ie{}how a methylation site affect the expression of a gene.
		Such approach allows to capture more nuanced and complex interactions.
		Aligment layers based on the scaled-dot product attention are used to capture the alignment.
		The attention can be used in various ways to align a modality pair:
		\begin{description}[%
				style=multiline,
				leftmargin=!,
				labelwidth=3cm,
				%format=\labelsinglespace
			]
			\item[Early concatenation]
				The representation of two modalities \(X_A \in \symbb{R}^{L_A \times d}\) and \(X_B \in \symbb{R}^{L_B \times d}\) are concatenated and passed to a self-attention layer: \(\operatorname{Attention}\left(\symbfup{C}\left(X_A,X_B\right)\right)\), where \(\symbfup{C}\) is the concatenation operation.
				This early fusion technique learns undirected connections between modalities: all modality elements are connected to all other modality elements.
				This fusion approach considers both intra-modality and inter-modality interactions.
				While this fusion method can be adapted to any number of modalities, increasing the number of modalities increases the input dimension \(L'=\sum_i L_i\) of the attention layer, therefore increasing the computational complexity.
			\item[Hierarchical attention]
				Each modality is individually encoded with an attention based encoder, the resulting representation are concatenated and passed to another attention layer:
				\begingroup
				\setlength{\abovedisplayskip}{2pt}
				\setlength{\belowdisplayskip}{2pt}
				\setlength{\abovedisplayshortskip}{0pt}
				\setlength{\belowdisplayshortskip}{0pt}
				\begin{align*}
					Z_A & = \operatorname{Attention}\left(X_A\right)                             \\
					Z_B & = \operatorname{Attention}\left(X_B\right)                             \\
					Z   & = \operatorname{Attention}\left(\symbfup{C}\left(Z_A,Z_B\right)\right)
				\end{align*}
				Decoupling the learning of the intra-modality interactions from the inter-modality ones can help construct better individual representation, and modality encoders could be pre-trained.
				This approach can also be adapted to any number of modalities but faces the same limitations as early concatenation.
				\endgroup
			\item[Cross attention]
				This fusion approach is used to compute asymmetric interactions, a target modality is reinforced in a directed manner by a source modality.
				Usually, this methods is done in a bidirectional manner resulting in two contextualized representation accounting for the assymetric interactions:
				\begingroup
				\setlength{\abovedisplayskip}{2pt}
				\setlength{\belowdisplayskip}{2pt}
				\setlength{\abovedisplayshortskip}{0pt}
				\setlength{\belowdisplayshortskip}{0pt}
				\begin{align*}
					Z_{A \rightarrow B} & = \operatorname{CA}\left(X_A, X_B \right) \\
					Z_{B \rightarrow A} & = \operatorname{CA}\left(X_B, X_A \right)
				\end{align*} % add a ref to equation
				\endgroup
				This fusion techniques can be adapted to any number \(m\) of modalities, but it requires to compute \(m\left(m-1\right)\) different cross-attentions.
				While this approach can capture cross-modal interactions between pairs of modality, it is not able to capture the whole multimodal context.
				It has been proposed to concatenate the outputs of the different cross-attention and process the resulting embedding by another attention layer to capture the global context. % vilbert
		\end{description}
		The different approaches can be combined in a single architecture.
		For instance, in the TriBERT architecture, three cross-attention are used to integrate three modalities.
		For each cross-attention the query is obtained from one modality, and the key and value results from the concatenation of the two others modalities.
		% https://www.biorxiv.org/content/10.1101/2024.02.26.582051v1.full.pdf
		% https://openaccess.thecvf.com/content/CVPR2023/papers/Wang_Multi-Modal_Learning_With_Missing_Modality_via_Shared-Specific_Feature_Modelling_CVPR_2023_paper.pdf
		% https://people.csail.mit.edu/khosla/papers/icml2011_ngiam.pdf

		In biology, connections between modalities are already known: an mRNA transcribed from a coding gene will likely be translated into a protein, and the impact of promoter methylation on the mRNA expression level is known.
		Moreover, the connections between individual features from different modalities are also known.
		For instance, connections from a coding gene and the corresponding protein are known, or CpGs methylation can be mapped to a gene based on their position in the genome.
		Adding this knowledge into the integration process could help construct better multimodal representation.
		Restricting the integration to features with known regulatory links would leave out many features, as a large part of \gls{ncrna} roles still need to be discovered.
		Even though their role is unknown, they can still play an essential role in gene expression regulation, particularly in cancer development.
		Using data-driven approaches, such as contextualized representations based on the attention mechanism, can help identify new connections and generate new knowledge.

		Access to paired multimodal data can be difficult during inference.
		Only some modalities will be generated to reduce costs, or a patient may refuse some analysis.
		A solution would be to use methods that exploit paired multi-omics datasets during training, and inference is done using only one omics.
		With co-learning, all available modalities are used to construct a joint or coordinated multimodal representation.
		Adding more modalities during the training allows for the transfer of information and enriches the multimodal representation.
		While paired multimodal datasets are the ideal scenario, the availability of such datasets remains limited.
		Many multi-omics datasets contain samples where one or more modalities are missing, and with single-cell, many of the available multi-omics datasets are unpaired.
		New methods that can adapt to missing modalities, or perform diagonal integration of unpaired datasets are required to exploit the currently available datasets.

		An important challenge not specific to multimodal machine learning is the interpretability aspects of the model predictions.
		A better understanding of multimodal models is critical to gain insights into multimodal learning and help improve the model design.
		In a multimodal setting, modality and multimodal-level interpretability are essential, in addition to feature-level interpretability.
		At the modality level, it is interesting to know how modalities are used to make a prediction and their individual contribution to the prediction.
		Indeed, in multimodal training, modalities compete with each other as they learn at different rates, only a subset of modalities is explored by the network~\cite{pmlr-v162-huang22e}.
		Detecting modalities that hinder model performances is essential as adding more modalities does not necessarily mean adding more information; sometimes, it adds noise and reduces performance.
		With multiple modalities, one important interest consists in knowing the type of interaction between modalities: additive, \( w_A\cdot g_A\left(X_A\right) + w_B\cdot g_B\left(X_B\right)\),  or multiplicative, \(w\cdot  g\left( X_A \times X_B\right)\), where \(g\) is a model function or the identity function.
		Models using additive interactions do not use cross-modal information but only the unimodal information.
		While complex models are able to learn complex multiplicative interactions, it is essential to ensure that a complex models is not learning simpler additive interactions.
		To detect models not using multiplicative fusion, \citeauthor{EMAP} proposed to approximate a multimodal model \(g\left(X_A,X_B\right)\) by a multimodal additive model: \(\hat{g}\left(X_A,X_B\right) = \hat{g}_A\left(X_A\right) + \hat{g}_B\left(X_B\right) - \hat{\mu}\).
		Perfomance degradation from the additive model \(\hat{g}\) indicates that the model \(g\) did learn cross-modal interactions.
		Approaches to understanding multimodal deep learning models can be categorized into approaches producing post-hoc explanations or interpretable models by design.
		Post-hoc explainability of multimodal models uses attribution methods, such as Grad-CAM~\cite{Chandrasekaran2018DoEM} or LRP~\cite{Ellis2021}, to understand modality importance. % todo add examples of usage
		Model weights can be directly used to detect feature interactions~\cite{tsang2017detecting}.
		In DIME~\cite{DIME}, authors propose to estimate the different contributions with LIME, a widely used interpretation approach~\cite{lime}.
		They decompose a model \(g\) into a sum of two terms: unimodal contributions \(\operatorname{UC}\) and multimodal interactions \(\operatorname{MI}\).
		Then, LIME is applied on this two newly defined models perturbing only one modality at a time to obtain the different explanation: the unimodal contribution, \(\operatorname{UC}_A\) and \(\operatorname{UC}_B\), and the contribution of modality \(A\) to the multimodal interactions, \(\operatorname{MI}_A\) and reciprocally the contribution of modality \(B\) to the multimodal interactions \(\operatorname{MI}_A\).
		Perceptual scores~\cite{PerceptualScores} have been proposed as a way to detect on which modality a model relies on for the predicitons.
		The perceptual score towards a modality \(A\), \(\symcal{P}_A\), is the normalized expectation of sample perceptual scores: \(\symcal{P}_A = \frac{1}{nz}\sum_{i=1}^{N}\symcal{P}_{x_i}\), where the sample perceptual score \(\symcal{P}_{x_i}\) correspond to the difference of accuracy between the model using all modalities and the model the model that do not use the modality \(A\) and \(z\) is the normalization factor.
		In~\cite{SHAPE}, authors propose to use game theory and Shapley values, \(\phi_i\), to estimate the marginal contribution and the cooperation of individual modalities.
		In the Shapley framework, individuals modalities are considered as players.
		The marignal contirbution of a modality \(m\) corresponds to its scaled Shapley value, \(\symbfcal{S}_m = \frac{1}{z}\phi_m\) and the modality cooperation of a set of modalities \(\symbfcal{A}\),  corresponds to the Shapley value of the set of modalities where the individual modality contributions are substracted, \(\symbfcal{C}_{\symbfcal{A}} = \phi_{\symbfcal{A}} - \sum_{m \in \symbfcal{A}}\phi_m\).
		A new approach based on information decomposition quantifies the degree of redundancy, uniqueness, and synergy between input modalities and a specific task~\cite{liang2023multimodal,liang2023quantifying}.
		Post-hoc approaches, while model-agnostic, are computationally expensive as they estimate the different contributions by substituting part of the input.
		The cost to run those methods increases with the number of modalities.
		Models interpretable by design exploit new blocks, such as attention, to quantify modality importance or cross-modal interactions.
		The PJ-X architecture~\cite{8579013} uses a block to produce an attention map that highlights the important parts of an image in a visual-question-answering model.
		Graph-based fusion explicitly models modality fusion as a directed graph where each node represents a modality or a fusion of modalities.
		A learnable weight is associated with each edge representing the modality importance in the fused representation~\cite{bagher-zadeh-etal-2018-multimodal}.

		%file:///home/aurel/T%C3%A9l%C3%A9chargements/s41467-022-31104-x.pdf
		%file:///home/aurel/T%C3%A9l%C3%A9chargements/s41592-024-02391-7.pdf
		%https://www.nature.com/articles/s41467-023-43019-2
		%https://cdn.aaai.org/ojs/16330/16330-13-19824-1-2-20210518.pdf


	\subsection{Multi-omics data integration}
		In the field of multi-omics there are two main integration types: N or horizontal integration and P or vertical integration.
		Horizontal integration corresponds to the integration of omics measurment for the same samples, whereas vertical integration integrate multiple studies of the same omic type.
		This manuscript focuses on horizontal integration methods.
		Various machine learning methods have been proposed to horizontally integrate multi-omics data.
		Methods can be supervised, where label information is incorporated in the learning of the multimodal representation, or unsupervised.
		Many unsupervised integration methods are based on matrix factorization, also known as matrix decomposition methods, the factorization of a matrix into the product of two matrices:
		\begin{equation}
			X \approx WH
		\end{equation}
		where \(X \in \symbb{R}^{N\times d}\) is a data matrix,\(W \in \symbb{R}^{N\times k}\) a factor matrix and \(H \symbb{R}^{k\times d}\) a loading matrix, \(k < d\) reprsents the number of dimensions used to construct the low dimensional space.
		As there are more than one data matrix in multi-omics studies, \(X_i\quad i \in \left\{1, \cdots, m\right\}\), different strategies have been depicted to perform the integration:
		\begin{itemize}[nosep]
			\item Each data matrix is decomposed individually, \(X_i \approx W_iH_i\) and one of the factor matrix \(W_i\) is used for downstream analyses. The cost function used to find the optimal factor and loading matrix should be designed to allow multi-omics information to be shared across factor matrices.
			\item A common factor matrix is used across the different modalities, \(X_i \approx WH_i\) and the shared factor matrix is used for downstream analyses.
		\end{itemize}
		Finding the optimal factor and loadings matrix is achieved by solving the following optimization problem:
		\begin{equation}
			\min_{W,H} {\left\| X - WH \right\|}_{F} \label{eq:matrix_fact_opt}
		\end{equation}
		To find the matrices \(W\) and \(H\), \gls{nmf} is a common approach to solve the \cref{eq:matrix_fact_opt} by adding a non negativity constraint: \(W \geq 0\) and \(H \geq 0\).
		An extension have been proposed to handle mulitple omics, \gls{intnmf}, that adds a parameter \(\phi_{i}\) controlling the weight of each modality:
		\begin{equation}
			\min_{W,H_i} \sum_{i=1}^{m} \phi_{i}{\left\| X_i - WH_i \right\|}_{F} \quad \text{s.t} \quad W \geq 0 \;\text{and}\; H_i \geq 0
		\end{equation}
		\Gls{jive} adds an omics specific term to the decomposition:
		\begin{equation}
			X_i \approx W_iZ + W_i^sZ_i^s
		\end{equation}
		Parameters are estimated by minimimizing \({\left\|E\right\|}^2 = {\left\|\left[E_1, \cdots, E_m \right]^T\right\|}^2\), where \(E_i = X_i - W_iZ + W_i^sZ_i^s\) and adding an orthogonality constraints between the joint, \(W_iZ\), and individual terms, \( W_i^sZ_i^s\).
		iCluster is another approach to integrate multi-omics data where the factor matrix, \(H\) is shared across all modalities.
		By assuming that the factor matrix and the residuals, \(E_i = X_i - W_iH\) are norammly distributed, a multivariate normal log-likelihood can be derived before applying the expectation-maximization algorithm to solve this problem.
		The variational Bayesian framework can also be used for the estimation of the matrix \(W\) and \(H\) by adding prior assumptions on their distributions~\cite{MOFA}.
		Bayesian consensus clustering is another bayesian approach used to perform multi-omics data integration.
		Each omics has its own clustering loosely adhering to a consensus clustering, making the individual clustering connected and sharing information.

		Previsouly, we showed that the coordination of multiple representation is a way to perform data integration.
		\Gls{cca} is a method whose objective is to finding linear combinations of variables in each set that are maximally correlated with each other.
		The idea is to construct \(W=X_1u_1\) and \(Z=X_2u_2\) such that the correlation between \(W\) and \(Z\) is maximal.
		This methods only works for the integration of two modalities but has been generalized to work with any number of modalities:
		\begin{equation}
			\max_{u_k} \sum_{\substack{i=1 \\ j=1 \\ i\neq k}}^{m} c_{ij}\operatorname{Cov}\left(X_iu_i,X_ju_j\right)
		\end{equation}
		the term \(c_{ij}\) indicates if the modality \(i\) and \(j\) are connected.
		Solving this minimization problem implies the inversion of the covariance matrix, which in the case of omics data is often not invertible therefore regularization terms are added.
		This methods has been extended to adress variable selection by adding a sparsity constraints, known as \gls{sgcca}: \({\left\|u_k \right\|}_2 = 1\) and \({\left\|u_k \right\|}_1 \leq s_k\), where \(s_k\) is an hyperparmeters controlling the level of sparsity.
		The DIABLO methods~\cite{DIABLO} is an extension of \gls{sgcca} to supervised tasks.
		It replaces one the data matrix \(X_k\) by an indicator matrix \(Y\)which indicates the class of each samples.
		Network-based methods are another way of integrating multi-omics data.
		For instance, \gls{snf} infer relationships based on similarity between patients or features in the different omics, and the combine the various similarity matrices into one.

		\begin{mybox}[label=box:snf]{\Glsfmtfull{snf}}
			This is a numbered box.
		\end{mybox}

		Classical machine learning algorithm like \gls{svm} can be used to integrate multi-omics data with a late fusion approach.
		An \gls{svm} classifier is trained for each omics and preditions are combined with learnable weights~\cite{CarrilloPerez2022}.

		rise of deep learning approaches
		motivate use of deep learning methods
		multimodal DL is able to model nonlinear within and cross-modality relationships
		%https://www.ncbi.nlm.nih.gov/pmc/articles/PMC8829812/pdf/main.pdf
		%http://eprints-phd.biblio.unitn.it/2981/1/thesis.pdf
		%https://compgenomr.github.io/book/matrix-factorization-methods-for-unsupervised-multi-omics-data-integration.html
		%https://academic.oup.com/bioinformatics/article/35/17/3055/5292387#409338217
		%https://www.embopress.org/doi/full/10.15252/msb.20178124
		%https://academic.oup.com/bib/article/21/6/2011/5645549
		%https://academic.oup.com/bib/article/21/2/541/5316049?login=false (bayesian methods)
		% https://link.springer.com/article/10.1186/s12859-015-0857-9
		%https://www.frontiersin.org/journals/genetics/articles/10.3389/fgene.2017.00084/full
		% https://www.frontiersin.org/journals/genetics/articles/10.3389/fgene.2017.00084/full
		% more unsupervised methods: https://academic.oup.com/nar/article/46/20/10546/5123392
		% https://www.ncbi.nlm.nih.gov/pmc/articles/PMC8981526/
		% https://www.sciencedirect.com/science/article/pii/S200103702200544X
		% https://www.ncbi.nlm.nih.gov/pmc/articles/PMC8829812/
		% https://www.frontiersin.org/journals/genetics/articles/10.3389/fgene.2022.854752/full
		% https://www.frontiersin.org/journals/artificial-intelligence/articles/10.3389/frai.2023.1098308/full
		% https://www.nature.com/articles/s41467-020-20430-7

	\subsection{Deep learning models for multi-omics data}
		%TODO
		%\cite{Ma2019} AEs one per omics intermediate fusion
		%Self-supervised learning of multi-omics embeddings in the low-label, high-data regime

		Different deep-learning architectures like \gls{mlp}, \gls{ae}, \gls{vae}, \gls{cnn}, \gls{gnn}, or transformers can be combined with the various fusion methods: early fusion, intermediate fusion, and late fusion.
		Early fusion combines raw or abstracted features before feeding them to a model.
		Intermediate fusion combines latent representation obtained with modality-specific encoders.
		With this approach, it is not necessary to fuse all modalities at once; the modalities' fusion can happen gradually.
		Late fusion combines the predictions obtained from each omics.
		The different fusion strategies can also be combined to have more complex integration schemes.

		The different approaches previously developed for omics data integration will be presented according to their fusion strategy.

		\subsubsection{Early fusion}
			Early fusion was primarily used with \glspl{ae} as a dimension reduction method.
			Features of the different modalities are concatenated before being encoded into a multi-omics latent representation.
			The latent representation is then used for various tasks.
			In~\cite{Chaudhary2018}, survival-associated features of the latent representation are selected with a Cox model, and a K-means clustering is applied to the selected features to infer survival groups.
			The resulting labels are used to construct a \gls{svm} based classifier from a subset of the multi-omics features.
			To prevent overfitting, an elastic net regularization\footnote{\(\lambda_{1} \left\|\theta\right\|_{1}^{2} + \lambda_{2} \left\|\theta\right\|_{2}^{2}\), where \(\theta\) are model parameters} is added to the reconstruction loss, \(\symcal{L}_{AE}\)~\cref{eq:loss_ae}.
			\citeauthor{Lee2020} applied a similar strategy to identify survival groups in \gls{luad} patient but used a classifier based on \glspl{rf}~\cite{Lee2020}.
			Similarly, \citeauthor{Guo2020} extracted a multi-omics latent representation by applying an \gls{ae} on concatenated multi-omics data and constructed a logistic classifier to predict ovarian subtypes from mRNA data.
			The labels used to train the classifier were obtained by applying a K-means clustering on the multi-omics latent representation~\cite{Guo2020}.
			A recent study~\cite{Yu2022} showed that \glspl{ae} was an excellent method to extract nonlinear relationships in multi-omics data when the sample size is large enough.
			The \gls{ae} architecture also excluded the influence of confounding factors by adding them to the latent representation before passing it to the decoder during training.

			Those reconstruction methods are suitable for unsupervised training when no labels are available, but reconstruction of large dimensional vectors such as omics profiles is challenging.
			Supervised training of \gls{mlp} architectures is a better approach in the presence of known labels.
			However, with an early fusion strategy, the \gls{mlp} input is very high-dimensional, leading to a high number of parameters.
			The introduction of biological knowledge in the \gls{mlp} architecture is a common way to reduce the number of parameters to estimate.
			Similarly to PasNet~\cite{haoPASNetPathwayassociatedSparse2018} or DeepGONet~\cite{bourgeaisDeepGONetSelfexplainable2021}, connections between neurons are restricted to known biological relationships: Reactome~\cite{PNet} or pathway gene set~\cite{DeepOmix}.
			A neuron \(h^{0}_{i}\) from the first layer combines features \(x_{i}^{m}\) from the \(M\) different omics correponding to the gene \(i\): \(h^{0}_{i} = f\left( \sum_{m=1}^{M} w_{m}x_{i}^{m} \right) \).
			Neurons in the following layer combine the information from these functional genes.
			MiNet~\cite{Hao2019} explicitly models feature interactions by creating a multi-omics feature vector accounting for the regulatory effects of CNV and DNAm on gene expression.
			The multi-omics feature vector is constructed as follows:
			\begin{equation}
				\symbf{x} = {
				\begin{bmatrix}
					\symbf{x}_{mRNA}                      \\
					\symbf{x}_{CNV}                       \\
					\symbf{x}_{DNAm}                      \\
					\symbf{x}_{mRNA}\odot \symbf{x}_{CNV} \\
					\symbf{x}_{mRNA}\odot \symbf{x}_{DNAm}
				\end{bmatrix}
				}^T
			\end{equation}
			where \(\symbf{x}_{mRNA}\odot \symbf{x}_{CNV}\) models the regulatory effect of CNV on mRNA.
			The multi-omics feature vector is then transformed to a functional representation by projection: \(\tilde{\symbf{x}} = f\left(\symbf{x}W\right)\).
			The resulting functional gene representation is fed to a network based on the PasNet architecture.

			Some architectures drew inspiration from the promising results of \gls{cnn} in computer vision.
			The application of 1D-\gls{cnn} on concatenated multi-omics features was explored~\cite{CNN1D} or the development of complex transformation, \(\symcal{T}\), to obtain a 2D image~\cite{PathCNN}.
			PathCNN's transformation is based on biological knowledge: 146 KEGG pathways.
			For each pathway \(p\) and each omics \(m\), the input matrix \(X^{m} \in \symbb{R}^{n \times d}\) is transformed into a pathway matrix, \(X_{p}^{m} \in \symbb{R}^{n \times d_{p}}\), by selecting only the \(d_{p}\) genes related to this pathway.
			The pathway matrix is decomposed into its PCA representation: \(Z_{p}^{m} \in \symbb{R}^{n \times k}\), where \(k\) is the number of selected principal components.
			All the PCAs pathway matrices are concatenated to form a single matrix: \(Z^{m} \in \symbb{R}^{n \times 146k}\).
			The matrix is then rearranged to obtain a matrix representation for each sample: \(Z_{i}^{m} \in \symbb{R}^{146 \times k}\), then the representation of the different omics are combined to form the pathway image \(Z_{i} \in \symbb{R}^{146 \times mk}\).
			Finally, pathways (\ie{}rows) are reordered to have correlated pathways close to each other.

			\Glspl{gnn} with an early fusion strategy have also been applied to integrate multi-omics data with different types of graphs: \gls{ppi}~\cite{Guo2023} or patient similarity graphs~\cite{MoGCN}.
			In MoGCN~\cite{MoGCN}, the graph is obtained by fusing the patient similarity graphs obtained on each omics with \gls{snf}.
			In an early fusion context, the features from the different omics are concatenated before being mapped to the graph nodes.
			Each node represents a gene for the \gls{ppi} graph; therefore, features corresponding to this gene from the different omics are concatenated.
			The node embedding is a vector \(h_{v} \in \symbb{R}^{M}\), where \(M\) is the number of omics considered.
			For patient graphs, each node represents a patient, and the node embedding is the concatenation of the \(M\) expression profile of the corresponding patient: \(h_{v} \in \symbb{R}^{\sum_{i=1}^{M}d_i}\).
			Instead, the latent representation obtained with an \gls{ae} on the concatenated expression profile can also be used~\cite{MoGCN}.
			This strategy reduces the node embedding dimension.
			Then, \gls{gcn} are applied on the graph to obtain a new graph representation for \glsxtrshort{ppi}-based approach and a new node representation for the patient-graph approach.
			\citeauthor{Guo2023}~\cite{Guo2023} applied an attention mechanism to the initial node embedding \(X\).
			This attention mechanism is used to capture the association between the different omics.

			Architectures based on the Transformer architecture have been proposed.
			\citeauthor{SubtypeFormer} proposed applying self-attention on the concatenation of the multi-omics features~\cite{SubtypeFormer}.
			However, the method is only succinctly described, and they do not specify how the self-attention mechanism is applied to one-dimensional inputs.
			Based on the code, they reshape the input vector to a three-dimensional vector to consider multi-heads.
			In practice, this reshaping corresponds to creating multiple input elements or tokens and considering a single head in the self-attention mechanism.
			DeepPathNet is another architecture combining the self-attention mechanism with an early fusion strategy~\cite{DeepPathNet}.
			Multi-omics features are transformed into a pathway representation, and they used the 241 curated pathways from LCpathways~\cite{LCpathways}.
			Multi-omics features belonging to pathways are concatenated and projected to a 512-dimensional vector with pathway-specific weights.
			Interactions between the 241 pathways are then computed with a transformer block~\cref{eq:transformer_block}.

		\subsubsection{Late fusion}
			Unlike early fusion, late fusion allows for the independent processing of each omics before merging their predictions.
			The independent processing allows to better exploit the complementarity between omics but fails to capture complex dependencies between omics.

			\citeauthor{Sun2019} proposed a late fusion of mRNA, CNV, and clinical data defined as a weighted combination of the predictions obtained from an \gls{mlp} trained on each modality~\cite{Sun2019}.
			The optimal weights are obtained after a grid search.
			Instead of \textit{manually} searching for the set of optimal weights, they can be learned with gradient descent~\cite{CarrilloPerez2022}.

			MOGONET~\cite{MOGONET} developed a late fusion strategy combining \glspl{gnn} with \gls{vcdn} module~\cite{VCDN}.
			In the first phase, for each omics, a patient similarity graph is constructed and used to train a \gls{gcn} predicting cancer subtypes.
			In the second phase, the \gls{vcdn} module is trained, and the \gls{gnn} are fine-tuned.
			The \gls{vcdn} module takes as input the \textit{cross-omics discovery tensor} \(C^{o}\), an \(M\)-dimensional tensor, obtained by applying a generalized outer-product of predictions \(\hat{y}^{m} \in \symbb{R}^{C}\) obtained from each omics:
			\begin{align}
				C^{o}                     & = \left(C^{o}_{a_{1}\cdots a_{M}}\right)_{\substack{1 \leq a_{i} \leq C \\ 1 \leq i \leq M}} \\
				C^{o}_{a_{1}\cdots a_{M}} & = \prod_{m=1}^{M}\hat{y}_{a_{m}}^{m}
			\end{align}
			The \textit{cross-omics discovery tensor} is reshaped into a \(C^{M}\)-dimensional vector.
			The resulting vector is passed to the \gls{vcdn} module, a \gls{fcn} that maps the tensor to a space of the same dimension before projecting to the label space.
			The first projection requires \(C^{2M}\) parameters; this can easily represent billions of parameters for multiclass problems with multiple modalities.
			This integration strategy cannot be considered in some scenarios as the memory requirements are too high.
			Subsequent works replaced the individual omics predictors based on \gls{gcn} with one based on \gls{gat}~\cite{MODILM,Leng2022}.
			HyperTMO~\cite{Wang2024} is another late fusion strategy used to integrate mRNA, miRNA and DNAm. 
			Each omics is processed independently using \gls{hgcn}. 
			\Glspl{hgcn} are similar to \glspl{gcn} but instead of using a graph they use an hypergraph. 
			An hypergraph is a generalization of a graph where edges can join any number of nodes. 
			In their study, each nodes represents a patient and the edges were determined by applying the \gls{knn} algorithm on the patient similairity matrix. 
			\Glspl{hgcn} are used to extract evidences from each omics. 
			The evidences are then parameterized with a 
			Each network provide evidences Dirichlet distribution for uncertainty estimation. 
			This uncertainty estimation us used to weight the reliability of each omics. 
			The different evidences are combined using an evidential reasoning approach, the fusion is done considering the uncertainty of each evidence. 
			The combined evidence is used to make the final classification. 
			
			Late fusion approaches cannot capture inter-omics interactions and use the complementarity information between omics~\cite{Picard2021}. 
			A simple combination of the different predicitons might not be sufficient to consider the complex omics interactions.

		\subsubsection{Intermediate fusion}
			Intermediate fusion is a flexible approach that can adapt complex integration scheme.
			This flexibility might explain why this approach has been widely studied. 

			\Glspl{ae} have been widely used for intermediate fusion of multi-omics data. 
			One \gls{ae} is trained per omics and the latent representation are combined into a multimodal representation. 
			The newly constructed representation is then used for downstream tasks such as cancer classification or survival predictions.
			\citeauthor{DeepProg} proposed DeepProg~\cite{DeepProg} an architecture based on \glspl{ae}.
			Survival associated features are selected in each omics latent representation. 
			The resulting features are concatenated and used to construct an \gls{svm} based classifier to predict the survival risk groups. 
			Other ways of combining the latent representation have been explored. 
			In~\cite{Wissel2021}, \citeauthor{Wissel2021} explored various ways of combining the latent representation: 
			\begin{itemize}[nosep]
				\item concatenation of the different latent spaces,
				\item mean or max pooling of the different latent spaces,
				\item after concatenating the different latent spaces a second level \gls{ae} is added (hierarchical).
			\end{itemize} 
			To ensure a relevant latent space for downstream tasks, a supervision of the individual latent spaces was added and all \glspl{ae} are jointly trained. 
			They showed that the best strategy was the hierarchical \gls{ae}. 
			\citeauthor{Tong2021}~\cite{Tong2021} proposed to combine the latent spaces by taking the average of the latent spaces learned from each omics and added consensus constraints between latent representation.
			The consensus constraints is achieved by adding a regularization maximizing the cosine similarity between latent spaces. 
			Hierarchical \glspl{ae} concatenate all the latent representation available before passing the resulting representation to the secondary \gls{ae}. 
			\citeauthor{Wu2022StackedAB} proposed a variation of this scheme where only two modalities are initially combined and passed to secondary \gls{ae}. 
			The latent spaces obtained with the secondary \gls{ae} is combined with a third modality and passed to a third \gls{ae}. 
			If they are more modalities the process is repeated. 
			They called this approach stacked-\gls{ae}~\cite{Wu2022StackedAB}. 
			The encoder-decoder architecture allows to create new training strategies. 
			The classical way to train an \gls{ae} is to use the decoder to reconstruct the input, in a multi-omics framework the decoder can also be used to reconstruct the other modalities. 
			DeePathlogy\footnote{Suprisingly they used a cosine-similarity loss for their classification tasks.}~\cite{Azarkhalili2019} uses an encoder to encode mRNA expression profile and the a decoder that reconstruct the mRNA profile but also predict the level of miRNA expression. 
			They did ensure that no miRNA genes were present in the mRNA expression profile. 
			This translation tasks forces the network to learn the relationships between the two omics and thus should improve the latent representation.
			Instead of having a single \gls{ae}, \citeauthor{CrossAE} proposed to have one \gls{ae} per omics and each \gls{ae} reconstruct its input and the other omics. 
			In a first step, each \gls{ae} is trained to reconstruct its input, in a second step the latent representation of each omics is passed to the decoder of the other omics to reconstruct the second omics~\cite{CrossAE}. 
			In a final step the latent spaces are combined with an element-wise average and passsed to task-specific network. 
			With this strategy, the authors aims to extract the consensus information across the different omics. 
			As their strategy of combining encoders and decoders from different omics significantly increases the number of calculation, for \(M\) modalities there are \(M(M-1)\) combinations to consider, they restricted their analysis to pair of modalities. 
			\Glspl{vae} have also been explored with intermediate fusion strategies. 
			Each omics is individually processed with \glspl{fcn}, the latent representation are combined and the variational framework is applied on this combined representation. 
			In~\cite{Zhang2019}, authors used this strategy to integrate mRNA and DNAm expression profiles. 
			However as they used raw CpGs features, the first \gls{fcl} were restricted to CpGs features belonging to the same chromosome in order to reduce the number of parameters in the network encoding DNAm.
			The mean of the latent state distributions was used in a task-specific network. 
			The task-specific network and the \gls{vae} were jointly trained. 
			The authors proposed an extension of their approach were they added multiple supervision of the latent space: cancer classification, survival prediction, demographic (age, gender) prediction~\cite{Zhang2021}. 
			Instead of jointly training the networks \citeauthor{Hira2021} used the mean of the latent state distributions to construct an \gls{svm} classifier and a Cox model~\cite{Hira2021}.
			\citeauthor{customics} proposed a method combining \glspl{ae} and \glspl{vae}~\cite{customics}.
			Each modality is encoded with its own \gls{ae}; the different \glspl{ae} are trained individually. 
			Afterwards, the different latent spaces are passed to a central \gls{vae}. 
			During the training of the central \gls{vae} omics \glspl{ae} are finetuned.  
			
			In custOmics, \citeauthor{customics} combine \gls{ae} and \gls{vae}; each modality is encoded with an \gls{ae}, and the multimodal representation is obtained by passing all the latent spaces in a \gls{vae}~\cite{customics}.
			hierarchical mixed integration, 2 training phases: 1 each ae individually then a joint integration and fine tuning the AEs
			task: survival or classification (cancer or subtype)
			CNV, DNAm, mRNA, IF

			omiVAE~\cite{Zhang2019}: VAE intermediate fusion
			use of CpG features, high numbers,
			These methylation
			features are separated into different fully connected blocks
			according to their targeting chromosomes in order to reduce
			the number of parameters in fully connected layers that encode
			methylation data
			each modality process individually before conatenting a latent representation, decoder mirrored architecture
			multimodal latent space supervision cancer type
			mRNA and DNAm

			omiEMbedded~\cite{Zhang2021} VAE similar to omiVAE but FC or CNN based
			multitask cancer, survival, demographic gender/age predicition
			miRNA, mRNA, DNAm

			\cite{Hira2021}: similar to omiVAE but on survival task
			use 3 omics: mRNA, CNV, DNAm and combination of only 2
			focus on ovarian cancer
			different tasks after unsupervised training: subtype (SVM), survival (CoxPH)
			intermediate fusion
			multi omics no improvement over single omics equivalent for classification tasks
			% AE and VAE
			\cite{DeepProg} mRNA, miRNA, DNAm, intermediate fusion
			clustering to idneitfy subtypes, labels used to construct a SVM classifier
			In DeepProg, \citeauthor{DeepProg} proposed an ensemble of \gls{ae}; for each omics, an \gls{ae} is used to create a lower dimensional representation~\cite{DeepProg}.
			Survival-associated features are then selected in each latent representation to construct the multimodal representation.
			Feature selection variance threshold 

			\cite{Wissel2021} hierarchical autoencoders
			modality specific autoencoders supervised with a cox loss
			an integration methods to combine the different latent spaces
			intermediate fusion, comparison of different approaches
			concatenation, pooling (mean or max), another level of AE (hierarchical)
			hierarchical methods is the best
			survival tasks (Cox)
			omics CNV, SNV, miRNA, mRNA, DNA, proteins

			Deepathology~\cite{Azarkhalili2019}: multitask model, includes modality translation
			AE and VAE: encode mRNA profile.
			multi task from the latent representation
			coding genes only
			using of cosine similarity for classification tasks ??
			4 tasks: mRNA and miRNA profile prediction, tissue and caner type predicition, jointly trained on all losses

			\cite{Tong2021} autoencoders intermediate fusion
			concatenate latent spaces, or pooling
			for pooling add contraints by macimizing cosine similarity of the latent spaces
			mRNa, MiRNA, DNAm, CNV
			PCA transformation of each modality or threshold on variance
			cancer type or survival

			Stacked AEs~\cite{Wu2022StackedAB}
			mRNA, CNV, clinical
			log rank test to selecte features
			intermediate fusion
			1 AEs per omics, concat latent spaces, new AE, ...
			survival prediction
			multistep predicition

			ConcatAE and CrossAE~\cite{CrossAE} mRNA, miRNA, DNAm, CNV / IF
			task = survival or BRCA subtype
			features selection: variance threshold or PCA dim reduction
			ConcatAE jointly trained
			CrossAE multiple step training: 1 reconstruct the encoded omics, 2 reconstruct the other omics (translation) 3 elementwise average of latent spaces, passed to a classifier
			actually applied only on pairs of modality
			\glspl{ae} are also used with intermediate fusion strategies.
			An \gls{ae} is trained for each modality, and latent spaces are concatenated and fed to a multimodal \gls{ae} or a prediction network~\cite{CrossAE, HierachicalAE}.
			New training strategies like cross-modality translation have been devised with the encoder-decoder architecture.
			After encoding a modality, the decoder reconstructs not only the input modality but also other modalities~\cite{CrossAE}.

			% FCN/MLP

			SALMON~\cite{SALMON} mRNA, miRNA, CNV
			variance threshold on features
			SALMON is a model combining an intermediate fusion strategy with a \gls{mlp}~\cite{SALMON}.
			Omics features are transformed to an eigengene matrix, a weighted average of gene expression of genes in a given set of genes, before being fed to an \gls{mlp}.
			survival task

			MultiSurv~\cite{MultiSurv} indiividual encoders for each modality (CNN for WSI), mlp for others,
			intermediate fusion by pooling the various latent representatino
			WSI encoder pretrained on imagenet
			tested various combination of modalities but always  included clinical information and obtained similar performances to model trained only on clinical data only +mRNA obtain better perf on this specific taks, suggest that survival prediciotn is driven by clinical information (gender, disease treatment, ...)
			survival prediciotn
			WSI, clinical, mRNA, miRNA, DNAm, CNV

			MOLI~\cite{MOLI}
			mRNA, SNV, CNV
			task = drug response, IF
			1 encoder per omics , concat features, passed to a classifier network
			MLP

			Classifying Breast Cancer Subtypes Using Deep Neural Networks Based on Multi-Omics Data~\cite{Lin2020} = DeepMO
			mRNA, DNAm, CNV
			subtype classification ?
			chi2 test to select top features done separately on each omics
			each omics is encoded individually MLP
			IF concat latent representation
			passed to another network for prediction

			% GAN
			\Glspl{gan} have been explored as a way of integrating two omics.
			The generative process is used to construct an enriched feature set combining the information of the other omics and their interaction network~\cite{omicsGAN}.
			miRNA and mRNA, exploit miRNA-mRNA network
			applied on a bipartite graph intermediate Fusion
			survival task

			SADLN~\cite{SADLN} Encoder decoder architecture VAEs + GAN with self attention
			CNV, DNAm, miRNA, mRNA
			Feature selection threshold
			subtype clustering (Gaussian mixture model)
			IF concat latent representation
			compute self attention between samples of the same batch
			add a discriminator

			% cnn 
			An integrative deep learning framework for classifying molecular subtypes of breast cancer~\cite{MohaiminulIslam2020}
			CNV + mRNA
			CNN IF, subtype classification
			explore sharing weight between the 2 branch (ie each omics) but reduced perf

			% self-supervised / contrastive 
			SelfOmics~\cite{selfOmics}
			cancer type classification
			mRNA, DNAm, miRNA
			focus on self supervised training, and pretrining of modality AEs
			obtain 3 latent representation one for each omics
			mRNA and DNAm features are grouped by chromosomes, Fullyconnected only if on the same chr
			losses:
			1. reconstruction loss, from one latent representation reconstruct all the omics
			2. reconstruct the input omics but some features associated to a chr where masked
			3. alignment loss of the 3 latent representation based on CLIP

			cat all latent repre and mlp to predict cancer type = IF
			showed that pretraining encoders help boos perf however the impact of the different self supervised loss is limited on the classification perf
			while they state that their methods is able to handle missing omics data during pretraining by only training the AEs of the available omics they do not evaluate the impact of omics data

			\cite{Cheerla2019}
			miRNA, mRNA, WSI
			intermediate fusion transform each modality in a latent representation
			force similarity of the latent representation, for each pairs of modality that is present similairity based loss
			survival task
			multimodal dropout improve the network’s ability to deal with missing data. In this method, instead of dropping neurons, we drop entire feature vectors corresponding to each modality

			A Variational Information Bottleneck Approach to Multi-Omics Data Integration
			\cite{Lee2021AVI}
			mRNA, DNAm, miRNA, Protein
			learning from incomplete multi-view observations
			omics = views handle missingness patterns
			IF, view specific encoders = obtain a latent representation
			product of experts get a joint representation
			predictor from the joint repr and one predictor for each encoder

			% GNN 
			DeepMOCCA~\cite{Althubaiti_2021}: GCN on PPI graph, concat features and map them to the corresponding nodes
			use a self attention based pooling to identify prognostic markers genes
			omics: mRNA, DNAm, CNV, SNV, clinical
			use of absolute and differential data
			survival cox

			MultiGATAE~\cite{MultiGATAE} GAT + AE
			patient similarity graph for each omics, then apply SNF
			mRNA, DNAm, miRNA
			GAT with the SNF graph for each omics, combine the latent representation with an attention based mechanism
			attention weight is determinded by a linear combination of the graph representation and structure
			latent representation passed to decoder to reconstruct the similairity graph
			early and intermediate fusion
			K-means clustering (cancer subtyping) on the latent representation after training

			Variation of MultiGATAE~\cite{Zhang2022}, instead of representing each patient with its full profile, use autoencoders to embedded each expression profile.
			pass to a GCN where all layers are densely connected, ie a layer is connected to all downstream layers
			compared combination show that the multiomics was the best
			task: predict subtype
			mRNA, DNAm, CNV

			\cite{Yin2022} mRNA, CNV, SNV,
			threshold filtering
			supervised feature slecteion based on LASSO
			patient graph from mRNA data
			non linear transformation of the omics data then concatenation, IF
			maps the new representation to each nodes
			GCN , subtype prediction

			AGMI~\cite{AGMI} GNN based on a multi-edge graph: multi types of edges between two nodes. node = genes , edges a known relation between 2 genes PPI, gene pathway, correlation
			Early fusion concat features, map to nodes
			aggregation from neighbouring nodes in three step, one for each edge type with a GRU as an attention guidance
			CNV, mRNA, SNV drug adequation

			Multi-Omic Graph Transformers for Cancer Classification and Interpretation~\cite{Kaczmarek2021}
			mRNA and miRNA
			GAT on a bipartite graph, miRNA-mRNA [TargetScan]
			positional encoding of the nodes
			feature selection: knowledge and variance threshold
			IF  cancer classification

			SUPREME~\cite{Kesimoglu2022} GCN subtype or survival patient similairity graph 1 per omics
			apply GCN to generate a patient embedding by concat all the patient embedings combination obtain from the different graphs
			feature selection step to create the raw feature embedings
			mRNA, miRNA, DNAm, CNV, SNV
			kept differentially expressed genes

			A multimodal graph neural network framework for cancer molecular subtype classification~\cite{Li2024}
			mRNA, miRNA, CNV, subtype
			feature selection variance threshold
			heterogeneous graph gene-gene interaction from biogrid and miRNA-mRNA from miRDB + metapath connect miRNA if they are connected to the same genes
			3 type of edges
			two types of nodes: genes 2D features (mRNA+CNV), miRNA (1d features)
			nodes features are passed through a FCN to increase their dimension to F
			GCN or GAT
			mixed integration: EF and IF

			Classifying breast cancer using multi-view graph neural network based on multi-omics data~\cite{Ren2024}
			mRNA, DNAm, CNV
			feature selection top based on chi2 + minimum Redundancy Maximum Relevance
			GNN patient similarity
			1 graph and 1 GCN per omics IF
			Fuse the omics based on attention: GAT ?

			% Attention
			TransVCOX: Bridging Transformer Encoder and Pre-trained VAE for Robust Cancer Multi-Omics Survival Analysis~\cite{10385668}
			survival, pretrain a VAE for each omics, used the latent space in a second step for the integration
			IF, use of self-attention on each omics after embedding them embedding to get the correct shape for attention
			survival pred
			mRNA, CNV, microbiome data

			Multimodal attention-based variational autoencoder for clinical risk prediction~\cite{Li2023}
			same as previous but no VAE pretraining

			%DeepKEGG: a multi-omics data integration framework with biological insights for cancer recurrence prediction and biomarker discovery
			MOMA~\cite{MOMA}
			project the different features into a set of modules of same dimensions, each module is represented by a vector
			attention mechanism (cosine similarity) used to focus on modules with high similarity between 2 different modalities
			mRNA, DNAm, cancer type classification
			DNAm CpG mapped to genes
			explainable model, detect important modules related to the predicited phenotype then select the most important features of each modules, pathway enrichment analysis applied afterwards to identify the most important pathways
			intermediate fusion

			MOCAT~\cite{Yao_2024}
			mRNA, DNAm, miRNA
			subtype or grade
			pretrain AEs to have a latent repr for each omics
			supervised latent space
			IF, cross omics fusion
			concat latent repr, passed to a multimodal AEs
			latent space passed to a attention layer, self attention but applied between samples !!

			moBRCA~\cite{moBRCA}
			DNAm, mRNA, miRNA, cancer subtype
			Feature selection based on gene differential expression
			selected features linke to the selected genes in the other omics
			targeted by a miRNAs or methylation of the corresponding promoter
			embedding of the expression profile to high dim. Each scalar feature randomly mapped to a Kdim vector
			compute attention score with tanh activation gate
			IF = concat new representation
			subtype prediction with FC network

			\ifSubfilesClassLoaded{%
				\setlength\rotheadsize{1.45cm}

\begin{longtblr}[
	caption = {examples multi omics},
	entry = {short caption},
	note{a} = {EF = Early Fusion, LF = Late Fusion, IF = Intermediate Fusion},
	note{1} = {As a classification task, estimate survival groups},
	note{2} = {based on the result of a clustering},
	note{3} = {A patient hypergraph},
	note{4} = {use the latent space to construct a second model (\glsxtrshort{svm}, CoxPH)},
	note{5} = {only pairs of modalities are considered},
	note{6} = {also inlcudes \glsxtrshort{wsi}}, 
	]{
	colspec = {Q[c,m]Q[c,m]Q[c,m]Q[c,m]Q[c,m]Q[c,m]Q[c,m]Q[c,m]Q[c,m, wd=2cm]Q[c,m]Q[c,m, wd=1.5cm]Q[l,m,, wd=2.8cm]},%
	cell{1}{2-8} = {halign=c,cmd=\rothead},
	row{2-Z} = {font=\small},%
	cell{2-Z}{2-7} = {font=\scriptsize},
	row{1} = {valign=m},
	row{14} = {belowsep=10pt},
	row{15} = {abovesep=10pt},
	hline{15} = {dashed, 0.75pt},
	row{18} = {belowsep=10pt},
	row{19} = {abovesep=10pt},
	hline{19} = {dashed, 0.75pt},
	column{2-7} = {colsep=1pt},
	column{1} = {colsep=3pt},
	column{8-12} = {colsep=5pt},
	hline{1,Z} = {2pt},%
			hline{2} = {1pt},%
			rowhead = 1, %
			rowfoot = 0%
		}
	Ref.                   & mRNA                  & miRNA                 & DNAm                  & Protein   & CNV                   & SNV       & Fusion\TblrNote{a} & Task                       & Arch.              & Graph               & Feature engineering                                               \\
	\cite{Althubaiti_2021} & \faCircle             &                       & \faCircle             &           & \faCircle             & \faCircle & EF                 & surv                       & \gls{gat}          & PPI                 & Differential expression                                           \\
	\cite{Chaudhary2018}   & \faCircle             & \faCircle             & \faCircle             &           &                       &           & EF                 & surv\TblrNote{1}           & \glsxtrshort{ae}   &                     & Survival related features                                         \\
	\cite{Lee2020}         & \faCircle             & \faCircle             & \faCircle             &           & \faCircle             &           & EF                 & surv\TblrNote{1}           & \glsxtrshort{ae}   &                     & Survival related features                                         \\
	\cite{Guo2020}         & \faCircle             & \faCircle             &                       &           & \faCircle             &           & EF                 & subtype\TblrNote{2}        & \glsxtrshort{ae}   &                     &                                                                   \\
	\cite{Yu2022}          & \faCircle             & \faCircle             &                       &           &                       &           & EF                 & subtype\TblrNote{2}        & \glsxtrshort{ae}   &                     &                                                                   \\
	\cite{DeepOmix}        & \faCircle             &                       & \faCircle             &           & \faCircle             & \faCircle & EF                 & surv                       & \glsxtrshort{mlp}  &                     & Feature selection (knowledge)                                     \\
	\cite{PNet}            &                       &                       &                       &           & \faCircle             & \faCircle & EF                 & surv                       & \glsxtrshort{mlp}  &                     & Feature selection (knowledge)                                     \\
	\cite{Hao2019}         & \faCircle             &                       & \faCircle             &           & \faCircle             &           & EF                 & surv                       & \glsxtrshort{mlp}  &                     & Feature selection (threshold)                                     \\
	\cite{PathCNN}         & \faCircle             &                       & \faCircle             &           & \faCircle             &           & EF                 & surv\TblrNote{1}           & \glsxtrshort{mlp}  &                     & PCA                                                               \\
	\cite{MoGCN}           & \faCircle             &                       &                       & \faCircle & \faCircle             &           & EF                 & subtype                    & \glsxtrshort{gnn}  & \glsxtrshort{snf}   & \glsxtrshort{ae}                                                  \\
	\cite{Guo2023}         & \faCircle             &                       & \faCircle             &           & \faCircle             &           & EF                 & subtype                    & \glsxtrshort{gnn}  & PPI                 & DisGenet + MAD                                                    \\
	\cite{AGMI}         & \faCircle             &                       &              &           & \faCircle             &  \faCircle         & EF                 & drug                    & \glsxtrshort{gnn}  & PPI + pathway + correlation                 &                                 \\
	\cite{SubtypeFormer}   & \faCircle             & \faCircle             & \faCircle             &           & \faCircle             &           & EF                 & subtype\TblrNote{2}        & Trans.             &                     &                                                                   \\
	\cite{DeepPathNet}     & \faCircle             &                       &                       &           & \faCircle             & \faCircle & EF                 & {subtype                                                                                                                                  \\ cancer}                & Trans.       &  & Feature selection (knowledge) \\
	\cite{MOGONET}         & \faCircle             & \faCircle             & \faCircle             &           &                       &           & LF                 & subtype                    & \glsxtrshort{gnn}  & patient             &                                                                   \\
	\cite{MODILM}          & \faCircle             & \faCircle             & \faCircle             &           &                       &           & LF                 & subtype                    & \glsxtrshort{gat}  & patient             & Feature selection (variance threshold)                            \\
	\cite{Sun2019}         &                       &                       &                       &           &                       &           & LF                 &                            & \glsxtrshort{mlp}  &                     &                                                                   \\
	\cite{Wang2024}        & \faCircle             & \faCircle             & \faCircle             &           &                       &           & LF                 & subtype                    & \glsxtrshort{hgcn} & Patient\TblrNote{3} &                                                                   \\
	\cite{customics}       & \faCircle             &                       & \faCircle             &           & \faCircle             &           & IF                 & {subtype                                                                                                                                  \\ cancer \\ survival}                & \glsxtrshort{ae} + \glsxtrshort{vae}       &  &  \\
	\cite{Zhang2019}       & \faCircle             &                       & \faCircle             &           &                       &           & IF                 & cancer                     & \glsxtrshort{vae}  &                     &                                                                   \\
	\cite{Zhang2021}       & \faCircle             & \faCircle             & \faCircle             &           &                       &           & IF                 & {cancer                                                                                                                                   \\ survival}               & \glsxtrshort{vae}      &  &  \\
	\cite{Hira2021}        & \faCircle             &                       & \faCircle             &           & \faCircle             &           & IF                 & {survival\TblrNote{4}                                                                                                                     \\ subtype\TblrNote{4}}              & \glsxtrshort{vae}      &  &  \\
	\cite{DeepProg}        & \faCircle             & \faCircle             & \faCircle             &           &                       &           & IF                 & subtype\TblrNote{4}        & \glsxtrshort{ae}   &                     & Feature selection (threshold) Survival-associated latent features \\
	\cite{Wissel2021}      & \faCircle             & \faCircle             & \faCircle             & \faCircle & \faCircle             & \faCircle & IF                 & survival                   & \glsxtrshort{ae}   &                     &                                                                   \\
	\cite{Azarkhalili2019} & \faCircle             & \faCircle             &                       &           &                       &           & IF                 & cancer + omics translation & {\glsxtrshort{ae}                                                                                            \\ \glsxtrshort{vae}}      &  &   \\
	\cite{Tong2021}        & \faCircle             & \faCircle             & \faCircle             &           & \faCircle             &           & IF                 & {cancer                                                                                                                                   \\ survival}             & \glsxtrshort{ae}    &  & {Feature selection (threshold) \\ PCA}   \\
	\cite{Wu2022StackedAB} & \faCircle             &                       &                       &           & \faCircle             &           & IF                 & survival                   & \glsxtrshort{ae}   &                     & Feature selection (log-rank test)                                 \\
	\cite{CrossAE}         & \faCircle\TblrNote{5} & \faCircle\TblrNote{5} & \faCircle\TblrNote{5} &           & \faCircle\TblrNote{5} &           & IF                 & survival                   & \glsxtrshort{ae}   &                     & {Feature selection (threshold)                                    \\ PCA}   \\
	\cite{MultiSurv}         & \faCircle & \faCircle & \faCircle &           & \faCircle &           & IF                 & survival                   & \glsxtrshort{mlp}   &                     &    \\
	\cite{MOLI}         & \faCircle &  &  &           & \faCircle &\faCircle           & IF                 & drug                   & \glsxtrshort{mlp}   &                     &    \\
	\cite{Lin2020}         & \faCircle &  &\faCircle  &           & \faCircle &           & IF                 & subtype                   & \glsxtrshort{mlp}   &                     & Feature selection (\(\chi^{2}\))   \\

	\cite{SALMON}          & \faCircle             & \faCircle             &                       &           & \faCircle             &           & IF                 & surv                       & \glsxtrshort{mlp}  &                     & Feature selection (threshold)   + knowledge                                  \\
	\cite{MohaiminulIslam2020}          & \faCircle             &              &                       &           & \faCircle             &           & IF                 & subtype                       & \glsxtrshort{cnn}  &                     &                                 \\
	\cite{omicsGAN}          & \faCircle             &  \faCircle             &                       &           &              &           & IF                 & survival                       & \glsxtrshort{gan}  &                     &                                 \\
	\cite{Cheerla2019}          & \faCircle\TblrNote{6}             &  \faCircle\TblrNote{6}             &                       &           &              &           & IF                 & survival                       & \glsxtrshort{mlp}  &                     &                                 \\
	\cite{selfOmics}          & \faCircle            &  \faCircle             &   \faCircle                    &           &              &           & IF                 & cancer                       & \glsxtrshort{ae}  &                     &                                 \\
	\cite{Kesimoglu2022}          & \faCircle            &  \faCircle             &   \faCircle                    &           & \faCircle             &  \faCircle         & IF                 & {subtype \\ survival}                       & \glsxtrshort{gcn}  & Patient                    &  Feature selection (\glsxtrshort{rf})                               \\
	\cite{MultiGATAE}          & \faCircle            &  \faCircle             &   \faCircle                    &           &              &           & IF                 & subtype\TblrNote{2}                       & \glsxtrshort{gat}  & Patient                    &                               \\
	\cite{Zhang2022}          & \faCircle            &               &   \faCircle                    &           &  \faCircle            &           & IF                 & subtype                      & \glsxtrshort{gat}  & Patient                    &                               \\
	\cite{Kaczmarek2021}          & \faCircle            & \faCircle              &                       &           &              &           & IF                 & cancer                      & \glsxtrshort{gat}  & miRNA-mRNA                    &  Feature selection (variance threshold and knowledge)                             \\
	\cite{Li2024}          & \faCircle            & \faCircle              &                       &           &  \faCircle             &           & EF + IF                 & subtype                      & \glsxtrshort{gat}  & BioGRID + miRDB                    &  Feature selection (variance threshold)                             \\
\end{longtblr}%
			}{
				\setlength\rotheadsize{1.45cm}

\begin{longtblr}[
	caption = {examples multi omics},
	entry = {short caption},
	note{a} = {EF = Early Fusion, LF = Late Fusion, IF = Intermediate Fusion},
	note{1} = {As a classification task, estimate survival groups},
	note{2} = {based on the result of a clustering},
	note{3} = {A patient hypergraph},
	note{4} = {use the latent space to construct a second model (\glsxtrshort{svm}, CoxPH)},
	note{5} = {only pairs of modalities are considered},
	note{6} = {also inlcudes \glsxtrshort{wsi}}, 
	]{
	colspec = {Q[c,m]Q[c,m]Q[c,m]Q[c,m]Q[c,m]Q[c,m]Q[c,m]Q[c,m]Q[c,m, wd=2cm]Q[c,m]Q[c,m, wd=1.5cm]Q[l,m,, wd=2.8cm]},%
	cell{1}{2-8} = {halign=c,cmd=\rothead},
	row{2-Z} = {font=\small},%
	cell{2-Z}{2-7} = {font=\scriptsize},
	row{1} = {valign=m},
	row{14} = {belowsep=10pt},
	row{15} = {abovesep=10pt},
	hline{15} = {dashed, 0.75pt},
	row{18} = {belowsep=10pt},
	row{19} = {abovesep=10pt},
	hline{19} = {dashed, 0.75pt},
	column{2-7} = {colsep=1pt},
	column{1} = {colsep=3pt},
	column{8-12} = {colsep=5pt},
	hline{1,Z} = {2pt},%
			hline{2} = {1pt},%
			rowhead = 1, %
			rowfoot = 0%
		}
	Ref.                   & mRNA                  & miRNA                 & DNAm                  & Protein   & CNV                   & SNV       & Fusion\TblrNote{a} & Task                       & Arch.              & Graph               & Feature engineering                                               \\
	\cite{Althubaiti_2021} & \faCircle             &                       & \faCircle             &           & \faCircle             & \faCircle & EF                 & surv                       & \gls{gat}          & PPI                 & Differential expression                                           \\
	\cite{Chaudhary2018}   & \faCircle             & \faCircle             & \faCircle             &           &                       &           & EF                 & surv\TblrNote{1}           & \glsxtrshort{ae}   &                     & Survival related features                                         \\
	\cite{Lee2020}         & \faCircle             & \faCircle             & \faCircle             &           & \faCircle             &           & EF                 & surv\TblrNote{1}           & \glsxtrshort{ae}   &                     & Survival related features                                         \\
	\cite{Guo2020}         & \faCircle             & \faCircle             &                       &           & \faCircle             &           & EF                 & subtype\TblrNote{2}        & \glsxtrshort{ae}   &                     &                                                                   \\
	\cite{Yu2022}          & \faCircle             & \faCircle             &                       &           &                       &           & EF                 & subtype\TblrNote{2}        & \glsxtrshort{ae}   &                     &                                                                   \\
	\cite{DeepOmix}        & \faCircle             &                       & \faCircle             &           & \faCircle             & \faCircle & EF                 & surv                       & \glsxtrshort{mlp}  &                     & Feature selection (knowledge)                                     \\
	\cite{PNet}            &                       &                       &                       &           & \faCircle             & \faCircle & EF                 & surv                       & \glsxtrshort{mlp}  &                     & Feature selection (knowledge)                                     \\
	\cite{Hao2019}         & \faCircle             &                       & \faCircle             &           & \faCircle             &           & EF                 & surv                       & \glsxtrshort{mlp}  &                     & Feature selection (threshold)                                     \\
	\cite{PathCNN}         & \faCircle             &                       & \faCircle             &           & \faCircle             &           & EF                 & surv\TblrNote{1}           & \glsxtrshort{mlp}  &                     & PCA                                                               \\
	\cite{MoGCN}           & \faCircle             &                       &                       & \faCircle & \faCircle             &           & EF                 & subtype                    & \glsxtrshort{gnn}  & \glsxtrshort{snf}   & \glsxtrshort{ae}                                                  \\
	\cite{Guo2023}         & \faCircle             &                       & \faCircle             &           & \faCircle             &           & EF                 & subtype                    & \glsxtrshort{gnn}  & PPI                 & DisGenet + MAD                                                    \\
	\cite{AGMI}         & \faCircle             &                       &              &           & \faCircle             &  \faCircle         & EF                 & drug                    & \glsxtrshort{gnn}  & PPI + pathway + correlation                 &                                 \\
	\cite{SubtypeFormer}   & \faCircle             & \faCircle             & \faCircle             &           & \faCircle             &           & EF                 & subtype\TblrNote{2}        & Trans.             &                     &                                                                   \\
	\cite{DeepPathNet}     & \faCircle             &                       &                       &           & \faCircle             & \faCircle & EF                 & {subtype                                                                                                                                  \\ cancer}                & Trans.       &  & Feature selection (knowledge) \\
	\cite{MOGONET}         & \faCircle             & \faCircle             & \faCircle             &           &                       &           & LF                 & subtype                    & \glsxtrshort{gnn}  & patient             &                                                                   \\
	\cite{MODILM}          & \faCircle             & \faCircle             & \faCircle             &           &                       &           & LF                 & subtype                    & \glsxtrshort{gat}  & patient             & Feature selection (variance threshold)                            \\
	\cite{Sun2019}         &                       &                       &                       &           &                       &           & LF                 &                            & \glsxtrshort{mlp}  &                     &                                                                   \\
	\cite{Wang2024}        & \faCircle             & \faCircle             & \faCircle             &           &                       &           & LF                 & subtype                    & \glsxtrshort{hgcn} & Patient\TblrNote{3} &                                                                   \\
	\cite{customics}       & \faCircle             &                       & \faCircle             &           & \faCircle             &           & IF                 & {subtype                                                                                                                                  \\ cancer \\ survival}                & \glsxtrshort{ae} + \glsxtrshort{vae}       &  &  \\
	\cite{Zhang2019}       & \faCircle             &                       & \faCircle             &           &                       &           & IF                 & cancer                     & \glsxtrshort{vae}  &                     &                                                                   \\
	\cite{Zhang2021}       & \faCircle             & \faCircle             & \faCircle             &           &                       &           & IF                 & {cancer                                                                                                                                   \\ survival}               & \glsxtrshort{vae}      &  &  \\
	\cite{Hira2021}        & \faCircle             &                       & \faCircle             &           & \faCircle             &           & IF                 & {survival\TblrNote{4}                                                                                                                     \\ subtype\TblrNote{4}}              & \glsxtrshort{vae}      &  &  \\
	\cite{DeepProg}        & \faCircle             & \faCircle             & \faCircle             &           &                       &           & IF                 & subtype\TblrNote{4}        & \glsxtrshort{ae}   &                     & Feature selection (threshold) Survival-associated latent features \\
	\cite{Wissel2021}      & \faCircle             & \faCircle             & \faCircle             & \faCircle & \faCircle             & \faCircle & IF                 & survival                   & \glsxtrshort{ae}   &                     &                                                                   \\
	\cite{Azarkhalili2019} & \faCircle             & \faCircle             &                       &           &                       &           & IF                 & cancer + omics translation & {\glsxtrshort{ae}                                                                                            \\ \glsxtrshort{vae}}      &  &   \\
	\cite{Tong2021}        & \faCircle             & \faCircle             & \faCircle             &           & \faCircle             &           & IF                 & {cancer                                                                                                                                   \\ survival}             & \glsxtrshort{ae}    &  & {Feature selection (threshold) \\ PCA}   \\
	\cite{Wu2022StackedAB} & \faCircle             &                       &                       &           & \faCircle             &           & IF                 & survival                   & \glsxtrshort{ae}   &                     & Feature selection (log-rank test)                                 \\
	\cite{CrossAE}         & \faCircle\TblrNote{5} & \faCircle\TblrNote{5} & \faCircle\TblrNote{5} &           & \faCircle\TblrNote{5} &           & IF                 & survival                   & \glsxtrshort{ae}   &                     & {Feature selection (threshold)                                    \\ PCA}   \\
	\cite{MultiSurv}         & \faCircle & \faCircle & \faCircle &           & \faCircle &           & IF                 & survival                   & \glsxtrshort{mlp}   &                     &    \\
	\cite{MOLI}         & \faCircle &  &  &           & \faCircle &\faCircle           & IF                 & drug                   & \glsxtrshort{mlp}   &                     &    \\
	\cite{Lin2020}         & \faCircle &  &\faCircle  &           & \faCircle &           & IF                 & subtype                   & \glsxtrshort{mlp}   &                     & Feature selection (\(\chi^{2}\))   \\

	\cite{SALMON}          & \faCircle             & \faCircle             &                       &           & \faCircle             &           & IF                 & surv                       & \glsxtrshort{mlp}  &                     & Feature selection (threshold)   + knowledge                                  \\
	\cite{MohaiminulIslam2020}          & \faCircle             &              &                       &           & \faCircle             &           & IF                 & subtype                       & \glsxtrshort{cnn}  &                     &                                 \\
	\cite{omicsGAN}          & \faCircle             &  \faCircle             &                       &           &              &           & IF                 & survival                       & \glsxtrshort{gan}  &                     &                                 \\
	\cite{Cheerla2019}          & \faCircle\TblrNote{6}             &  \faCircle\TblrNote{6}             &                       &           &              &           & IF                 & survival                       & \glsxtrshort{mlp}  &                     &                                 \\
	\cite{selfOmics}          & \faCircle            &  \faCircle             &   \faCircle                    &           &              &           & IF                 & cancer                       & \glsxtrshort{ae}  &                     &                                 \\
	\cite{Kesimoglu2022}          & \faCircle            &  \faCircle             &   \faCircle                    &           & \faCircle             &  \faCircle         & IF                 & {subtype \\ survival}                       & \glsxtrshort{gcn}  & Patient                    &  Feature selection (\glsxtrshort{rf})                               \\
	\cite{MultiGATAE}          & \faCircle            &  \faCircle             &   \faCircle                    &           &              &           & IF                 & subtype\TblrNote{2}                       & \glsxtrshort{gat}  & Patient                    &                               \\
	\cite{Zhang2022}          & \faCircle            &               &   \faCircle                    &           &  \faCircle            &           & IF                 & subtype                      & \glsxtrshort{gat}  & Patient                    &                               \\
	\cite{Kaczmarek2021}          & \faCircle            & \faCircle              &                       &           &              &           & IF                 & cancer                      & \glsxtrshort{gat}  & miRNA-mRNA                    &  Feature selection (variance threshold and knowledge)                             \\
	\cite{Li2024}          & \faCircle            & \faCircle              &                       &           &  \faCircle             &           & EF + IF                 & subtype                      & \glsxtrshort{gat}  & BioGRID + miRDB                    &  Feature selection (variance threshold)                             \\
\end{longtblr}%
			}

			%A benchmark study of deep learning‑based multi‑omics data fusion methods for cancer
			precise the CCL of this study



\section{Model interpretability}
	% general about interpretability/explanability
	% fix the vocabulary that is used
	% present different methods, advantage drawbacks

	% way to introduce counterfactuals

	% remind what are counterfactuals in plain english: what if
	% show equation
	% desired property of CFs + construct the optimization problem
	% how to solve it: gradient descent show algo
	% discuss lambda
	% show some methods

	% need to talk about adversarial examples, details in annex?
	% how to have a model robust to adv attacks, defenses, adv training

\end{document}
