\documentclass[../main.tex]{subfiles}
\usepackage{silence}
\WarningFilter{glossaries}{No \printglossary or \printglossaries found}

\begin{document}
\ifSubfilesClassLoaded{%
	\graphicspath{{figures/2-Background/}}%
}{
	\graphicspath{{../figures/2-Background/}}%
}
\chapter{Background}\label{chap:background}
\minitocpage

\section{Deep-Learning}
present the different architecture that are available and how to train them.
supervised, self supervised
MLP
AE/VAE
CNN1D/CNN2D
GNN
Attention (Transformers), self and cross (causal? not used in this thesis)
fig with the correct dim
dynamic mlp
some characteristics, memory footprint, visualization

how to train a model, loss definition
regularization
optimization, gradient descent

hyperparameters, search, cost of search
\section{Cancer biology}
%https://www.cell.com/fulltext/S0092-8674(11)00127-9
%https://en.wikipedia.org/wiki/The_Hallmarks_of_Cancer#Emerging_Hallmarks
%file:///nhome/siniac/abeaude/T%C3%A9l%C3%A9chargements/nrc2827.pdf
%https://en.wikipedia.org/wiki/Cancer
%https://www.ncbi.nlm.nih.gov/pmc/articles/PMC10618731/pdf/1142.pdf
%https://www.cancer.gov/about-cancer/understanding/what-is-cancer
%https://www.who.int/en/health-topics/cancer#tab=tab_1
%file:///nhome/siniac/abeaude/T%C3%A9l%C3%A9chargements/Molecular%20Biology%20of%20Human%20Cancers_%20Second%20Edition%20--%20Wolfgang%20A_%20Schulz%20--%202,%202023%20--%20Springer,%20Springer%20Nature%20Switzerland%20AG%20--%209783031162855%20--%20c4af4ff1d84151f27d34687b2ca398f9%20--%20Anna%E2%80%99s%20Archive.pdf
This section does not presume to be a complete presentation of cancer biology, but aims to provide the necessary keys to understand cancer origins and the different molecular components involved.
present normal function of a cell with various components, DNA to protein in a cell. Protein = structure = function.
How this process is regulated: present central dogma, cell regulation
cancer = dysregulation of the functionning of a cell.
origin: multiscale, can happens at diferent place of the process.
talk about cancer hallmarks

cancer origin, promotion, growth metastasis = dvt of cancer
\subsection{Signaling in healthy cells}
\subsection{cancer}


\section{Omics data}
\subsection{Genomics}
\subsection{Transcriptomics}
\subsection{Epigenomics}
\subsection{Proteomics}
\subsection{Other omics}
different components, access with different technologies
present the different technologies
particularity of omics data, low samples highly dimensionnal
various distribution

\section{Available datasets}
\subsection{GEO}
archs4, batch effect, no multiomics, various conditions, various worflow of analyses.
could have unpaired data, but challenges are then different
\subsection{TCGA}
talk about cancer stage, and why not use it as a predicitve endpoint
possible tasks: healthy or cancerous samples, cancer type, subtype [limits], survival [death, relapse: tumour recurrence] metastasis,
% https://www.ncbi.nlm.nih.gov/pmc/articles/PMC8368987/
primary origin of a metastasis % https://www.nature.com/articles/s41467-019-13825-8
predict primary or metastasis % https://www.frontiersin.org/articles/10.3389/fmolb.2022.913602/full
% https://www.nature.com/articles/s41598-023-35842-w

give some stats of the different category present

icgc, european initiative? less data, not as strong.
\subsection{CCLE}

stat + labels = possible task

\end{document}
