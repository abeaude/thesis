\documentclass[../main.tex]{subfiles}
\usepackage{silence}
\WarningFilter{glossaries}{No \printglossary or \printglossaries found}
\robExtConfigure{disable externalization}
\begin{document}
\ifSubfilesClassLoaded{%
	\graphicspath{{figures/7-Interpretability/}}%
	\setcounter{chapter}{6}%
}{
	\graphicspath{{../figures/7-Interpretability/}}%
}
\chapter{Explain with counterfactuals}\label{chap:counterfactuals}
\minitocpage
\section{Methods}
	\subsection{GAN}
		\begin{figure}[htbp]
			\centering
			\begin{tikzpicture}[inner sep=0pt, very thick]
				\node at (0,0) [square, draw, minimum size=1.5cm] (xin) {\(x\)};
				\node (gen) [right=1cm of xin.east, anchor=west, rounded rectangle, rounded rectangle west arc=none, draw, minimum height=1cm, minimum width=11mm] {\(G\)};
				\node (delta) [right=1cm of gen.east, anchor=west,square, draw, minimum size=1.5cm] {\(\delta\)};
				\node (add) [right=5mm of delta.east, anchor=west, draw, circle, minimum size=7.5mm] {};
				\draw (add.north) -- (add.south);
				\draw (add.west) -- (add.east);
				\node (xcf) [right=5mm of add.east, anchor=west,square, draw, minimum size=1.5cm] {\(x^{\text{CF}}\)};
				\node (cl1) [right=1cm of xcf.east, anchor=west, rounded rectangle, rounded rectangle west arc=none, draw, minimum height=1cm, minimum width=11mm] {\(\operatorname{Cl}\)};
				\node (cly) [right=6mm of cl1.east, anchor=west] {\(\hat{y}\)};
				\node (ycf) [right=1.5cm of cly.east, anchor=west] {\(y^{\text{CF}}\)};

				\node (dis) [below=7mm of cl1.south, anchor=north, rounded rectangle, rounded rectangle west arc=none, draw, minimum height=1cm, minimum width=11mm] {\(D\)};
				\node (realfake) at (dis.east -| cly.south)  {\(s\)};
				\node[align=left] (labelrealfake) at (realfake.east -| ycf.south) {0 \\ 1};

				\node (cl2) [above=7mm of cl1.north, anchor=south, rounded rectangle, rounded rectangle west arc=none, draw, minimum height=1cm, minimum width=11mm] {\(\operatorname{Cl}_{T}\)};
				\node (cl2y) at (cl2.east -| cly.north)  {\(\hat{y}_{T}\)};
				\node (ytissue) at (cl2y.east -| ycf.north) {\(y_{T}\)};

				\node (xreal) at (xcf.south |- dis.east) [draw, square, minimum size=1.5cm, double copy shadow={shadow xshift=-.5ex, shadow yshift=.5ex, opacity=.5}, fill=white] {\(\symcal{X}\)};

				\draw[-stealth] (xin.east) -- (gen.west);
				\draw[-stealth] (gen.east) -- (delta.west);
				\draw[-stealth] (delta.east) -- (add.west);
				\draw[-stealth]  (xin.south) -- ++(0,-5mm) -| (add.south);
				\draw[-stealth] (add.east) -- (xcf.west);
				\draw[-stealth] (xcf.east) -- (cl1.west);
				\draw[-stealth] (xcf.east) -- ++(4mm, 0) |- ([yshift=2mm]dis.west);
				\draw[-stealth] ([yshift=-2mm]xreal.east) -- ([yshift=-2mm]dis.west);
				\draw[-stealth] (xcf.east) -- ++(4mm, 0) |- (cl2.west);
				\draw[-stealth] (cl1.east) -- (cly.west);
				\draw[-stealth] (cl2.east) -- (cl2y.west);
				\draw[-stealth] (dis.east) -- (realfake.west);
				\draw[stealth-stealth, dashed] (cly.east) -- node [midway,above, yshift=0.5ex] {\(\symcal{L}_{\text{CE}}\)}  (ycf.west);
				\draw[stealth-stealth, dashed] (cl2y.east) -- node [midway,above, yshift=0.5ex] {\(\symcal{L}_{\text{CE}}\)}  (ytissue.west);
				\draw[stealth-stealth, dashed] (realfake.east) -- node [midway,above, yshift=0.5ex] {\(\symcal{L}_{\text{GAN}}\)}  (labelrealfake.west);
			\end{tikzpicture}
			\caption{Architecture of the counterfactual GAN}\label{fig:gan_cf}
		\end{figure}
	\subsection{Metrics}
\section{Preliminary results}
	\subsection{Adversarial study}
		adversarial training used this package
		comparison of different models, justification problem BatchNorm
		how did we choose epsilon (distance class)
		identify the optimal number of training steps for each model
		compare different epsilon attacks
		targeted vs untargeted
		Metrics
		neighbors characterization
		\cite{SzegedyZSBEGF13}
		\begin{table}[htbp]
			\sisetup{round-mode=places,round-precision=2}
			\caption{Some LEGEND, a robust model corresponds to a model that have been adversarially trained.}\label{tab:metrics_adv}
			\csvreader[
				no head,
				centered tabularray={%
						colspec={%
								Q[c,m]
								Q[si={table-format=2.2,table-number-alignment=center}, wd=1.4cm]%
								Q[si={table-format=2.2,table-number-alignment=center}, wd=1.4cm]%
								Q[si={table-format=2.2,table-number-alignment=center}, wd=1.4cm]%
								Q[si={table-format=2.2,table-number-alignment=center}, wd=1.4cm]%
								Q[si={table-format=2.2,table-number-alignment=center}, wd=1.6cm]%
								Q[si={table-format=2.2,table-number-alignment=center}, wd=1.6cm]%
							},%
						row{1-2} = {guard},
						row{2} = {c,m},
						row{3-Z} = {font=\small},
						row{1} = {font=\bfseries},
						cell{1}{2} = {r=1,c=2}{c, wd=2.8cm},
						cell{1}{4} = {r=1,c=2}{c, wd=2.8cm},
						cell{1}{6} = {r=1,c=2}{c, wd=3.2cm},
						hline{1} = {2-Z}{2pt},%
						hline{Z} = {2pt},%
						hline{2-3} = {1pt},%
						column{even} = {rightsep=2pt},
						column{odd[3-Z]} = {leftsep=2pt},
						cell{3-Z}{1} = {font=\normalsize}
					}
			]{%
				\ifSubfilesClassLoaded{%
					data/AdversarialMetrics.csv%
				}{%
					../data/AdversarialMetrics.csv%
				}%
			}{}{\csvlinetotablerow}
		\end{table}
	\subsection{Counterfactuals}







		counterfactuals, could we compare the proposed treatments with, when available, the treatment scheme of the patient.

\end{document}
