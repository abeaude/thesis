\documentclass[../main.tex]{subfiles}
\usepackage{silence}
\WarningFilter{glossaries}{No \printglossary or \printglossaries found}

\begin{document}
\ifSubfilesClassLoaded{%
	\graphicspath{{figures/8-Conclusions/}}%
	\setcounter{chapter}{7}%
}{
	\graphicspath{{../figures/8-Conclusions/}}%
}
\chapter{Conclusions}
\minitocpage

https://www.science.org/doi/10.1126/science.abc4552

Lung Cancer Subtype Diagnosis using Weakly-paired Multi-omics Data~\cite{Wang2022}
subtype; miRNA, SNV, DNAm, WSI
IF
encode each omics individually with an attention based encoder
how are obtained the attention weight ??
from latent representation, predict the subtype (first loss component)
how they handle missing omics: GAN based on the combination of the available modalities

\ifSubfilesClassLoaded{%
	\inputminted[breaklines, autogobble, linenos, numberblanklines=false]{python}{figures/8-Conclusions/ScalarAttention.py}%
}{
	\inputminted[breaklines]{python}{figures/8-Conclusions/ScalarAttention.py}%
}

missingness
A Variational Information Bottleneck Approach to Multi-Omics Data Integration
\cite{Lee2021AVI}
mRNA, DNAm, miRNA, Protein
learning from incomplete multi-view observations
omics = views handle missingness patterns
IF, view specific encoders = obtain a latent representation
product of experts get a joint representation
predictor from the joint repr and one predictor for each encoder

\end{document}
