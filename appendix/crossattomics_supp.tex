\chapter{CrossAttOmics supplementary}\label{chap:crossattomics_appendix}

\section{Experiments}

 \subsection{Architecture and training details}\label{sec:crossattomics_training_details}
     For a comprehensive and comparative evaluation, we chose various deep learning architectures with different integration strategies: \gls{mlp} early fusion (MLP EF), \gls{mlp} intermediate fusion (MLP IF), AttOmics early fusion (AttOmics EF), AttOmics intermediate fusion (AttOmics IF), \gls{gnn} early fusion (GNN EF), P-NET and MOGONET\@.
     We also considered single-omics architecture for comparison: attention-based model~(AttOmics), \gls{mlp} and \gls{gnn}.

     The \gls{mlp} is based on a \gls{fcn} with \gls{relu} activations.
     For the \gls{mlp} EF architecture, input modalities are concatenanted before being passed to a \gls{mlp}.
     For the \gls{mlp} IF architecture, the modalities encoders were constructed similarly to the unimodal \gls{mlp} architecture described previously.
     Unimodal latent representations are concatenated and passed to a 2-layers \gls{fcn} with \gls{relu} activations.

     AttOmics is an attention-based architecture.
     Input is randomly split in groups, then each group in embed in a latent space with a group-specific \gls{fcn}.
     Groups are enriched with information from other groups with \gls{mhsa}.
     For the AttOmics IF architecture, the modalities encoders were constructed similarly to the unimodal AttOmics architecture described previously.
     Unimodal latent representations are concatenated and passed to a 3-layers \gls{fcn} with \gls{relu} activations.

     For the \gls{gnn} architecture the \gls{ppi} graph is based on data available in the STRING database and was constructed by retaining only high-confidence links: edges with a score higher than 700.
     We used graph convolutions described in Kipf et al., with a single input and output channel for unimodal training.
     All the nodes are concatenated and passed to a 2-layers \gls{fcn} with a \gls{relu} activation.
     For the \gls{gnn} EF, the number of channels was set to the number of modalities used for the training.

     In all the architectures the output layer has a \(\operatorname{softmax}\) activation.

     CrossAttOmics is trained using the SGD optimizer with a learning rate of 0.001, a momentum of 0.9, and a batch size of 512.
     Our 2-phase training allows us to train each encoder with more training examples.
     An ablation study showed that the \gls{mhsa} after concatenating the cross-attention outputs was not necessary.
     When creating the different splits, we paid attention to not including examples used in the pre-training phase in our final test set.
     The maximum number of epochs is set to 200.
     An early stopping strategy is deployed to avoid over-fitting with a patience of 10 and a delta of 0.001 on the validation loss between two epochs.
     Models were evaluated with the accuracy metric; other classification metrics are available in the supplementary file.
     All models were trained on an Nvidia GeForce RTX 3090.

     \newpage
\section{Supplementary figures}

 \begin{table}[htbp]
     \centering
     \caption{Distribution of the samples across cancer and splits for (\subref{tab:tcga}) TCGA and (\subref{tab:ccle}) CCLE.}
     \begin{subtable}[t]{0.45\textwidth}
         \subcaption{}
         \begin{tabular}{lrrrr}
             \toprule
             Cancer & Train & Validation & Test & Total \\
             \midrule
             BLCA   & 222   & 48         & 48   & 318   \\
             BRCA   & 574   & 123        & 124  & 821   \\
             CESC   & 113   & 24         & 25   & 162   \\
             COAD   & 230   & 50         & 50   & 330   \\
             HNSC   & 228   & 49         & 49   & 326   \\
             KIRC   & 171   & 37         & 37   & 245   \\
             KIRP   & 145   & 31         & 32   & 208   \\
             LGG    & 296   & 64         & 64   & 424   \\
             LIHC   & 118   & 25         & 26   & 169   \\
             LUAD   & 238   & 51         & 52   & 341   \\
             LUSC   & 205   & 44         & 44   & 293   \\
             OV     & 182   & 39         & 39   & 260   \\
             PRAD   & 234   & 50         & 51   & 335   \\
             SARC   & 143   & 31         & 31   & 205   \\
             SKCM   & 227   & 49         & 49   & 325   \\
             STAD   & 226   & 49         & 49   & 324   \\
             THCA   & 254   & 55         & 55   & 364   \\
             UCEC   & 288   & 62         & 62   & 412   \\
             \midrule
             Total  & 4094  & 881        & 887  & 5862  \\
             \bottomrule
         \end{tabular}\label{tab:tcga}
     \end{subtable}
     \begin{subtable}[t]{0.45\textwidth}
         \subcaption{}
         \begin{tabular}{lrrrr}
             \toprule
             Cancer    & Train & Validation & Test & Total \\
             \midrule
             ALL       & 16    & 4          & 4    & 24    \\
             BRCA      & 29    & 6          & 7    & 42    \\
             COAD-READ & 32    & 7          & 7    & 46    \\
             DLBC      & 23    & 5          & 6    & 34    \\
             GBM       & 19    & 4          & 5    & 28    \\
             HNSC      & 20    & 4          & 5    & 29    \\
             KIRC      & 12    & 3          & 3    & 18    \\
             LAML      & 19    & 4          & 5    & 28    \\
             LUAD      & 39    & 8          & 9    & 56    \\
             MM        & 11    & 2          & 3    & 16    \\
             OV        & 25    & 6          & 6    & 37    \\
             PAAD      & 23    & 5          & 5    & 33    \\
             SARC      & 16    & 4          & 4    & 24    \\
             SCLC      & 30    & 6          & 7    & 43    \\
             SKCM      & 32    & 7          & 7    & 46    \\
             STAD      & 22    & 5          & 5    & 32    \\
             \midrule
             Total     & 368   & 80         & 88   & 536   \\
             \bottomrule
         \end{tabular}\label{tab:ccle}
     \end{subtable}
 \end{table}

 \begin{figure}
     \centering
     \includegraphics[width=\linewidth]{multi_omics_latency.pdf}
     \caption{Comparison of the latency, or time in milliseconds to get a prediction for one sample with various number of input omics. Error-bars represent latency variation across the various possible omics combinations.}\label{fig:latency}
 \end{figure}

 \begin{figure}[htbp]
     \centering
     \includegraphics{perf_gain_prot.pdf}
     \caption{Impact of proteins on the test accuracy when training CrossAttOmics with a fixed number of omics on the TCGA dataset.}\label{fig:perf_gain_prot}
 \end{figure}

 \begin{figure}[htbp]
     \centering
     \includegraphics[width=\textwidth]{limited_training_3_omics.pdf}
     \caption{Accuracy on the TCGA test set according to the size of the training set for various multi-omics deep learning models when trained on the best combination of 3 omics. When trained with the minimum samples possible, CrossAttOmics outperformed MLP IF\@. When the number of training samples was increased, MLP IF became the better architecture. The performance of CrossAttOmics depends on the number of modalities used, which impacts the number of cross-attentions. With the selected omics, miRNA, nc mRNA, and DNAm~(Figure~\ref{fig:tcga_perf_comb}), there is only two cross-attention to compute. The model cannot benefit from the cross-attention mechanism compared to situations with more modalities~(Figure~\ref{fig:lim_train_6_omics}).}\label{fig:lim_train_3_omics}
 \end{figure}



 \begin{figure}[htbp]
     \centering
     \includegraphics[width=\textwidth]{ccle_limited_training_3_omics.pdf}
     \caption{Accuracy on the CCLE test set according to the size of the training set for various multi-omics deep learning models when trained on the best combination of 3 omics.}\label{fig:ccle_limited_train}
 \end{figure}

 \begin{figure}[htbp]
     \centering
     \includegraphics[width=\textwidth]{LRP_crossattomics.pdf}
     \caption{Comparison of the LRP relevance score for the different modelled modality interactions across various cancer.}\label{fig:LRP_CrossAttOmics}
 \end{figure}

 \begin{figure}[htbp]
     \centering
     \includegraphics[width=\textwidth]{ccle_omics_com.pdf}
     \caption{Comparison of the test accuracy of different multi-omics deep learning integration models across different omics combination on the CCLE dataset. Each dot represents the mean accuracy obtained by a model on the test set after 5 different training. The error-bars represents the standard-error. For each combination a \cmark means that the omics is included in the combination and a \xmark means that the omics is excluded from the combination.}\label{fig:ccle_perf_comb}
 \end{figure}

 \begin{landscape}
     \section{Supplementary tables}
      \subsection{Models performances on various combinations of omics on TCGA dataset}
          \begin{longtable}{ccccccrrrrrr}
\caption{Classification metrics for AttOmics with different omics combination on TCGA dataset}\label{tab:perf_comb_AttOmics} \\
\toprule
mRNA & nc mRNA & miRNA & CNV & DNAm & Protein & AUROC & Accuracy & F1 & Precision & Recall & Specificity \\
\midrule
\endfirsthead
\caption[]{Classification metrics for AttOmics with different omics combination} \\
\toprule
mRNA & nc mRNA & miRNA & CNV & DNAm & Protein & AUROC & Accuracy & F1 & Precision & Recall & Specificity \\
\midrule
\endhead
\midrule
\multicolumn{12}{r}{Continued on next page} \\
\midrule
\endfoot
\bottomrule
\endlastfoot
 &  &  &  &  & \textbullet & 1.000 ± 0.000 & 0.985 ± 0.006 & 0.984 ± 0.006 & 0.984 ± 0.006 & 0.985 ± 0.006 & 0.999 ± 0.000 \\
 &  &  &  & \textbullet &  & 0.999 ± 0.000 & 0.973 ± 0.004 & 0.971 ± 0.004 & 0.970 ± 0.005 & 0.973 ± 0.004 & 0.998 ± 0.000 \\
 &  &  & \textbullet &  &  & 0.969 ± 0.003 & 0.727 ± 0.013 & 0.722 ± 0.015 & 0.729 ± 0.016 & 0.727 ± 0.013 & 0.984 ± 0.001 \\
 &  & \textbullet &  &  &  & 0.996 ± 0.001 & 0.915 ± 0.004 & 0.912 ± 0.004 & 0.912 ± 0.004 & 0.915 ± 0.004 & 0.995 ± 0.000 \\
 & \textbullet &  &  &  &  & 0.999 ± 0.000 & 0.965 ± 0.003 & 0.962 ± 0.003 & 0.961 ± 0.004 & 0.965 ± 0.003 & 0.998 ± 0.000 \\
\textbullet &  &  &  &  &  & 0.999 ± 0.000 & 0.969 ± 0.003 & 0.967 ± 0.004 & 0.967 ± 0.004 & 0.969 ± 0.003 & 0.998 ± 0.000 \\
\end{longtable}

          \begin{longtable}{ccccccrrrrrr}
\caption{Classification metrics for AttOmics EF with different omics combination on TCGA dataset} \label{tab:perf_comb_AttOmicsEarlyFusion} \\
\toprule
mRNA & nc mRNA & miRNA & CNV & DNAm & Protein & AUROC & Accuracy & F1 & Precision & Recall & Specificity \\
\midrule
\endfirsthead
\caption[]{Classification metrics for AttOmics EF with different omics combination} \\
\toprule
mRNA & nc mRNA & miRNA & CNV & DNAm & Protein & AUROC & Accuracy & F1 & Precision & Recall & Specificity \\
\midrule
\endhead
\midrule
\multicolumn{12}{r}{Continued on next page} \\
\midrule
\endfoot
\bottomrule
\endlastfoot
 &  &  &  & \textbullet & \textbullet & 0.999 ± 0.000 & 0.968 ± 0.003 & 0.966 ± 0.002 & 0.966 ± 0.002 & 0.968 ± 0.003 & 0.998 ± 0.000 \\
 &  &  & \textbullet &  & \textbullet & 0.999 ± 0.000 & 0.943 ± 0.005 & 0.940 ± 0.004 & 0.938 ± 0.003 & 0.943 ± 0.005 & 0.997 ± 0.000 \\
 &  &  & \textbullet & \textbullet &  & 0.998 ± 0.000 & 0.947 ± 0.005 & 0.945 ± 0.006 & 0.943 ± 0.007 & 0.947 ± 0.005 & 0.997 ± 0.000 \\
 &  &  & \textbullet & \textbullet & \textbullet & 0.999 ± 0.000 & 0.961 ± 0.003 & 0.959 ± 0.003 & 0.958 ± 0.003 & 0.961 ± 0.003 & 0.998 ± 0.000 \\
 &  & \textbullet &  &  & \textbullet & 0.999 ± 0.000 & 0.966 ± 0.005 & 0.966 ± 0.005 & 0.967 ± 0.004 & 0.966 ± 0.005 & 0.998 ± 0.000 \\
 &  & \textbullet &  & \textbullet &  & 0.998 ± 0.000 & 0.955 ± 0.005 & 0.953 ± 0.004 & 0.952 ± 0.003 & 0.955 ± 0.005 & 0.998 ± 0.000 \\
 &  & \textbullet &  & \textbullet & \textbullet & 0.999 ± 0.000 & 0.968 ± 0.003 & 0.966 ± 0.003 & 0.965 ± 0.004 & 0.968 ± 0.003 & 0.998 ± 0.000 \\
 &  & \textbullet & \textbullet &  &  & 0.990 ± 0.001 & 0.849 ± 0.009 & 0.840 ± 0.009 & 0.837 ± 0.011 & 0.849 ± 0.009 & 0.991 ± 0.000 \\
 &  & \textbullet & \textbullet &  & \textbullet & 0.999 ± 0.000 & 0.944 ± 0.006 & 0.941 ± 0.007 & 0.939 ± 0.007 & 0.944 ± 0.006 & 0.997 ± 0.000 \\
 &  & \textbullet & \textbullet & \textbullet &  & 0.998 ± 0.000 & 0.945 ± 0.004 & 0.943 ± 0.004 & 0.941 ± 0.005 & 0.945 ± 0.004 & 0.997 ± 0.000 \\
 &  & \textbullet & \textbullet & \textbullet & \textbullet & 0.999 ± 0.000 & 0.964 ± 0.003 & 0.962 ± 0.002 & 0.962 ± 0.002 & 0.964 ± 0.003 & 0.998 ± 0.000 \\
 & \textbullet &  &  &  & \textbullet & 1.000 ± 0.000 & 0.982 ± 0.003 & 0.982 ± 0.003 & 0.983 ± 0.003 & 0.982 ± 0.003 & 0.999 ± 0.000 \\
 & \textbullet &  &  & \textbullet &  & 0.999 ± 0.000 & 0.964 ± 0.005 & 0.962 ± 0.005 & 0.961 ± 0.006 & 0.964 ± 0.005 & 0.998 ± 0.000 \\
 & \textbullet &  &  & \textbullet & \textbullet & 0.999 ± 0.000 & 0.974 ± 0.004 & 0.973 ± 0.004 & 0.972 ± 0.004 & 0.974 ± 0.004 & 0.999 ± 0.000 \\
 & \textbullet &  & \textbullet &  &  & 0.997 ± 0.000 & 0.930 ± 0.004 & 0.927 ± 0.004 & 0.925 ± 0.003 & 0.930 ± 0.004 & 0.996 ± 0.000 \\
 & \textbullet &  & \textbullet &  & \textbullet & 0.999 ± 0.000 & 0.954 ± 0.004 & 0.952 ± 0.005 & 0.951 ± 0.006 & 0.954 ± 0.004 & 0.997 ± 0.000 \\
 & \textbullet &  & \textbullet & \textbullet &  & 0.999 ± 0.000 & 0.959 ± 0.005 & 0.958 ± 0.004 & 0.958 ± 0.004 & 0.959 ± 0.005 & 0.998 ± 0.000 \\
 & \textbullet &  & \textbullet & \textbullet & \textbullet & 0.999 ± 0.000 & 0.965 ± 0.005 & 0.964 ± 0.005 & 0.963 ± 0.004 & 0.965 ± 0.005 & 0.998 ± 0.000 \\
 & \textbullet & \textbullet &  &  &  & 0.999 ± 0.000 & 0.958 ± 0.007 & 0.958 ± 0.006 & 0.958 ± 0.006 & 0.958 ± 0.007 & 0.998 ± 0.000 \\
 & \textbullet & \textbullet &  &  & \textbullet & 1.000 ± 0.000 & 0.979 ± 0.001 & 0.979 ± 0.002 & 0.979 ± 0.002 & 0.979 ± 0.001 & 0.999 ± 0.000 \\
 & \textbullet & \textbullet &  & \textbullet &  & 0.999 ± 0.000 & 0.964 ± 0.007 & 0.962 ± 0.007 & 0.962 ± 0.007 & 0.964 ± 0.007 & 0.998 ± 0.000 \\
 & \textbullet & \textbullet &  & \textbullet & \textbullet & 0.999 ± 0.000 & 0.973 ± 0.003 & 0.972 ± 0.004 & 0.971 ± 0.004 & 0.973 ± 0.003 & 0.999 ± 0.000 \\
 & \textbullet & \textbullet & \textbullet &  &  & 0.998 ± 0.000 & 0.935 ± 0.003 & 0.932 ± 0.003 & 0.930 ± 0.003 & 0.935 ± 0.003 & 0.996 ± 0.000 \\
 & \textbullet & \textbullet & \textbullet &  & \textbullet & 0.999 ± 0.000 & 0.958 ± 0.006 & 0.955 ± 0.007 & 0.953 ± 0.008 & 0.958 ± 0.006 & 0.998 ± 0.000 \\
 & \textbullet & \textbullet & \textbullet & \textbullet &  & 0.999 ± 0.000 & 0.956 ± 0.006 & 0.955 ± 0.006 & 0.954 ± 0.006 & 0.956 ± 0.006 & 0.998 ± 0.000 \\
 & \textbullet & \textbullet & \textbullet & \textbullet & \textbullet & 0.999 ± 0.000 & 0.964 ± 0.002 & 0.962 ± 0.001 & 0.961 ± 0.001 & 0.964 ± 0.002 & 0.998 ± 0.000 \\
\textbullet &  &  &  &  & \textbullet & 1.000 ± 0.000 & 0.975 ± 0.002 & 0.974 ± 0.002 & 0.975 ± 0.002 & 0.975 ± 0.002 & 0.999 ± 0.000 \\
\textbullet &  &  &  & \textbullet &  & 0.999 ± 0.000 & 0.966 ± 0.005 & 0.964 ± 0.005 & 0.963 ± 0.004 & 0.966 ± 0.005 & 0.998 ± 0.000 \\
\textbullet &  &  &  & \textbullet & \textbullet & 0.999 ± 0.000 & 0.972 ± 0.003 & 0.971 ± 0.003 & 0.970 ± 0.003 & 0.972 ± 0.003 & 0.998 ± 0.000 \\
\textbullet &  &  & \textbullet &  &  & 0.998 ± 0.000 & 0.943 ± 0.005 & 0.940 ± 0.004 & 0.939 ± 0.004 & 0.943 ± 0.005 & 0.997 ± 0.000 \\
\textbullet &  &  & \textbullet &  & \textbullet & 0.999 ± 0.000 & 0.961 ± 0.006 & 0.958 ± 0.004 & 0.957 ± 0.003 & 0.961 ± 0.006 & 0.998 ± 0.000 \\
\textbullet &  &  & \textbullet & \textbullet &  & 0.999 ± 0.000 & 0.961 ± 0.005 & 0.960 ± 0.005 & 0.959 ± 0.006 & 0.961 ± 0.005 & 0.998 ± 0.000 \\
\textbullet &  &  & \textbullet & \textbullet & \textbullet & 0.999 ± 0.000 & 0.966 ± 0.003 & 0.965 ± 0.003 & 0.964 ± 0.003 & 0.966 ± 0.003 & 0.998 ± 0.000 \\
\textbullet &  & \textbullet &  &  &  & 0.999 ± 0.000 & 0.955 ± 0.004 & 0.952 ± 0.005 & 0.951 ± 0.006 & 0.955 ± 0.004 & 0.998 ± 0.000 \\
\textbullet &  & \textbullet &  &  & \textbullet & 1.000 ± 0.000 & 0.974 ± 0.003 & 0.972 ± 0.002 & 0.972 ± 0.003 & 0.974 ± 0.003 & 0.999 ± 0.000 \\
\textbullet &  & \textbullet &  & \textbullet &  & 0.999 ± 0.000 & 0.966 ± 0.006 & 0.965 ± 0.007 & 0.964 ± 0.008 & 0.966 ± 0.006 & 0.998 ± 0.000 \\
\textbullet &  & \textbullet &  & \textbullet & \textbullet & 0.999 ± 0.000 & 0.972 ± 0.006 & 0.970 ± 0.006 & 0.970 ± 0.005 & 0.972 ± 0.006 & 0.998 ± 0.000 \\
\textbullet &  & \textbullet & \textbullet &  &  & 0.998 ± 0.000 & 0.947 ± 0.003 & 0.945 ± 0.003 & 0.945 ± 0.003 & 0.947 ± 0.003 & 0.997 ± 0.000 \\
\textbullet &  & \textbullet & \textbullet &  & \textbullet & 0.999 ± 0.000 & 0.959 ± 0.003 & 0.957 ± 0.002 & 0.957 ± 0.002 & 0.959 ± 0.003 & 0.998 ± 0.000 \\
\textbullet &  & \textbullet & \textbullet & \textbullet &  & 0.999 ± 0.000 & 0.957 ± 0.003 & 0.956 ± 0.003 & 0.955 ± 0.003 & 0.957 ± 0.003 & 0.998 ± 0.000 \\
\textbullet &  & \textbullet & \textbullet & \textbullet & \textbullet & 0.999 ± 0.000 & 0.966 ± 0.004 & 0.965 ± 0.004 & 0.964 ± 0.004 & 0.966 ± 0.004 & 0.998 ± 0.000 \\
\textbullet & \textbullet &  &  &  &  & 0.999 ± 0.000 & 0.959 ± 0.006 & 0.957 ± 0.006 & 0.956 ± 0.007 & 0.959 ± 0.006 & 0.998 ± 0.000 \\
\textbullet & \textbullet &  &  &  & \textbullet & 1.000 ± 0.000 & 0.972 ± 0.001 & 0.972 ± 0.001 & 0.972 ± 0.002 & 0.972 ± 0.001 & 0.999 ± 0.000 \\
\textbullet & \textbullet &  &  & \textbullet &  & 0.999 ± 0.000 & 0.967 ± 0.003 & 0.966 ± 0.003 & 0.966 ± 0.003 & 0.967 ± 0.003 & 0.998 ± 0.000 \\
\textbullet & \textbullet &  &  & \textbullet & \textbullet & 0.999 ± 0.000 & 0.974 ± 0.004 & 0.973 ± 0.004 & 0.972 ± 0.004 & 0.974 ± 0.004 & 0.999 ± 0.000 \\
\textbullet & \textbullet &  & \textbullet &  &  & 0.999 ± 0.000 & 0.950 ± 0.005 & 0.948 ± 0.006 & 0.947 ± 0.007 & 0.950 ± 0.005 & 0.997 ± 0.000 \\
\textbullet & \textbullet &  & \textbullet &  & \textbullet & 0.999 ± 0.000 & 0.961 ± 0.001 & 0.958 ± 0.002 & 0.957 ± 0.002 & 0.961 ± 0.001 & 0.998 ± 0.000 \\
\textbullet & \textbullet &  & \textbullet & \textbullet &  & 0.999 ± 0.000 & 0.963 ± 0.004 & 0.962 ± 0.004 & 0.962 ± 0.004 & 0.963 ± 0.004 & 0.998 ± 0.000 \\
\textbullet & \textbullet &  & \textbullet & \textbullet & \textbullet & 0.999 ± 0.000 & 0.967 ± 0.002 & 0.965 ± 0.003 & 0.965 ± 0.004 & 0.967 ± 0.002 & 0.998 ± 0.000 \\
\textbullet & \textbullet & \textbullet &  &  &  & 0.999 ± 0.000 & 0.956 ± 0.002 & 0.954 ± 0.002 & 0.953 ± 0.002 & 0.956 ± 0.002 & 0.998 ± 0.000 \\
\textbullet & \textbullet & \textbullet &  &  & \textbullet & 0.999 ± 0.000 & 0.973 ± 0.004 & 0.972 ± 0.004 & 0.972 ± 0.004 & 0.973 ± 0.004 & 0.999 ± 0.000 \\
\textbullet & \textbullet & \textbullet &  & \textbullet &  & 0.999 ± 0.000 & 0.966 ± 0.002 & 0.965 ± 0.003 & 0.964 ± 0.003 & 0.966 ± 0.002 & 0.998 ± 0.000 \\
\textbullet & \textbullet & \textbullet &  & \textbullet & \textbullet & 0.999 ± 0.000 & 0.977 ± 0.003 & 0.975 ± 0.003 & 0.974 ± 0.003 & 0.977 ± 0.003 & 0.999 ± 0.000 \\
\textbullet & \textbullet & \textbullet & \textbullet &  &  & 0.999 ± 0.000 & 0.950 ± 0.004 & 0.947 ± 0.004 & 0.947 ± 0.004 & 0.950 ± 0.004 & 0.997 ± 0.000 \\
\textbullet & \textbullet & \textbullet & \textbullet &  & \textbullet & 0.999 ± 0.000 & 0.962 ± 0.003 & 0.960 ± 0.003 & 0.960 ± 0.004 & 0.962 ± 0.003 & 0.998 ± 0.000 \\
\textbullet & \textbullet & \textbullet & \textbullet & \textbullet &  & 0.999 ± 0.000 & 0.962 ± 0.005 & 0.960 ± 0.005 & 0.959 ± 0.005 & 0.962 ± 0.005 & 0.998 ± 0.000 \\
\textbullet & \textbullet & \textbullet & \textbullet & \textbullet & \textbullet & 0.999 ± 0.000 & 0.968 ± 0.004 & 0.967 ± 0.005 & 0.966 ± 0.005 & 0.968 ± 0.004 & 0.998 ± 0.000 \\
\end{longtable}

          \begin{longtable}{ccccccrrrrrr}
\caption{Classification metrics for AttOmics IF with different omics combination on TCGA dataset} \label{tab:perf_comb_AttOmicsIntermediateFusion} \\
\toprule
mRNA & nc mRNA & miRNA & CNV & DNAm & Protein & AUROC & Accuracy & F1 & Precision & Recall & Specificity \\
\midrule
\endfirsthead
\caption[]{Classification metrics for AttOmics IF with different omics combination} \\
\toprule
mRNA & nc mRNA & miRNA & CNV & DNAm & Protein & AUROC & Accuracy & F1 & Precision & Recall & Specificity \\
\midrule
\endhead
\midrule
\multicolumn{12}{r}{Continued on next page} \\
\midrule
\endfoot
\bottomrule
\endlastfoot
 &  &  &  & \textbullet & \textbullet & 1.000 ± 0.000 & 0.987 ± 0.002 & 0.986 ± 0.003 & 0.986 ± 0.003 & 0.987 ± 0.002 & 0.999 ± 0.000 \\
 &  &  & \textbullet &  & \textbullet & 0.999 ± 0.000 & 0.959 ± 0.002 & 0.958 ± 0.002 & 0.958 ± 0.002 & 0.959 ± 0.002 & 0.998 ± 0.000 \\
 &  &  & \textbullet & \textbullet &  & 0.999 ± 0.000 & 0.970 ± 0.001 & 0.968 ± 0.001 & 0.968 ± 0.001 & 0.970 ± 0.001 & 0.998 ± 0.000 \\
 &  &  & \textbullet & \textbullet & \textbullet & 1.000 ± 0.000 & 0.981 ± 0.002 & 0.979 ± 0.003 & 0.979 ± 0.004 & 0.981 ± 0.002 & 0.999 ± 0.000 \\
 &  & \textbullet &  &  & \textbullet & 1.000 ± 0.000 & 0.975 ± 0.005 & 0.975 ± 0.006 & 0.976 ± 0.006 & 0.975 ± 0.005 & 0.999 ± 0.000 \\
 &  & \textbullet &  & \textbullet &  & 1.000 ± 0.000 & 0.977 ± 0.003 & 0.975 ± 0.003 & 0.975 ± 0.003 & 0.977 ± 0.003 & 0.999 ± 0.000 \\
 &  & \textbullet &  & \textbullet & \textbullet & 1.000 ± 0.000 & 0.983 ± 0.002 & 0.982 ± 0.003 & 0.982 ± 0.003 & 0.983 ± 0.002 & 0.999 ± 0.000 \\
 &  & \textbullet & \textbullet &  &  & 0.997 ± 0.000 & 0.925 ± 0.004 & 0.922 ± 0.006 & 0.921 ± 0.008 & 0.925 ± 0.004 & 0.996 ± 0.000 \\
 &  & \textbullet & \textbullet &  & \textbullet & 0.999 ± 0.000 & 0.971 ± 0.004 & 0.969 ± 0.005 & 0.969 ± 0.007 & 0.971 ± 0.004 & 0.998 ± 0.000 \\
 &  & \textbullet & \textbullet & \textbullet &  & 0.999 ± 0.000 & 0.973 ± 0.005 & 0.972 ± 0.005 & 0.973 ± 0.006 & 0.973 ± 0.005 & 0.999 ± 0.000 \\
 &  & \textbullet & \textbullet & \textbullet & \textbullet & 1.000 ± 0.000 & 0.981 ± 0.002 & 0.980 ± 0.002 & 0.980 ± 0.002 & 0.981 ± 0.002 & 0.999 ± 0.000 \\
 & \textbullet &  &  &  & \textbullet & 1.000 ± 0.000 & 0.984 ± 0.003 & 0.983 ± 0.002 & 0.983 ± 0.002 & 0.984 ± 0.003 & 0.999 ± 0.000 \\
 & \textbullet &  &  & \textbullet &  & 1.000 ± 0.000 & 0.984 ± 0.003 & 0.982 ± 0.003 & 0.982 ± 0.004 & 0.984 ± 0.003 & 0.999 ± 0.000 \\
 & \textbullet &  &  & \textbullet & \textbullet & 1.000 ± 0.000 & 0.987 ± 0.001 & 0.986 ± 0.001 & 0.986 ± 0.001 & 0.987 ± 0.001 & 0.999 ± 0.000 \\
 & \textbullet &  & \textbullet &  &  & 0.999 ± 0.000 & 0.973 ± 0.001 & 0.972 ± 0.001 & 0.971 ± 0.001 & 0.973 ± 0.001 & 0.999 ± 0.000 \\
 & \textbullet &  & \textbullet &  & \textbullet & 1.000 ± 0.000 & 0.983 ± 0.002 & 0.981 ± 0.002 & 0.981 ± 0.002 & 0.983 ± 0.002 & 0.999 ± 0.000 \\
 & \textbullet &  & \textbullet & \textbullet &  & 1.000 ± 0.000 & 0.981 ± 0.005 & 0.980 ± 0.005 & 0.979 ± 0.004 & 0.981 ± 0.005 & 0.999 ± 0.000 \\
 & \textbullet &  & \textbullet & \textbullet & \textbullet & 1.000 ± 0.000 & 0.988 ± 0.003 & 0.987 ± 0.004 & 0.986 ± 0.004 & 0.988 ± 0.003 & 0.999 ± 0.000 \\
 & \textbullet & \textbullet &  &  &  & 1.000 ± 0.000 & 0.976 ± 0.006 & 0.974 ± 0.007 & 0.974 ± 0.007 & 0.976 ± 0.006 & 0.999 ± 0.000 \\
 & \textbullet & \textbullet &  &  & \textbullet & 1.000 ± 0.000 & 0.983 ± 0.005 & 0.982 ± 0.005 & 0.982 ± 0.005 & 0.983 ± 0.005 & 0.999 ± 0.000 \\
 & \textbullet & \textbullet &  & \textbullet &  & 1.000 ± 0.000 & 0.985 ± 0.001 & 0.983 ± 0.001 & 0.983 ± 0.001 & 0.985 ± 0.001 & 0.999 ± 0.000 \\
 & \textbullet & \textbullet &  & \textbullet & \textbullet & 1.000 ± 0.000 & 0.988 ± 0.001 & 0.987 ± 0.001 & 0.986 ± 0.001 & 0.988 ± 0.001 & 0.999 ± 0.000 \\
 & \textbullet & \textbullet & \textbullet &  &  & 1.000 ± 0.000 & 0.978 ± 0.002 & 0.975 ± 0.003 & 0.975 ± 0.004 & 0.978 ± 0.002 & 0.999 ± 0.000 \\
 & \textbullet & \textbullet & \textbullet &  & \textbullet & 1.000 ± 0.000 & 0.984 ± 0.003 & 0.984 ± 0.003 & 0.984 ± 0.003 & 0.984 ± 0.003 & 0.999 ± 0.000 \\
 & \textbullet & \textbullet & \textbullet & \textbullet &  & 1.000 ± 0.000 & 0.983 ± 0.003 & 0.982 ± 0.003 & 0.981 ± 0.002 & 0.983 ± 0.003 & 0.999 ± 0.000 \\
 & \textbullet & \textbullet & \textbullet & \textbullet & \textbullet & 1.000 ± 0.000 & 0.987 ± 0.004 & 0.985 ± 0.004 & 0.984 ± 0.004 & 0.987 ± 0.004 & 0.999 ± 0.000 \\
\textbullet &  &  &  &  & \textbullet & 1.000 ± 0.000 & 0.981 ± 0.002 & 0.980 ± 0.002 & 0.980 ± 0.002 & 0.981 ± 0.002 & 0.999 ± 0.000 \\
\textbullet &  &  &  & \textbullet &  & 1.000 ± 0.000 & 0.982 ± 0.001 & 0.981 ± 0.001 & 0.981 ± 0.001 & 0.982 ± 0.001 & 0.999 ± 0.000 \\
\textbullet &  &  &  & \textbullet & \textbullet & 1.000 ± 0.000 & 0.986 ± 0.002 & 0.984 ± 0.002 & 0.983 ± 0.003 & 0.986 ± 0.002 & 0.999 ± 0.000 \\
\textbullet &  &  & \textbullet &  &  & 0.999 ± 0.000 & 0.967 ± 0.004 & 0.964 ± 0.003 & 0.963 ± 0.002 & 0.967 ± 0.004 & 0.998 ± 0.000 \\
\textbullet &  &  & \textbullet &  & \textbullet & 1.000 ± 0.000 & 0.978 ± 0.004 & 0.977 ± 0.004 & 0.977 ± 0.004 & 0.978 ± 0.004 & 0.999 ± 0.000 \\
\textbullet &  &  & \textbullet & \textbullet &  & 1.000 ± 0.000 & 0.978 ± 0.001 & 0.976 ± 0.001 & 0.975 ± 0.002 & 0.978 ± 0.001 & 0.999 ± 0.000 \\
\textbullet &  &  & \textbullet & \textbullet & \textbullet & 1.000 ± 0.000 & 0.984 ± 0.002 & 0.983 ± 0.002 & 0.983 ± 0.002 & 0.984 ± 0.002 & 0.999 ± 0.000 \\
\textbullet &  & \textbullet &  &  &  & 0.999 ± 0.000 & 0.973 ± 0.003 & 0.972 ± 0.003 & 0.971 ± 0.003 & 0.973 ± 0.003 & 0.999 ± 0.000 \\
\textbullet &  & \textbullet &  &  & \textbullet & 1.000 ± 0.000 & 0.983 ± 0.001 & 0.982 ± 0.001 & 0.981 ± 0.001 & 0.983 ± 0.001 & 0.999 ± 0.000 \\
\textbullet &  & \textbullet &  & \textbullet &  & 1.000 ± 0.000 & 0.983 ± 0.002 & 0.981 ± 0.003 & 0.980 ± 0.003 & 0.983 ± 0.002 & 0.999 ± 0.000 \\
\textbullet &  & \textbullet &  & \textbullet & \textbullet & 1.000 ± 0.000 & 0.985 ± 0.003 & 0.984 ± 0.003 & 0.983 ± 0.003 & 0.985 ± 0.003 & 0.999 ± 0.000 \\
\textbullet &  & \textbullet & \textbullet &  &  & 0.999 ± 0.000 & 0.971 ± 0.003 & 0.969 ± 0.004 & 0.968 ± 0.004 & 0.971 ± 0.003 & 0.998 ± 0.000 \\
\textbullet &  & \textbullet & \textbullet &  & \textbullet & 1.000 ± 0.000 & 0.981 ± 0.002 & 0.980 ± 0.002 & 0.980 ± 0.002 & 0.981 ± 0.002 & 0.999 ± 0.000 \\
\textbullet &  & \textbullet & \textbullet & \textbullet &  & 1.000 ± 0.000 & 0.978 ± 0.001 & 0.977 ± 0.002 & 0.977 ± 0.002 & 0.978 ± 0.001 & 0.999 ± 0.000 \\
\textbullet &  & \textbullet & \textbullet & \textbullet & \textbullet & 1.000 ± 0.000 & 0.987 ± 0.002 & 0.986 ± 0.001 & 0.985 ± 0.001 & 0.987 ± 0.002 & 0.999 ± 0.000 \\
\textbullet & \textbullet &  &  &  &  & 1.000 ± 0.000 & 0.976 ± 0.003 & 0.974 ± 0.004 & 0.974 ± 0.004 & 0.976 ± 0.003 & 0.999 ± 0.000 \\
\textbullet & \textbullet &  &  &  & \textbullet & 1.000 ± 0.000 & 0.984 ± 0.002 & 0.983 ± 0.003 & 0.982 ± 0.003 & 0.984 ± 0.002 & 0.999 ± 0.000 \\
\textbullet & \textbullet &  &  & \textbullet &  & 1.000 ± 0.000 & 0.984 ± 0.004 & 0.982 ± 0.004 & 0.981 ± 0.004 & 0.984 ± 0.004 & 0.999 ± 0.000 \\
\textbullet & \textbullet &  &  & \textbullet & \textbullet & 1.000 ± 0.000 & 0.987 ± 0.002 & 0.986 ± 0.002 & 0.985 ± 0.002 & 0.987 ± 0.002 & 0.999 ± 0.000 \\
\textbullet & \textbullet &  & \textbullet &  &  & 0.999 ± 0.000 & 0.979 ± 0.003 & 0.976 ± 0.003 & 0.975 ± 0.004 & 0.979 ± 0.003 & 0.999 ± 0.000 \\
\textbullet & \textbullet &  & \textbullet &  & \textbullet & 1.000 ± 0.000 & 0.979 ± 0.002 & 0.978 ± 0.002 & 0.978 ± 0.001 & 0.979 ± 0.002 & 0.999 ± 0.000 \\
\textbullet & \textbullet &  & \textbullet & \textbullet &  & 1.000 ± 0.000 & 0.981 ± 0.003 & 0.979 ± 0.003 & 0.979 ± 0.003 & 0.981 ± 0.003 & 0.999 ± 0.000 \\
\textbullet & \textbullet &  & \textbullet & \textbullet & \textbullet & 1.000 ± 0.000 & 0.983 ± 0.002 & 0.982 ± 0.001 & 0.982 ± 0.002 & 0.983 ± 0.002 & 0.999 ± 0.000 \\
\textbullet & \textbullet & \textbullet &  &  &  & 1.000 ± 0.000 & 0.980 ± 0.003 & 0.978 ± 0.003 & 0.977 ± 0.003 & 0.980 ± 0.003 & 0.999 ± 0.000 \\
\textbullet & \textbullet & \textbullet &  &  & \textbullet & 1.000 ± 0.000 & 0.985 ± 0.001 & 0.984 ± 0.001 & 0.983 ± 0.002 & 0.985 ± 0.001 & 0.999 ± 0.000 \\
\textbullet & \textbullet & \textbullet &  & \textbullet &  & 1.000 ± 0.000 & 0.982 ± 0.002 & 0.981 ± 0.002 & 0.980 ± 0.001 & 0.982 ± 0.002 & 0.999 ± 0.000 \\
\textbullet & \textbullet & \textbullet &  & \textbullet & \textbullet & 1.000 ± 0.000 & 0.988 ± 0.001 & 0.986 ± 0.001 & 0.985 ± 0.001 & 0.988 ± 0.001 & 0.999 ± 0.000 \\
\textbullet & \textbullet & \textbullet & \textbullet &  &  & 1.000 ± 0.000 & 0.979 ± 0.002 & 0.978 ± 0.002 & 0.978 ± 0.002 & 0.979 ± 0.002 & 0.999 ± 0.000 \\
\textbullet & \textbullet & \textbullet & \textbullet &  & \textbullet & 1.000 ± 0.000 & 0.984 ± 0.003 & 0.984 ± 0.003 & 0.984 ± 0.003 & 0.984 ± 0.003 & 0.999 ± 0.000 \\
\textbullet & \textbullet & \textbullet & \textbullet & \textbullet &  & 1.000 ± 0.000 & 0.982 ± 0.001 & 0.981 ± 0.002 & 0.980 ± 0.003 & 0.982 ± 0.001 & 0.999 ± 0.000 \\
\textbullet & \textbullet & \textbullet & \textbullet & \textbullet & \textbullet & 1.000 ± 0.000 & 0.985 ± 0.002 & 0.983 ± 0.002 & 0.983 ± 0.002 & 0.985 ± 0.002 & 0.999 ± 0.000 \\
\end{longtable}

          \begin{longtable}{ccccccrrrrrr}
\caption{Classification metrics for CrossAttOmics with different omics combination on TCGA dataset} \label{tab:perf_comb_CrossAttOmics} \\
\toprule
mRNA & nc mRNA & miRNA & CNV & DNAm & Protein & AUROC & Accuracy & F1 & Precision & Recall & Specificity \\
\midrule
\endfirsthead
\caption[]{Classification metrics for CrossAttOmics with different omics combination} \\
\toprule
mRNA & nc mRNA & miRNA & CNV & DNAm & Protein & AUROC & Accuracy & F1 & Precision & Recall & Specificity \\
\midrule
\endhead
\midrule
\multicolumn{12}{r}{Continued on next page} \\
\midrule
\endfoot
\bottomrule
\endlastfoot
 &  & \textbullet &  & \textbullet &  & 1.000 ± 0.000 & 0.976 ± 0.003 & 0.975 ± 0.003 & 0.975 ± 0.003 & 0.976 ± 0.003 & 0.999 ± 0.000 \\
 &  & \textbullet &  & \textbullet & \textbullet & 1.000 ± 0.000 & 0.984 ± 0.001 & 0.983 ± 0.001 & 0.983 ± 0.001 & 0.984 ± 0.001 & 0.999 ± 0.000 \\
 &  & \textbullet & \textbullet &  &  & 0.997 ± 0.000 & 0.928 ± 0.006 & 0.926 ± 0.006 & 0.926 ± 0.006 & 0.928 ± 0.006 & 0.996 ± 0.000 \\
 &  & \textbullet & \textbullet &  & \textbullet & 1.000 ± 0.000 & 0.969 ± 0.004 & 0.968 ± 0.004 & 0.968 ± 0.004 & 0.969 ± 0.004 & 0.998 ± 0.000 \\
 &  & \textbullet & \textbullet & \textbullet &  & 0.999 ± 0.000 & 0.971 ± 0.005 & 0.969 ± 0.005 & 0.970 ± 0.007 & 0.971 ± 0.005 & 0.998 ± 0.000 \\
 &  & \textbullet & \textbullet & \textbullet & \textbullet & 1.000 ± 0.000 & 0.979 ± 0.002 & 0.979 ± 0.002 & 0.979 ± 0.002 & 0.979 ± 0.002 & 0.999 ± 0.000 \\
 & \textbullet &  &  & \textbullet &  & 1.000 ± 0.000 & 0.986 ± 0.003 & 0.984 ± 0.003 & 0.983 ± 0.003 & 0.986 ± 0.003 & 0.999 ± 0.000 \\
 & \textbullet &  &  & \textbullet & \textbullet & 1.000 ± 0.000 & 0.987 ± 0.001 & 0.985 ± 0.001 & 0.985 ± 0.001 & 0.987 ± 0.001 & 0.999 ± 0.000 \\
 & \textbullet &  & \textbullet &  &  & 0.999 ± 0.000 & 0.971 ± 0.004 & 0.970 ± 0.004 & 0.969 ± 0.004 & 0.971 ± 0.004 & 0.998 ± 0.000 \\
 & \textbullet &  & \textbullet &  & \textbullet & 1.000 ± 0.000 & 0.985 ± 0.001 & 0.984 ± 0.001 & 0.984 ± 0.001 & 0.985 ± 0.001 & 0.999 ± 0.000 \\
 & \textbullet &  & \textbullet & \textbullet &  & 1.000 ± 0.000 & 0.984 ± 0.003 & 0.983 ± 0.003 & 0.982 ± 0.003 & 0.984 ± 0.003 & 0.999 ± 0.000 \\
 & \textbullet &  & \textbullet & \textbullet & \textbullet & 1.000 ± 0.000 & 0.986 ± 0.001 & 0.985 ± 0.001 & 0.984 ± 0.001 & 0.986 ± 0.001 & 0.999 ± 0.000 \\
 & \textbullet & \textbullet &  & \textbullet &  & 1.000 ± 0.000 & 0.987 ± 0.003 & 0.985 ± 0.003 & 0.985 ± 0.002 & 0.987 ± 0.003 & 0.999 ± 0.000 \\
 & \textbullet & \textbullet &  & \textbullet & \textbullet & 1.000 ± 0.000 & 0.987 ± 0.004 & 0.986 ± 0.004 & 0.985 ± 0.004 & 0.987 ± 0.004 & 0.999 ± 0.000 \\
 & \textbullet & \textbullet & \textbullet &  &  & 1.000 ± 0.000 & 0.977 ± 0.005 & 0.974 ± 0.005 & 0.974 ± 0.005 & 0.977 ± 0.005 & 0.999 ± 0.000 \\
 & \textbullet & \textbullet & \textbullet &  & \textbullet & 1.000 ± 0.000 & 0.984 ± 0.002 & 0.984 ± 0.002 & 0.984 ± 0.001 & 0.984 ± 0.002 & 0.999 ± 0.000 \\
 & \textbullet & \textbullet & \textbullet & \textbullet &  & 1.000 ± 0.000 & 0.984 ± 0.003 & 0.982 ± 0.003 & 0.982 ± 0.003 & 0.984 ± 0.003 & 0.999 ± 0.000 \\
 & \textbullet & \textbullet & \textbullet & \textbullet & \textbullet & 1.000 ± 0.000 & 0.985 ± 0.004 & 0.983 ± 0.004 & 0.983 ± 0.005 & 0.985 ± 0.004 & 0.999 ± 0.000 \\
\textbullet &  &  &  &  & \textbullet & 1.000 ± 0.000 & 0.981 ± 0.004 & 0.980 ± 0.004 & 0.979 ± 0.004 & 0.981 ± 0.004 & 0.999 ± 0.000 \\
\textbullet &  &  &  & \textbullet &  & 1.000 ± 0.000 & 0.982 ± 0.002 & 0.980 ± 0.002 & 0.979 ± 0.002 & 0.982 ± 0.002 & 0.999 ± 0.000 \\
\textbullet &  &  &  & \textbullet & \textbullet & 1.000 ± 0.000 & 0.986 ± 0.001 & 0.984 ± 0.001 & 0.983 ± 0.002 & 0.986 ± 0.001 & 0.999 ± 0.000 \\
\textbullet &  &  & \textbullet &  &  & 0.999 ± 0.000 & 0.965 ± 0.002 & 0.963 ± 0.002 & 0.963 ± 0.002 & 0.965 ± 0.002 & 0.998 ± 0.000 \\
\textbullet &  &  & \textbullet &  & \textbullet & 1.000 ± 0.000 & 0.977 ± 0.005 & 0.975 ± 0.004 & 0.975 ± 0.003 & 0.977 ± 0.005 & 0.999 ± 0.000 \\
\textbullet &  &  & \textbullet & \textbullet &  & 1.000 ± 0.000 & 0.978 ± 0.001 & 0.976 ± 0.001 & 0.976 ± 0.001 & 0.978 ± 0.001 & 0.999 ± 0.000 \\
\textbullet &  &  & \textbullet & \textbullet & \textbullet & 1.000 ± 0.000 & 0.981 ± 0.002 & 0.980 ± 0.002 & 0.979 ± 0.001 & 0.981 ± 0.002 & 0.999 ± 0.000 \\
\textbullet &  & \textbullet &  &  &  & 0.999 ± 0.000 & 0.972 ± 0.003 & 0.971 ± 0.003 & 0.971 ± 0.004 & 0.972 ± 0.003 & 0.999 ± 0.000 \\
\textbullet &  & \textbullet &  &  & \textbullet & 1.000 ± 0.000 & 0.984 ± 0.002 & 0.982 ± 0.002 & 0.982 ± 0.001 & 0.984 ± 0.002 & 0.999 ± 0.000 \\
\textbullet &  & \textbullet &  & \textbullet &  & 1.000 ± 0.000 & 0.980 ± 0.001 & 0.978 ± 0.001 & 0.978 ± 0.001 & 0.980 ± 0.001 & 0.999 ± 0.000 \\
\textbullet &  & \textbullet &  & \textbullet & \textbullet & 1.000 ± 0.000 & 0.984 ± 0.002 & 0.983 ± 0.002 & 0.983 ± 0.002 & 0.984 ± 0.002 & 0.999 ± 0.000 \\
\textbullet &  & \textbullet & \textbullet &  &  & 1.000 ± 0.000 & 0.972 ± 0.004 & 0.971 ± 0.004 & 0.970 ± 0.004 & 0.972 ± 0.004 & 0.999 ± 0.000 \\
\textbullet &  & \textbullet & \textbullet &  & \textbullet & 1.000 ± 0.000 & 0.979 ± 0.005 & 0.978 ± 0.005 & 0.978 ± 0.005 & 0.979 ± 0.005 & 0.999 ± 0.000 \\
\textbullet &  & \textbullet & \textbullet & \textbullet &  & 1.000 ± 0.000 & 0.978 ± 0.003 & 0.977 ± 0.002 & 0.977 ± 0.002 & 0.978 ± 0.003 & 0.999 ± 0.000 \\
\textbullet &  & \textbullet & \textbullet & \textbullet & \textbullet & 1.000 ± 0.000 & 0.983 ± 0.002 & 0.982 ± 0.002 & 0.982 ± 0.002 & 0.983 ± 0.002 & 0.999 ± 0.000 \\
\textbullet & \textbullet &  &  &  &  & 1.000 ± 0.000 & 0.976 ± 0.002 & 0.974 ± 0.002 & 0.973 ± 0.002 & 0.976 ± 0.002 & 0.999 ± 0.000 \\
\textbullet & \textbullet &  &  &  & \textbullet & 1.000 ± 0.000 & 0.982 ± 0.003 & 0.980 ± 0.003 & 0.980 ± 0.003 & 0.982 ± 0.003 & 0.999 ± 0.000 \\
\textbullet & \textbullet &  &  & \textbullet &  & 1.000 ± 0.000 & 0.983 ± 0.002 & 0.982 ± 0.002 & 0.981 ± 0.002 & 0.983 ± 0.002 & 0.999 ± 0.000 \\
\textbullet & \textbullet &  &  & \textbullet & \textbullet & 1.000 ± 0.000 & 0.985 ± 0.002 & 0.984 ± 0.001 & 0.983 ± 0.001 & 0.985 ± 0.002 & 0.999 ± 0.000 \\
\textbullet & \textbullet &  & \textbullet &  &  & 1.000 ± 0.000 & 0.975 ± 0.003 & 0.974 ± 0.003 & 0.974 ± 0.004 & 0.975 ± 0.003 & 0.999 ± 0.000 \\
\textbullet & \textbullet &  & \textbullet &  & \textbullet & 1.000 ± 0.000 & 0.979 ± 0.003 & 0.977 ± 0.003 & 0.977 ± 0.004 & 0.979 ± 0.003 & 0.999 ± 0.000 \\
\textbullet & \textbullet &  & \textbullet & \textbullet &  & 1.000 ± 0.000 & 0.980 ± 0.002 & 0.979 ± 0.002 & 0.978 ± 0.003 & 0.980 ± 0.002 & 0.999 ± 0.000 \\
\textbullet & \textbullet &  & \textbullet & \textbullet & \textbullet & 1.000 ± 0.000 & 0.983 ± 0.004 & 0.982 ± 0.004 & 0.981 ± 0.004 & 0.983 ± 0.004 & 0.999 ± 0.000 \\
\textbullet & \textbullet & \textbullet &  &  &  & 1.000 ± 0.000 & 0.976 ± 0.003 & 0.974 ± 0.003 & 0.974 ± 0.003 & 0.976 ± 0.003 & 0.999 ± 0.000 \\
\textbullet & \textbullet & \textbullet &  &  & \textbullet & 1.000 ± 0.000 & 0.981 ± 0.003 & 0.980 ± 0.003 & 0.980 ± 0.003 & 0.981 ± 0.003 & 0.999 ± 0.000 \\
\textbullet & \textbullet & \textbullet &  & \textbullet &  & 1.000 ± 0.000 & 0.982 ± 0.002 & 0.980 ± 0.003 & 0.979 ± 0.003 & 0.982 ± 0.002 & 0.999 ± 0.000 \\
\textbullet & \textbullet & \textbullet &  & \textbullet & \textbullet & 1.000 ± 0.000 & 0.983 ± 0.004 & 0.981 ± 0.004 & 0.981 ± 0.004 & 0.983 ± 0.004 & 0.999 ± 0.000 \\
\textbullet & \textbullet & \textbullet & \textbullet &  &  & 1.000 ± 0.000 & 0.976 ± 0.003 & 0.975 ± 0.003 & 0.975 ± 0.003 & 0.976 ± 0.003 & 0.999 ± 0.000 \\
\textbullet & \textbullet & \textbullet & \textbullet &  & \textbullet & 1.000 ± 0.000 & 0.978 ± 0.002 & 0.977 ± 0.001 & 0.977 ± 0.001 & 0.978 ± 0.002 & 0.999 ± 0.000 \\
\textbullet & \textbullet & \textbullet & \textbullet & \textbullet &  & 1.000 ± 0.000 & 0.981 ± 0.002 & 0.980 ± 0.002 & 0.979 ± 0.002 & 0.981 ± 0.002 & 0.999 ± 0.000 \\
\textbullet & \textbullet & \textbullet & \textbullet & \textbullet & \textbullet & 1.000 ± 0.000 & 0.981 ± 0.001 & 0.979 ± 0.002 & 0.979 ± 0.002 & 0.981 ± 0.001 & 0.999 ± 0.000 \\
\end{longtable}

          \begin{longtable}{ccccccrrrrrr}
\caption{Classification metrics for GNN with different omics combination on TCGA dataset} \label{tab:perf_comb_GraphClassifier} \\
\toprule
mRNA & nc mRNA & miRNA & CNV & DNAm & Protein & AUROC & Accuracy & F1 & Precision & Recall & Specificity \\
\midrule
\endfirsthead
\caption[]{Classification metrics for GNN with different omics combination} \\
\toprule
mRNA & nc mRNA & miRNA & CNV & DNAm & Protein & AUROC & Accuracy & F1 & Precision & Recall & Specificity \\
\midrule
\endhead
\midrule
\multicolumn{12}{r}{Continued on next page} \\
\midrule
\endfoot
\bottomrule
\endlastfoot
 &  &  &  & \textbullet &  & 0.998 ± 0.000 & 0.954 ± 0.003 & 0.954 ± 0.002 & 0.956 ± 0.002 & 0.954 ± 0.003 & 0.998 ± 0.000 \\
 &  &  & \textbullet &  &  & 0.946 ± 0.007 & 0.623 ± 0.017 & 0.615 ± 0.027 & 0.655 ± 0.022 & 0.623 ± 0.017 & 0.977 ± 0.001 \\
\textbullet &  &  &  &  &  & 0.998 ± 0.000 & 0.956 ± 0.005 & 0.955 ± 0.004 & 0.956 ± 0.005 & 0.956 ± 0.005 & 0.998 ± 0.000 \\
\end{longtable}

          \begin{longtable}{ccccccrrrrrr}
\caption{Classification metrics for GNN EF with different omics combination on TCGA dataset} \label{tab:perf_comb_GraphClassifierEarlyFusion} \\
\toprule
mRNA & nc mRNA & miRNA & CNV & DNAm & Protein & AUROC & Accuracy & F1 & Precision & Recall & Specificity \\
\midrule
\endfirsthead
\caption[]{Classification metrics for GNN EF with different omics combination} \\
\toprule
mRNA & nc mRNA & miRNA & CNV & DNAm & Protein & AUROC & Accuracy & F1 & Precision & Recall & Specificity \\
\midrule
\endhead
\midrule
\multicolumn{12}{r}{Continued on next page} \\
\midrule
\endfoot
\bottomrule
\endlastfoot
 &  &  & \textbullet & \textbullet &  & 0.998 ± 0.000 & 0.952 ± 0.006 & 0.952 ± 0.005 & 0.953 ± 0.005 & 0.952 ± 0.006 & 0.997 ± 0.000 \\
\textbullet &  &  &  & \textbullet &  & 0.999 ± 0.000 & 0.959 ± 0.007 & 0.958 ± 0.006 & 0.958 ± 0.006 & 0.959 ± 0.007 & 0.998 ± 0.000 \\
\textbullet &  &  & \textbullet &  &  & 0.999 ± 0.000 & 0.963 ± 0.005 & 0.960 ± 0.005 & 0.959 ± 0.005 & 0.963 ± 0.005 & 0.998 ± 0.000 \\
\textbullet &  &  & \textbullet & \textbullet &  & 0.999 ± 0.001 & 0.961 ± 0.001 & 0.960 ± 0.002 & 0.961 ± 0.002 & 0.961 ± 0.001 & 0.998 ± 0.000 \\
\end{longtable}

          \begin{longtable}{ccccccrrrrrr}
\caption{Classification metrics for MLP with different omics combination on TCGA dataset}\label{tab:perf_comb_MLP} \\
\toprule
mRNA & nc mRNA & miRNA & CNV & DNAm & Protein & AUROC & Accuracy & F1 & Precision & Recall & Specificity \\
\midrule
\endfirsthead
\caption[]{Classification metrics for MLP with different omics combination} \\
\toprule
mRNA & nc mRNA & miRNA & CNV & DNAm & Protein & AUROC & Accuracy & F1 & Precision & Recall & Specificity \\
\midrule
\endhead
\midrule
\multicolumn{12}{r}{Continued on next page} \\
\midrule
\endfoot
\bottomrule
\endlastfoot
 &  &  &  &  & \textbullet & 1.000 ± 0.000 & 0.990 ± 0.002 & 0.985 ± 0.002 & 0.981 ± 0.002 & 0.990 ± 0.002 & 0.999 ± 0.000 \\
 &  &  &  & \textbullet &  & 0.999 ± 0.000 & 0.965 ± 0.001 & 0.965 ± 0.001 & 0.967 ± 0.001 & 0.965 ± 0.001 & 0.998 ± 0.000 \\
 &  &  & \textbullet &  &  & 0.969 ± 0.001 & 0.733 ± 0.009 & 0.734 ± 0.010 & 0.749 ± 0.011 & 0.733 ± 0.009 & 0.985 ± 0.001 \\
 &  & \textbullet &  &  &  & 0.997 ± 0.000 & 0.926 ± 0.004 & 0.926 ± 0.004 & 0.928 ± 0.004 & 0.926 ± 0.004 & 0.996 ± 0.000 \\
 & \textbullet &  &  &  &  & 0.998 ± 0.001 & 0.953 ± 0.002 & 0.951 ± 0.002 & 0.951 ± 0.002 & 0.953 ± 0.002 & 0.998 ± 0.000 \\
\textbullet &  &  &  &  &  & 0.998 ± 0.000 & 0.967 ± 0.002 & 0.967 ± 0.002 & 0.967 ± 0.002 & 0.967 ± 0.002 & 0.998 ± 0.000 \\
\end{longtable}

          \begin{longtable}{ccccccrrrrrr}
\caption{Classification metrics for MLP EF with different omics combination on TCGA dataset}\label{tab:perf_comb_MLPEarlyFusion} \\
\toprule
mRNA & nc mRNA & miRNA & CNV & DNAm & Protein & AUROC & Accuracy & F1 & Precision & Recall & Specificity \\
\midrule
\endfirsthead
\caption[]{Classification metrics for MLP EF with different omics combination} \\
\toprule
mRNA & nc mRNA & miRNA & CNV & DNAm & Protein & AUROC & Accuracy & F1 & Precision & Recall & Specificity \\
\midrule
\endhead
\midrule
\multicolumn{12}{r}{Continued on next page} \\
\midrule
\endfoot
\bottomrule
\endlastfoot
 &  &  &  & \textbullet & \textbullet & 1.000 ± 0.000 & 0.983 ± 0.002 & 0.981 ± 0.001 & 0.982 ± 0.001 & 0.983 ± 0.002 & 0.999 ± 0.000 \\
 &  &  & \textbullet &  & \textbullet & 0.999 ± 0.000 & 0.965 ± 0.004 & 0.962 ± 0.003 & 0.961 ± 0.003 & 0.965 ± 0.004 & 0.998 ± 0.000 \\
 &  &  & \textbullet & \textbullet &  & 0.999 ± 0.000 & 0.959 ± 0.002 & 0.958 ± 0.002 & 0.957 ± 0.001 & 0.959 ± 0.002 & 0.998 ± 0.000 \\
 &  &  & \textbullet & \textbullet & \textbullet & 0.999 ± 0.000 & 0.971 ± 0.004 & 0.969 ± 0.005 & 0.968 ± 0.005 & 0.971 ± 0.004 & 0.998 ± 0.000 \\
 &  & \textbullet &  &  & \textbullet & 1.000 ± 0.000 & 0.990 ± 0.002 & 0.990 ± 0.002 & 0.990 ± 0.003 & 0.990 ± 0.002 & 0.999 ± 0.000 \\
 &  & \textbullet &  & \textbullet &  & 0.999 ± 0.000 & 0.975 ± 0.002 & 0.974 ± 0.002 & 0.974 ± 0.002 & 0.975 ± 0.002 & 0.999 ± 0.000 \\
 &  & \textbullet &  & \textbullet & \textbullet & 1.000 ± 0.000 & 0.983 ± 0.003 & 0.982 ± 0.003 & 0.982 ± 0.003 & 0.983 ± 0.003 & 0.999 ± 0.000 \\
 &  & \textbullet & \textbullet &  &  & 0.993 ± 0.001 & 0.877 ± 0.013 & 0.877 ± 0.013 & 0.881 ± 0.011 & 0.877 ± 0.013 & 0.993 ± 0.001 \\
 &  & \textbullet & \textbullet &  & \textbullet & 1.000 ± 0.000 & 0.968 ± 0.002 & 0.967 ± 0.003 & 0.966 ± 0.004 & 0.968 ± 0.002 & 0.998 ± 0.000 \\
 &  & \textbullet & \textbullet & \textbullet &  & 0.998 ± 0.001 & 0.963 ± 0.003 & 0.962 ± 0.004 & 0.962 ± 0.005 & 0.963 ± 0.003 & 0.998 ± 0.000 \\
 &  & \textbullet & \textbullet & \textbullet & \textbullet & 0.999 ± 0.000 & 0.975 ± 0.002 & 0.974 ± 0.003 & 0.973 ± 0.003 & 0.975 ± 0.002 & 0.999 ± 0.000 \\
 & \textbullet &  &  &  & \textbullet & 0.998 ± 0.001 & 0.979 ± 0.004 & 0.979 ± 0.003 & 0.980 ± 0.003 & 0.979 ± 0.004 & 0.999 ± 0.000 \\
 & \textbullet &  &  & \textbullet &  & 0.999 ± 0.001 & 0.973 ± 0.002 & 0.971 ± 0.002 & 0.972 ± 0.001 & 0.973 ± 0.002 & 0.999 ± 0.000 \\
 & \textbullet &  &  & \textbullet & \textbullet & 0.999 ± 0.000 & 0.977 ± 0.003 & 0.975 ± 0.003 & 0.975 ± 0.004 & 0.977 ± 0.003 & 0.999 ± 0.000 \\
 & \textbullet &  & \textbullet &  &  & 0.996 ± 0.001 & 0.942 ± 0.006 & 0.938 ± 0.006 & 0.936 ± 0.007 & 0.942 ± 0.006 & 0.997 ± 0.000 \\
 & \textbullet &  & \textbullet &  & \textbullet & 0.998 ± 0.000 & 0.967 ± 0.003 & 0.965 ± 0.004 & 0.964 ± 0.005 & 0.967 ± 0.003 & 0.998 ± 0.000 \\
 & \textbullet &  & \textbullet & \textbullet &  & 0.998 ± 0.000 & 0.968 ± 0.004 & 0.965 ± 0.004 & 0.964 ± 0.005 & 0.968 ± 0.004 & 0.998 ± 0.000 \\
 & \textbullet &  & \textbullet & \textbullet & \textbullet & 0.999 ± 0.000 & 0.975 ± 0.004 & 0.972 ± 0.004 & 0.971 ± 0.003 & 0.975 ± 0.004 & 0.999 ± 0.000 \\
 & \textbullet & \textbullet &  &  &  & 0.996 ± 0.001 & 0.961 ± 0.001 & 0.960 ± 0.002 & 0.961 ± 0.003 & 0.961 ± 0.001 & 0.998 ± 0.000 \\
 & \textbullet & \textbullet &  &  & \textbullet & 0.998 ± 0.001 & 0.979 ± 0.002 & 0.979 ± 0.002 & 0.980 ± 0.002 & 0.979 ± 0.002 & 0.999 ± 0.000 \\
 & \textbullet & \textbullet &  & \textbullet &  & 0.999 ± 0.001 & 0.972 ± 0.001 & 0.971 ± 0.001 & 0.972 ± 0.001 & 0.972 ± 0.001 & 0.998 ± 0.000 \\
 & \textbullet & \textbullet &  & \textbullet & \textbullet & 0.999 ± 0.000 & 0.979 ± 0.002 & 0.977 ± 0.002 & 0.977 ± 0.002 & 0.979 ± 0.002 & 0.999 ± 0.000 \\
 & \textbullet & \textbullet & \textbullet &  &  & 0.996 ± 0.001 & 0.943 ± 0.002 & 0.942 ± 0.003 & 0.941 ± 0.003 & 0.943 ± 0.002 & 0.997 ± 0.000 \\
 & \textbullet & \textbullet & \textbullet &  & \textbullet & 0.998 ± 0.000 & 0.968 ± 0.004 & 0.966 ± 0.005 & 0.967 ± 0.006 & 0.968 ± 0.004 & 0.998 ± 0.000 \\
 & \textbullet & \textbullet & \textbullet & \textbullet &  & 0.998 ± 0.001 & 0.968 ± 0.003 & 0.965 ± 0.002 & 0.965 ± 0.002 & 0.968 ± 0.003 & 0.998 ± 0.000 \\
 & \textbullet & \textbullet & \textbullet & \textbullet & \textbullet & 0.999 ± 0.000 & 0.974 ± 0.002 & 0.972 ± 0.002 & 0.971 ± 0.002 & 0.974 ± 0.002 & 0.999 ± 0.000 \\
\textbullet &  &  &  &  & \textbullet & 1.000 ± 0.000 & 0.984 ± 0.003 & 0.983 ± 0.003 & 0.982 ± 0.003 & 0.984 ± 0.003 & 0.999 ± 0.000 \\
\textbullet &  &  &  & \textbullet &  & 0.999 ± 0.000 & 0.980 ± 0.002 & 0.978 ± 0.002 & 0.977 ± 0.002 & 0.980 ± 0.002 & 0.999 ± 0.000 \\
\textbullet &  &  &  & \textbullet & \textbullet & 1.000 ± 0.000 & 0.982 ± 0.002 & 0.981 ± 0.002 & 0.981 ± 0.002 & 0.982 ± 0.002 & 0.999 ± 0.000 \\
\textbullet &  &  & \textbullet &  &  & 0.998 ± 0.000 & 0.955 ± 0.001 & 0.952 ± 0.002 & 0.951 ± 0.002 & 0.955 ± 0.001 & 0.998 ± 0.000 \\
\textbullet &  &  & \textbullet &  & \textbullet & 0.999 ± 0.000 & 0.973 ± 0.001 & 0.971 ± 0.001 & 0.970 ± 0.002 & 0.973 ± 0.001 & 0.999 ± 0.000 \\
\textbullet &  &  & \textbullet & \textbullet &  & 0.999 ± 0.000 & 0.971 ± 0.005 & 0.969 ± 0.005 & 0.968 ± 0.005 & 0.971 ± 0.005 & 0.998 ± 0.000 \\
\textbullet &  &  & \textbullet & \textbullet & \textbullet & 0.999 ± 0.000 & 0.980 ± 0.002 & 0.978 ± 0.003 & 0.977 ± 0.003 & 0.980 ± 0.002 & 0.999 ± 0.000 \\
\textbullet &  & \textbullet &  &  &  & 0.999 ± 0.000 & 0.969 ± 0.003 & 0.967 ± 0.003 & 0.967 ± 0.003 & 0.969 ± 0.003 & 0.998 ± 0.000 \\
\textbullet &  & \textbullet &  &  & \textbullet & 1.000 ± 0.000 & 0.985 ± 0.002 & 0.984 ± 0.002 & 0.984 ± 0.002 & 0.985 ± 0.002 & 0.999 ± 0.000 \\
\textbullet &  & \textbullet &  & \textbullet &  & 0.999 ± 0.000 & 0.979 ± 0.002 & 0.978 ± 0.002 & 0.977 ± 0.003 & 0.979 ± 0.002 & 0.999 ± 0.000 \\
\textbullet &  & \textbullet &  & \textbullet & \textbullet & 1.000 ± 0.000 & 0.983 ± 0.002 & 0.982 ± 0.002 & 0.982 ± 0.002 & 0.983 ± 0.002 & 0.999 ± 0.000 \\
\textbullet &  & \textbullet & \textbullet &  &  & 0.998 ± 0.000 & 0.955 ± 0.004 & 0.951 ± 0.005 & 0.950 ± 0.005 & 0.955 ± 0.004 & 0.998 ± 0.000 \\
\textbullet &  & \textbullet & \textbullet &  & \textbullet & 0.999 ± 0.000 & 0.969 ± 0.002 & 0.967 ± 0.004 & 0.966 ± 0.005 & 0.969 ± 0.002 & 0.998 ± 0.000 \\
\textbullet &  & \textbullet & \textbullet & \textbullet &  & 0.999 ± 0.000 & 0.969 ± 0.004 & 0.967 ± 0.004 & 0.967 ± 0.004 & 0.969 ± 0.004 & 0.998 ± 0.000 \\
\textbullet &  & \textbullet & \textbullet & \textbullet & \textbullet & 0.999 ± 0.000 & 0.978 ± 0.002 & 0.976 ± 0.003 & 0.975 ± 0.003 & 0.978 ± 0.002 & 0.999 ± 0.000 \\
\textbullet & \textbullet &  &  &  &  & 0.998 ± 0.001 & 0.963 ± 0.004 & 0.961 ± 0.004 & 0.962 ± 0.004 & 0.963 ± 0.004 & 0.998 ± 0.000 \\
\textbullet & \textbullet &  &  &  & \textbullet & 0.999 ± 0.000 & 0.978 ± 0.004 & 0.977 ± 0.004 & 0.978 ± 0.004 & 0.978 ± 0.004 & 0.999 ± 0.000 \\
\textbullet & \textbullet &  &  & \textbullet &  & 0.999 ± 0.000 & 0.974 ± 0.003 & 0.971 ± 0.003 & 0.971 ± 0.003 & 0.974 ± 0.003 & 0.999 ± 0.000 \\
\textbullet & \textbullet &  &  & \textbullet & \textbullet & 0.999 ± 0.000 & 0.976 ± 0.002 & 0.974 ± 0.003 & 0.973 ± 0.004 & 0.976 ± 0.002 & 0.999 ± 0.000 \\
\textbullet & \textbullet &  & \textbullet &  &  & 0.998 ± 0.000 & 0.955 ± 0.002 & 0.952 ± 0.002 & 0.952 ± 0.002 & 0.955 ± 0.002 & 0.998 ± 0.000 \\
\textbullet & \textbullet &  & \textbullet &  & \textbullet & 0.999 ± 0.001 & 0.966 ± 0.002 & 0.964 ± 0.003 & 0.964 ± 0.003 & 0.966 ± 0.002 & 0.998 ± 0.000 \\
\textbullet & \textbullet &  & \textbullet & \textbullet &  & 0.999 ± 0.000 & 0.972 ± 0.002 & 0.969 ± 0.002 & 0.969 ± 0.002 & 0.972 ± 0.002 & 0.998 ± 0.000 \\
\textbullet & \textbullet &  & \textbullet & \textbullet & \textbullet & 0.999 ± 0.000 & 0.974 ± 0.003 & 0.971 ± 0.003 & 0.971 ± 0.003 & 0.974 ± 0.003 & 0.999 ± 0.000 \\
\textbullet & \textbullet & \textbullet &  &  &  & 0.998 ± 0.000 & 0.964 ± 0.002 & 0.962 ± 0.001 & 0.963 ± 0.002 & 0.964 ± 0.002 & 0.998 ± 0.000 \\
\textbullet & \textbullet & \textbullet &  &  & \textbullet & 0.999 ± 0.001 & 0.979 ± 0.002 & 0.978 ± 0.003 & 0.978 ± 0.003 & 0.979 ± 0.002 & 0.999 ± 0.000 \\
\textbullet & \textbullet & \textbullet &  & \textbullet &  & 0.999 ± 0.000 & 0.974 ± 0.004 & 0.972 ± 0.003 & 0.972 ± 0.003 & 0.974 ± 0.004 & 0.999 ± 0.000 \\
\textbullet & \textbullet & \textbullet &  & \textbullet & \textbullet & 0.999 ± 0.000 & 0.976 ± 0.002 & 0.975 ± 0.002 & 0.975 ± 0.002 & 0.976 ± 0.002 & 0.999 ± 0.000 \\
\textbullet & \textbullet & \textbullet & \textbullet &  &  & 0.998 ± 0.000 & 0.955 ± 0.006 & 0.952 ± 0.007 & 0.951 ± 0.007 & 0.955 ± 0.006 & 0.998 ± 0.000 \\
\textbullet & \textbullet & \textbullet & \textbullet &  & \textbullet & 0.999 ± 0.000 & 0.965 ± 0.003 & 0.964 ± 0.003 & 0.964 ± 0.003 & 0.965 ± 0.003 & 0.998 ± 0.000 \\
\textbullet & \textbullet & \textbullet & \textbullet & \textbullet &  & 0.999 ± 0.000 & 0.974 ± 0.003 & 0.971 ± 0.003 & 0.970 ± 0.002 & 0.974 ± 0.003 & 0.999 ± 0.000 \\
\textbullet & \textbullet & \textbullet & \textbullet & \textbullet & \textbullet & 0.999 ± 0.000 & 0.975 ± 0.003 & 0.972 ± 0.003 & 0.971 ± 0.003 & 0.975 ± 0.003 & 0.999 ± 0.000 \\
\end{longtable}

          \begin{longtable}{ccccccrrrrrr}
\caption{Classification metrics for MLP IF with different omics combination on TCGA dataset}\label{tab:perf_comb_MLPIntermediateFusion} \\
\toprule
mRNA & nc mRNA & miRNA & CNV & DNAm & Protein & AUROC & Accuracy & F1 & Precision & Recall & Specificity \\
\midrule
\endfirsthead
\caption[]{Classification metrics for MLP IF with different omics combination} \\
\toprule
mRNA & nc mRNA & miRNA & CNV & DNAm & Protein & AUROC & Accuracy & F1 & Precision & Recall & Specificity \\
\midrule
\endhead
\midrule
\multicolumn{12}{r}{Continued on next page} \\
\midrule
\endfoot
\bottomrule
\endlastfoot
 &  &  &  & \textbullet & \textbullet & 0.998 ± 0.001 & 0.980 ± 0.002 & 0.978 ± 0.002 & 0.978 ± 0.003 & 0.980 ± 0.002 & 0.999 ± 0.000 \\
 &  &  & \textbullet &  & \textbullet & 0.999 ± 0.000 & 0.973 ± 0.002 & 0.972 ± 0.002 & 0.972 ± 0.003 & 0.973 ± 0.002 & 0.998 ± 0.000 \\
 &  &  & \textbullet & \textbullet &  & 0.998 ± 0.001 & 0.974 ± 0.002 & 0.973 ± 0.002 & 0.973 ± 0.002 & 0.974 ± 0.002 & 0.999 ± 0.000 \\
 &  &  & \textbullet & \textbullet & \textbullet & 0.999 ± 0.000 & 0.983 ± 0.003 & 0.982 ± 0.003 & 0.982 ± 0.003 & 0.983 ± 0.003 & 0.999 ± 0.000 \\
 &  & \textbullet &  &  & \textbullet & 0.999 ± 0.000 & 0.979 ± 0.002 & 0.979 ± 0.002 & 0.979 ± 0.002 & 0.979 ± 0.002 & 0.999 ± 0.000 \\
 &  & \textbullet &  & \textbullet &  & 0.998 ± 0.000 & 0.968 ± 0.002 & 0.967 ± 0.002 & 0.968 ± 0.003 & 0.968 ± 0.002 & 0.998 ± 0.000 \\
 &  & \textbullet &  & \textbullet & \textbullet & 0.998 ± 0.001 & 0.977 ± 0.003 & 0.976 ± 0.003 & 0.977 ± 0.004 & 0.977 ± 0.003 & 0.999 ± 0.000 \\
 &  & \textbullet & \textbullet &  &  & 0.997 ± 0.000 & 0.931 ± 0.003 & 0.928 ± 0.003 & 0.928 ± 0.004 & 0.931 ± 0.003 & 0.996 ± 0.000 \\
 &  & \textbullet & \textbullet &  & \textbullet & 0.999 ± 0.000 & 0.975 ± 0.004 & 0.975 ± 0.004 & 0.976 ± 0.004 & 0.975 ± 0.004 & 0.999 ± 0.000 \\
 &  & \textbullet & \textbullet & \textbullet &  & 0.998 ± 0.000 & 0.972 ± 0.002 & 0.970 ± 0.002 & 0.972 ± 0.002 & 0.972 ± 0.002 & 0.999 ± 0.000 \\
 &  & \textbullet & \textbullet & \textbullet & \textbullet & 0.999 ± 0.001 & 0.979 ± 0.002 & 0.977 ± 0.002 & 0.978 ± 0.002 & 0.979 ± 0.002 & 0.999 ± 0.000 \\
 & \textbullet &  &  &  & \textbullet & 0.998 ± 0.000 & 0.966 ± 0.005 & 0.964 ± 0.005 & 0.964 ± 0.004 & 0.966 ± 0.005 & 0.998 ± 0.000 \\
 & \textbullet &  &  & \textbullet &  & 0.998 ± 0.000 & 0.971 ± 0.001 & 0.968 ± 0.002 & 0.968 ± 0.002 & 0.971 ± 0.001 & 0.998 ± 0.000 \\
 & \textbullet &  &  & \textbullet & \textbullet & 0.998 ± 0.001 & 0.976 ± 0.001 & 0.974 ± 0.002 & 0.973 ± 0.002 & 0.976 ± 0.001 & 0.999 ± 0.000 \\
 & \textbullet &  & \textbullet &  &  & 0.996 ± 0.001 & 0.956 ± 0.003 & 0.953 ± 0.003 & 0.952 ± 0.003 & 0.956 ± 0.003 & 0.998 ± 0.000 \\
 & \textbullet &  & \textbullet &  & \textbullet & 0.998 ± 0.001 & 0.973 ± 0.004 & 0.972 ± 0.005 & 0.973 ± 0.005 & 0.973 ± 0.004 & 0.999 ± 0.000 \\
 & \textbullet &  & \textbullet & \textbullet &  & 0.998 ± 0.000 & 0.972 ± 0.003 & 0.969 ± 0.003 & 0.970 ± 0.003 & 0.972 ± 0.003 & 0.998 ± 0.000 \\
 & \textbullet &  & \textbullet & \textbullet & \textbullet & 0.999 ± 0.000 & 0.977 ± 0.002 & 0.975 ± 0.002 & 0.975 ± 0.002 & 0.977 ± 0.002 & 0.999 ± 0.000 \\
 & \textbullet & \textbullet &  &  &  & 0.997 ± 0.001 & 0.957 ± 0.003 & 0.957 ± 0.004 & 0.959 ± 0.003 & 0.957 ± 0.003 & 0.998 ± 0.000 \\
 & \textbullet & \textbullet &  &  & \textbullet & 0.998 ± 0.001 & 0.972 ± 0.004 & 0.971 ± 0.003 & 0.972 ± 0.003 & 0.972 ± 0.004 & 0.999 ± 0.000 \\
 & \textbullet & \textbullet &  & \textbullet &  & 0.998 ± 0.001 & 0.971 ± 0.003 & 0.969 ± 0.003 & 0.969 ± 0.003 & 0.971 ± 0.003 & 0.998 ± 0.000 \\
 & \textbullet & \textbullet &  & \textbullet & \textbullet & 0.998 ± 0.000 & 0.975 ± 0.003 & 0.973 ± 0.002 & 0.973 ± 0.002 & 0.975 ± 0.003 & 0.999 ± 0.000 \\
 & \textbullet & \textbullet & \textbullet &  &  & 0.997 ± 0.001 & 0.963 ± 0.004 & 0.961 ± 0.004 & 0.961 ± 0.003 & 0.963 ± 0.004 & 0.998 ± 0.000 \\
 & \textbullet & \textbullet & \textbullet &  & \textbullet & 0.998 ± 0.000 & 0.974 ± 0.003 & 0.973 ± 0.002 & 0.974 ± 0.002 & 0.974 ± 0.003 & 0.999 ± 0.000 \\
 & \textbullet & \textbullet & \textbullet & \textbullet &  & 0.998 ± 0.001 & 0.971 ± 0.003 & 0.969 ± 0.002 & 0.969 ± 0.002 & 0.971 ± 0.003 & 0.998 ± 0.000 \\
 & \textbullet & \textbullet & \textbullet & \textbullet & \textbullet & 0.999 ± 0.000 & 0.976 ± 0.002 & 0.974 ± 0.001 & 0.974 ± 0.001 & 0.976 ± 0.002 & 0.999 ± 0.000 \\
\textbullet &  &  &  &  & \textbullet & 0.999 ± 0.000 & 0.976 ± 0.003 & 0.974 ± 0.003 & 0.974 ± 0.003 & 0.976 ± 0.003 & 0.999 ± 0.000 \\
\textbullet &  &  &  & \textbullet &  & 0.998 ± 0.001 & 0.980 ± 0.003 & 0.977 ± 0.003 & 0.977 ± 0.003 & 0.980 ± 0.003 & 0.999 ± 0.000 \\
\textbullet &  &  &  & \textbullet & \textbullet & 0.999 ± 0.001 & 0.982 ± 0.001 & 0.981 ± 0.001 & 0.980 ± 0.002 & 0.982 ± 0.001 & 0.999 ± 0.000 \\
\textbullet &  &  & \textbullet &  &  & 0.998 ± 0.001 & 0.970 ± 0.002 & 0.967 ± 0.002 & 0.967 ± 0.002 & 0.970 ± 0.002 & 0.998 ± 0.000 \\
\textbullet &  &  & \textbullet &  & \textbullet & 0.999 ± 0.000 & 0.979 ± 0.003 & 0.977 ± 0.003 & 0.977 ± 0.003 & 0.979 ± 0.003 & 0.999 ± 0.000 \\
\textbullet &  &  & \textbullet & \textbullet &  & 0.999 ± 0.001 & 0.978 ± 0.002 & 0.976 ± 0.002 & 0.976 ± 0.001 & 0.978 ± 0.002 & 0.999 ± 0.000 \\
\textbullet &  &  & \textbullet & \textbullet & \textbullet & 0.999 ± 0.000 & 0.982 ± 0.002 & 0.980 ± 0.002 & 0.980 ± 0.002 & 0.982 ± 0.002 & 0.999 ± 0.000 \\
\textbullet &  & \textbullet &  &  &  & 0.998 ± 0.000 & 0.967 ± 0.002 & 0.965 ± 0.003 & 0.964 ± 0.004 & 0.967 ± 0.002 & 0.998 ± 0.000 \\
\textbullet &  & \textbullet &  &  & \textbullet & 0.999 ± 0.000 & 0.973 ± 0.002 & 0.971 ± 0.003 & 0.970 ± 0.003 & 0.973 ± 0.002 & 0.999 ± 0.000 \\
\textbullet &  & \textbullet &  & \textbullet &  & 0.998 ± 0.001 & 0.979 ± 0.002 & 0.977 ± 0.001 & 0.977 ± 0.001 & 0.979 ± 0.002 & 0.999 ± 0.000 \\
\textbullet &  & \textbullet &  & \textbullet & \textbullet & 0.999 ± 0.000 & 0.980 ± 0.002 & 0.978 ± 0.002 & 0.978 ± 0.002 & 0.980 ± 0.002 & 0.999 ± 0.000 \\
\textbullet &  & \textbullet & \textbullet &  &  & 0.999 ± 0.000 & 0.969 ± 0.003 & 0.966 ± 0.003 & 0.966 ± 0.003 & 0.969 ± 0.003 & 0.998 ± 0.000 \\
\textbullet &  & \textbullet & \textbullet &  & \textbullet & 0.999 ± 0.000 & 0.976 ± 0.004 & 0.975 ± 0.004 & 0.975 ± 0.004 & 0.976 ± 0.004 & 0.999 ± 0.000 \\
\textbullet &  & \textbullet & \textbullet & \textbullet &  & 0.999 ± 0.001 & 0.976 ± 0.003 & 0.974 ± 0.002 & 0.974 ± 0.002 & 0.976 ± 0.003 & 0.999 ± 0.000 \\
\textbullet &  & \textbullet & \textbullet & \textbullet & \textbullet & 0.999 ± 0.000 & 0.980 ± 0.002 & 0.977 ± 0.002 & 0.977 ± 0.002 & 0.980 ± 0.002 & 0.999 ± 0.000 \\
\textbullet & \textbullet &  &  &  &  & 0.997 ± 0.001 & 0.962 ± 0.002 & 0.959 ± 0.003 & 0.959 ± 0.003 & 0.962 ± 0.002 & 0.998 ± 0.000 \\
\textbullet & \textbullet &  &  &  & \textbullet & 0.998 ± 0.002 & 0.969 ± 0.002 & 0.966 ± 0.003 & 0.966 ± 0.004 & 0.969 ± 0.002 & 0.998 ± 0.000 \\
\textbullet & \textbullet &  &  & \textbullet &  & 0.998 ± 0.001 & 0.975 ± 0.002 & 0.972 ± 0.002 & 0.970 ± 0.003 & 0.975 ± 0.002 & 0.999 ± 0.000 \\
\textbullet & \textbullet &  &  & \textbullet & \textbullet & 0.999 ± 0.000 & 0.977 ± 0.003 & 0.974 ± 0.003 & 0.973 ± 0.003 & 0.977 ± 0.003 & 0.999 ± 0.000 \\
\textbullet & \textbullet &  & \textbullet &  &  & 0.998 ± 0.001 & 0.965 ± 0.003 & 0.963 ± 0.003 & 0.963 ± 0.003 & 0.965 ± 0.003 & 0.998 ± 0.000 \\
\textbullet & \textbullet &  & \textbullet &  & \textbullet & 0.998 ± 0.001 & 0.970 ± 0.003 & 0.968 ± 0.003 & 0.968 ± 0.003 & 0.970 ± 0.003 & 0.998 ± 0.000 \\
\textbullet & \textbullet &  & \textbullet & \textbullet &  & 0.998 ± 0.001 & 0.973 ± 0.002 & 0.970 ± 0.002 & 0.969 ± 0.003 & 0.973 ± 0.002 & 0.999 ± 0.000 \\
\textbullet & \textbullet &  & \textbullet & \textbullet & \textbullet & 0.999 ± 0.000 & 0.977 ± 0.003 & 0.975 ± 0.003 & 0.974 ± 0.003 & 0.977 ± 0.003 & 0.999 ± 0.000 \\
\textbullet & \textbullet & \textbullet &  &  &  & 0.997 ± 0.001 & 0.962 ± 0.004 & 0.960 ± 0.005 & 0.959 ± 0.006 & 0.962 ± 0.004 & 0.998 ± 0.000 \\
\textbullet & \textbullet & \textbullet &  &  & \textbullet & 0.998 ± 0.001 & 0.968 ± 0.001 & 0.966 ± 0.002 & 0.966 ± 0.002 & 0.968 ± 0.001 & 0.998 ± 0.000 \\
\textbullet & \textbullet & \textbullet &  & \textbullet &  & 0.998 ± 0.001 & 0.971 ± 0.003 & 0.968 ± 0.004 & 0.968 ± 0.004 & 0.971 ± 0.003 & 0.998 ± 0.000 \\
\textbullet & \textbullet & \textbullet &  & \textbullet & \textbullet & 0.998 ± 0.001 & 0.975 ± 0.004 & 0.972 ± 0.004 & 0.972 ± 0.004 & 0.975 ± 0.004 & 0.999 ± 0.000 \\
\textbullet & \textbullet & \textbullet & \textbullet &  &  & 0.998 ± 0.000 & 0.966 ± 0.003 & 0.964 ± 0.003 & 0.964 ± 0.003 & 0.966 ± 0.003 & 0.998 ± 0.000 \\
\textbullet & \textbullet & \textbullet & \textbullet &  & \textbullet & 0.999 ± 0.000 & 0.970 ± 0.001 & 0.968 ± 0.001 & 0.968 ± 0.001 & 0.970 ± 0.001 & 0.998 ± 0.000 \\
\textbullet & \textbullet & \textbullet & \textbullet & \textbullet &  & 0.999 ± 0.000 & 0.972 ± 0.002 & 0.970 ± 0.002 & 0.970 ± 0.002 & 0.972 ± 0.002 & 0.999 ± 0.000 \\
\textbullet & \textbullet & \textbullet & \textbullet & \textbullet & \textbullet & 0.999 ± 0.000 & 0.975 ± 0.003 & 0.972 ± 0.002 & 0.971 ± 0.002 & 0.975 ± 0.003 & 0.999 ± 0.000 \\
\end{longtable}

          \begin{longtable}{ccccccrrrrrr}
\caption{Classification metrics for MOGONET with different omics combination on TCGA dataset} \label{tab:perf_comb_MOGONET} \\
\toprule
mRNA & nc mRNA & miRNA & CNV & DNAm & Protein & AUROC & Accuracy & F1 & Precision & Recall & Specificity \\
\midrule
\endfirsthead
\caption[]{Classification metrics for MOGONET with different omics combination} \\
\toprule
mRNA & nc mRNA & miRNA & CNV & DNAm & Protein & AUROC & Accuracy & F1 & Precision & Recall & Specificity \\
\midrule
\endhead
\midrule
\multicolumn{12}{r}{Continued on next page} \\
\midrule
\endfoot
\bottomrule
\endlastfoot
 &  &  &  & \textbullet & \textbullet & 0.946 ± 0.000 & 0.986 ± 0.001 & 0.986 ± 0.001 & 0.987 ± 0.001 & 0.986 ± 0.001 & 0.999 ± 0.000 \\
 &  &  & \textbullet &  & \textbullet & 0.946 ± 0.000 & 0.975 ± 0.004 & 0.975 ± 0.005 & 0.976 ± 0.004 & 0.975 ± 0.004 & 0.999 ± 0.000 \\
 &  &  & \textbullet & \textbullet &  & 0.942 ± 0.001 & 0.919 ± 0.011 & 0.921 ± 0.011 & 0.927 ± 0.009 & 0.919 ± 0.011 & 0.996 ± 0.000 \\
 &  &  & \textbullet & \textbullet & \textbullet & 0.946 ± 0.000 & 0.983 ± 0.004 & 0.983 ± 0.004 & 0.985 ± 0.003 & 0.983 ± 0.004 & 0.999 ± 0.000 \\
 &  & \textbullet &  &  & \textbullet & 0.946 ± 0.001 & 0.966 ± 0.005 & 0.968 ± 0.005 & 0.971 ± 0.005 & 0.966 ± 0.005 & 0.998 ± 0.000 \\
 &  & \textbullet &  & \textbullet &  & 0.943 ± 0.001 & 0.931 ± 0.006 & 0.934 ± 0.005 & 0.940 ± 0.005 & 0.931 ± 0.006 & 0.997 ± 0.000 \\
 &  & \textbullet &  & \textbullet & \textbullet & 0.946 ± 0.001 & 0.980 ± 0.004 & 0.981 ± 0.004 & 0.983 ± 0.004 & 0.980 ± 0.004 & 0.999 ± 0.000 \\
 &  & \textbullet & \textbullet &  &  & 0.936 ± 0.001 & 0.860 ± 0.009 & 0.862 ± 0.009 & 0.869 ± 0.008 & 0.860 ± 0.009 & 0.993 ± 0.001 \\
 &  & \textbullet & \textbullet &  & \textbullet & 0.946 ± 0.000 & 0.971 ± 0.007 & 0.972 ± 0.008 & 0.975 ± 0.008 & 0.971 ± 0.007 & 0.999 ± 0.000 \\
 &  & \textbullet & \textbullet & \textbullet &  & 0.943 ± 0.001 & 0.935 ± 0.005 & 0.938 ± 0.004 & 0.944 ± 0.004 & 0.935 ± 0.005 & 0.997 ± 0.000 \\
 & \textbullet &  &  &  & \textbullet & 0.946 ± 0.001 & 0.981 ± 0.004 & 0.981 ± 0.004 & 0.982 ± 0.004 & 0.981 ± 0.004 & 0.999 ± 0.000 \\
 & \textbullet &  &  & \textbullet &  & 0.943 ± 0.001 & 0.951 ± 0.003 & 0.952 ± 0.003 & 0.954 ± 0.004 & 0.951 ± 0.003 & 0.998 ± 0.000 \\
 & \textbullet &  &  & \textbullet & \textbullet & 0.946 ± 0.001 & 0.981 ± 0.001 & 0.980 ± 0.001 & 0.980 ± 0.001 & 0.981 ± 0.001 & 0.999 ± 0.000 \\
 & \textbullet &  & \textbullet &  &  & 0.940 ± 0.001 & 0.904 ± 0.009 & 0.908 ± 0.009 & 0.916 ± 0.009 & 0.904 ± 0.009 & 0.995 ± 0.000 \\
 & \textbullet &  & \textbullet &  & \textbullet & 0.945 ± 0.000 & 0.977 ± 0.001 & 0.978 ± 0.002 & 0.980 ± 0.002 & 0.977 ± 0.001 & 0.999 ± 0.000 \\
 & \textbullet &  & \textbullet & \textbullet &  & 0.944 ± 0.000 & 0.944 ± 0.005 & 0.946 ± 0.005 & 0.949 ± 0.005 & 0.944 ± 0.005 & 0.997 ± 0.000 \\
 & \textbullet & \textbullet &  &  &  & 0.940 ± 0.001 & 0.909 ± 0.004 & 0.912 ± 0.004 & 0.917 ± 0.006 & 0.909 ± 0.004 & 0.996 ± 0.000 \\
 & \textbullet & \textbullet &  &  & \textbullet & 0.946 ± 0.001 & 0.975 ± 0.006 & 0.975 ± 0.005 & 0.976 ± 0.004 & 0.975 ± 0.006 & 0.999 ± 0.000 \\
 & \textbullet & \textbullet &  & \textbullet &  & 0.943 ± 0.001 & 0.944 ± 0.006 & 0.947 ± 0.005 & 0.952 ± 0.006 & 0.944 ± 0.006 & 0.997 ± 0.000 \\
 & \textbullet & \textbullet & \textbullet &  &  & 0.941 ± 0.001 & 0.916 ± 0.006 & 0.919 ± 0.006 & 0.925 ± 0.007 & 0.916 ± 0.006 & 0.996 ± 0.000 \\
\textbullet &  &  &  &  & \textbullet & 0.946 ± 0.001 & 0.981 ± 0.004 & 0.983 ± 0.004 & 0.985 ± 0.003 & 0.981 ± 0.004 & 0.999 ± 0.000 \\
\textbullet &  &  &  & \textbullet &  & 0.945 ± 0.001 & 0.959 ± 0.002 & 0.961 ± 0.002 & 0.965 ± 0.003 & 0.959 ± 0.002 & 0.998 ± 0.000 \\
\textbullet &  &  &  & \textbullet & \textbullet & 0.946 ± 0.001 & 0.982 ± 0.001 & 0.983 ± 0.002 & 0.985 ± 0.002 & 0.982 ± 0.001 & 0.999 ± 0.000 \\
\textbullet &  &  & \textbullet &  &  & 0.943 ± 0.001 & 0.921 ± 0.011 & 0.926 ± 0.010 & 0.933 ± 0.009 & 0.921 ± 0.011 & 0.996 ± 0.001 \\
\textbullet &  &  & \textbullet &  & \textbullet & 0.946 ± 0.000 & 0.976 ± 0.003 & 0.978 ± 0.003 & 0.980 ± 0.002 & 0.976 ± 0.003 & 0.999 ± 0.000 \\
\textbullet &  &  & \textbullet & \textbullet &  & 0.945 ± 0.000 & 0.957 ± 0.003 & 0.958 ± 0.003 & 0.962 ± 0.002 & 0.957 ± 0.003 & 0.998 ± 0.000 \\
\textbullet &  & \textbullet &  &  &  & 0.943 ± 0.001 & 0.934 ± 0.005 & 0.938 ± 0.005 & 0.943 ± 0.004 & 0.934 ± 0.005 & 0.997 ± 0.000 \\
\textbullet &  & \textbullet &  &  & \textbullet & 0.946 ± 0.001 & 0.978 ± 0.003 & 0.979 ± 0.002 & 0.981 ± 0.001 & 0.978 ± 0.003 & 0.999 ± 0.000 \\
\textbullet &  & \textbullet &  & \textbullet &  & 0.944 ± 0.001 & 0.957 ± 0.005 & 0.959 ± 0.004 & 0.963 ± 0.003 & 0.957 ± 0.005 & 0.998 ± 0.000 \\
\textbullet &  & \textbullet & \textbullet &  &  & 0.942 ± 0.001 & 0.933 ± 0.006 & 0.936 ± 0.005 & 0.942 ± 0.005 & 0.933 ± 0.006 & 0.997 ± 0.000 \\
\textbullet & \textbullet &  &  &  &  & 0.943 ± 0.000 & 0.941 ± 0.009 & 0.942 ± 0.007 & 0.945 ± 0.005 & 0.941 ± 0.009 & 0.997 ± 0.000 \\
\textbullet & \textbullet &  &  &  & \textbullet & 0.945 ± 0.001 & 0.976 ± 0.003 & 0.975 ± 0.003 & 0.975 ± 0.002 & 0.976 ± 0.003 & 0.999 ± 0.000 \\
\textbullet & \textbullet &  &  & \textbullet &  & 0.944 ± 0.000 & 0.955 ± 0.004 & 0.955 ± 0.003 & 0.958 ± 0.002 & 0.955 ± 0.004 & 0.998 ± 0.000 \\
\textbullet & \textbullet &  & \textbullet &  &  & 0.943 ± 0.001 & 0.939 ± 0.003 & 0.940 ± 0.003 & 0.944 ± 0.003 & 0.939 ± 0.003 & 0.997 ± 0.000 \\
\textbullet & \textbullet & \textbullet &  &  &  & 0.942 ± 0.000 & 0.941 ± 0.007 & 0.943 ± 0.008 & 0.947 ± 0.010 & 0.941 ± 0.007 & 0.997 ± 0.000 \\
\end{longtable}

          \begin{longtable}{ccccccrrrrrr}
\caption{Classification metrics for P-Net with different omics combination on TCGA dataset}\label{tab:perf_comb_PNet} \\
\toprule
mRNA & nc mRNA & miRNA & CNV & DNAm & Protein & AUROC & Accuracy & F1 & Precision & Recall & Specificity \\
\midrule
\endfirsthead
\caption[]{Classification metrics for P-Net with different omics combination} \\
\toprule
mRNA & nc mRNA & miRNA & CNV & DNAm & Protein & AUROC & Accuracy & F1 & Precision & Recall & Specificity \\
\midrule
\endhead
\midrule
\multicolumn{12}{r}{Continued on next page} \\
\midrule
\endfoot
\bottomrule
\endlastfoot
 &  &  & \textbullet & \textbullet &  & 0.997 ± 0.001 & 0.927 ± 0.007 & 0.924 ± 0.008 & 0.923 ± 0.009 & 0.927 ± 0.007 & 0.996 ± 0.000 \\
 &  & \textbullet &  & \textbullet &  & 0.997 ± 0.000 & 0.929 ± 0.007 & 0.927 ± 0.008 & 0.927 ± 0.009 & 0.929 ± 0.007 & 0.996 ± 0.000 \\
 &  & \textbullet & \textbullet &  &  & 0.952 ± 0.005 & 0.661 ± 0.016 & 0.645 ± 0.016 & 0.645 ± 0.014 & 0.661 ± 0.016 & 0.980 ± 0.001 \\
 &  & \textbullet & \textbullet & \textbullet &  & 0.997 ± 0.000 & 0.924 ± 0.004 & 0.921 ± 0.005 & 0.919 ± 0.006 & 0.924 ± 0.004 & 0.996 ± 0.000 \\
 & \textbullet &  &  & \textbullet &  & 0.997 ± 0.001 & 0.931 ± 0.003 & 0.930 ± 0.004 & 0.930 ± 0.005 & 0.931 ± 0.003 & 0.996 ± 0.000 \\
 & \textbullet &  & \textbullet &  &  & 0.949 ± 0.003 & 0.643 ± 0.007 & 0.626 ± 0.009 & 0.636 ± 0.011 & 0.643 ± 0.007 & 0.978 ± 0.000 \\
 & \textbullet &  & \textbullet & \textbullet &  & 0.997 ± 0.001 & 0.930 ± 0.009 & 0.928 ± 0.008 & 0.927 ± 0.007 & 0.930 ± 0.009 & 0.996 ± 0.000 \\
 & \textbullet & \textbullet &  & \textbullet &  & 0.997 ± 0.000 & 0.929 ± 0.006 & 0.928 ± 0.007 & 0.928 ± 0.007 & 0.929 ± 0.006 & 0.996 ± 0.000 \\
 & \textbullet & \textbullet & \textbullet &  &  & 0.954 ± 0.002 & 0.670 ± 0.017 & 0.660 ± 0.019 & 0.665 ± 0.017 & 0.670 ± 0.017 & 0.980 ± 0.001 \\
 & \textbullet & \textbullet & \textbullet & \textbullet &  & 0.996 ± 0.000 & 0.929 ± 0.003 & 0.927 ± 0.002 & 0.926 ± 0.002 & 0.929 ± 0.003 & 0.996 ± 0.000 \\
\textbullet &  &  &  & \textbullet &  & 0.998 ± 0.000 & 0.961 ± 0.009 & 0.959 ± 0.007 & 0.959 ± 0.005 & 0.961 ± 0.009 & 0.998 ± 0.000 \\
\textbullet &  &  & \textbullet &  &  & 0.998 ± 0.001 & 0.948 ± 0.007 & 0.947 ± 0.008 & 0.947 ± 0.009 & 0.948 ± 0.007 & 0.997 ± 0.000 \\
\textbullet &  &  & \textbullet & \textbullet &  & 0.998 ± 0.001 & 0.951 ± 0.008 & 0.949 ± 0.009 & 0.947 ± 0.010 & 0.951 ± 0.008 & 0.997 ± 0.001 \\
\textbullet &  & \textbullet &  &  &  & 0.997 ± 0.002 & 0.951 ± 0.006 & 0.948 ± 0.007 & 0.947 ± 0.008 & 0.951 ± 0.006 & 0.997 ± 0.000 \\
\textbullet &  & \textbullet &  & \textbullet &  & 0.998 ± 0.001 & 0.954 ± 0.010 & 0.954 ± 0.009 & 0.955 ± 0.008 & 0.954 ± 0.010 & 0.998 ± 0.000 \\
\textbullet &  & \textbullet & \textbullet &  &  & 0.997 ± 0.001 & 0.947 ± 0.004 & 0.945 ± 0.004 & 0.944 ± 0.004 & 0.947 ± 0.004 & 0.997 ± 0.000 \\
\textbullet &  & \textbullet & \textbullet & \textbullet &  & 0.998 ± 0.000 & 0.959 ± 0.003 & 0.957 ± 0.002 & 0.955 ± 0.001 & 0.959 ± 0.003 & 0.998 ± 0.000 \\
\textbullet & \textbullet &  &  &  &  & 0.998 ± 0.001 & 0.946 ± 0.006 & 0.943 ± 0.005 & 0.943 ± 0.007 & 0.946 ± 0.006 & 0.997 ± 0.000 \\
\textbullet & \textbullet &  &  & \textbullet &  & 0.999 ± 0.001 & 0.962 ± 0.005 & 0.960 ± 0.005 & 0.959 ± 0.004 & 0.962 ± 0.005 & 0.998 ± 0.000 \\
\textbullet & \textbullet &  & \textbullet &  &  & 0.997 ± 0.001 & 0.945 ± 0.011 & 0.942 ± 0.012 & 0.941 ± 0.012 & 0.945 ± 0.011 & 0.997 ± 0.001 \\
\textbullet & \textbullet &  & \textbullet & \textbullet &  & 0.999 ± 0.000 & 0.956 ± 0.006 & 0.956 ± 0.005 & 0.957 ± 0.005 & 0.956 ± 0.006 & 0.998 ± 0.000 \\
\textbullet & \textbullet & \textbullet &  &  &  & 0.997 ± 0.001 & 0.949 ± 0.004 & 0.947 ± 0.004 & 0.946 ± 0.004 & 0.949 ± 0.004 & 0.997 ± 0.000 \\
\textbullet & \textbullet & \textbullet &  & \textbullet &  & 0.998 ± 0.001 & 0.956 ± 0.006 & 0.955 ± 0.006 & 0.954 ± 0.006 & 0.956 ± 0.006 & 0.998 ± 0.000 \\
\textbullet & \textbullet & \textbullet & \textbullet &  &  & 0.997 ± 0.000 & 0.950 ± 0.002 & 0.947 ± 0.003 & 0.946 ± 0.004 & 0.950 ± 0.002 & 0.997 ± 0.000 \\
\textbullet & \textbullet & \textbullet & \textbullet & \textbullet &  & 0.998 ± 0.001 & 0.959 ± 0.006 & 0.956 ± 0.006 & 0.955 ± 0.006 & 0.959 ± 0.006 & 0.998 ± 0.000 \\
\end{longtable}

      \subsection{Models performances on various combinations of omics on CCLE dataset}
          \begin{longtable}{ccccccrrrrrr}
\caption{Classification metrics for AttOmics with different omics combination on CCLE dataset.}\label{tab:perf_comb_AttOmics_CCLE} \\
\toprule
mRNA & miRNA & CNV & DNAm & Protein & Metabolomics & AUROC & Accuracy & F1 & Precision & Recall & Specificity \\
\midrule
\endfirsthead
\caption[]{Classification metrics for AttOmics with different omics combination on CCLE dataset.} \\
\toprule
mRNA & miRNA & CNV & DNAm & Protein & Metabolomics & AUROC & Accuracy & F1 & Precision & Recall & Specificity \\
\midrule
\endhead
\midrule
\multicolumn{12}{r}{Continued on next page} \\
\midrule
\endfoot
\bottomrule
\endlastfoot
 &  &  &  &  & \textbullet & 0.894 ± 0.005 & 0.481 ± 0.039 & 0.460 ± 0.035 & 0.481 ± 0.029 & 0.481 ± 0.039 & 0.965 ± 0.003 \\
 &  &  &  & \textbullet &  & 0.932 ± 0.005 & 0.669 ± 0.015 & 0.650 ± 0.023 & 0.671 ± 0.025 & 0.669 ± 0.015 & 0.977 ± 0.001 \\
 &  &  & \textbullet &  &  & 0.971 ± 0.006 & 0.769 ± 0.032 & 0.759 ± 0.028 & 0.793 ± 0.027 & 0.769 ± 0.032 & 0.985 ± 0.002 \\
 &  & \textbullet &  &  &  & 0.914 ± 0.011 & 0.577 ± 0.030 & 0.565 ± 0.032 & 0.585 ± 0.043 & 0.577 ± 0.030 & 0.972 ± 0.002 \\
 & \textbullet &  &  &  &  & 0.947 ± 0.007 & 0.727 ± 0.024 & 0.705 ± 0.023 & 0.704 ± 0.022 & 0.727 ± 0.024 & 0.981 ± 0.002 \\
\textbullet &  &  &  &  &  & 0.970 ± 0.007 & 0.766 ± 0.024 & 0.757 ± 0.025 & 0.787 ± 0.023 & 0.766 ± 0.024 & 0.984 ± 0.002 \\
\end{longtable}

          \begin{longtable}{ccccccrrrrrr}
\caption{Classification metrics for AttOmics EF with different omics combination on CCLE dataset.}\label{tab:perf_comb_AttOmics EF_CCLE} \\
\toprule
mRNA & miRNA & CNV & DNAm & Protein & Metabolomics & AUROC & Accuracy & F1 & Precision & Recall & Specificity \\
\midrule
\endfirsthead
\caption[]{Classification metrics for AttOmics EF with different omics combination on CCLE dataset.} \\
\toprule
mRNA & miRNA & CNV & DNAm & Protein & Metabolomics & AUROC & Accuracy & F1 & Precision & Recall & Specificity \\
\midrule
\endhead
\midrule
\multicolumn{12}{r}{Continued on next page} \\
\midrule
\endfoot
\bottomrule
\endlastfoot
 &  &  &  & \textbullet & \textbullet & 0.881 ± 0.016 & 0.465 ± 0.053 & 0.433 ± 0.058 & 0.466 ± 0.077 & 0.465 ± 0.053 & 0.964 ± 0.003 \\
 &  &  & \textbullet &  & \textbullet & 0.984 ± 0.003 & 0.784 ± 0.048 & 0.777 ± 0.048 & 0.798 ± 0.049 & 0.784 ± 0.048 & 0.986 ± 0.004 \\
 &  &  & \textbullet & \textbullet &  & 0.984 ± 0.003 & 0.797 ± 0.016 & 0.791 ± 0.019 & 0.822 ± 0.022 & 0.797 ± 0.016 & 0.987 ± 0.001 \\
 &  &  & \textbullet & \textbullet & \textbullet & 0.983 ± 0.004 & 0.780 ± 0.034 & 0.778 ± 0.034 & 0.801 ± 0.041 & 0.780 ± 0.034 & 0.986 ± 0.002 \\
 &  & \textbullet &  &  & \textbullet & 0.917 ± 0.008 & 0.577 ± 0.037 & 0.547 ± 0.032 & 0.544 ± 0.032 & 0.577 ± 0.037 & 0.971 ± 0.003 \\
 &  & \textbullet &  & \textbullet &  & 0.922 ± 0.006 & 0.576 ± 0.024 & 0.544 ± 0.027 & 0.544 ± 0.040 & 0.576 ± 0.024 & 0.971 ± 0.002 \\
 &  & \textbullet &  & \textbullet & \textbullet & 0.923 ± 0.007 & 0.613 ± 0.033 & 0.589 ± 0.034 & 0.594 ± 0.040 & 0.613 ± 0.033 & 0.974 ± 0.002 \\
 &  & \textbullet & \textbullet &  &  & 0.973 ± 0.005 & 0.753 ± 0.033 & 0.732 ± 0.033 & 0.743 ± 0.030 & 0.753 ± 0.033 & 0.984 ± 0.002 \\
 &  & \textbullet & \textbullet &  & \textbullet & 0.972 ± 0.004 & 0.753 ± 0.027 & 0.734 ± 0.033 & 0.752 ± 0.040 & 0.753 ± 0.027 & 0.984 ± 0.002 \\
 &  & \textbullet & \textbullet & \textbullet &  & 0.975 ± 0.002 & 0.752 ± 0.024 & 0.732 ± 0.026 & 0.752 ± 0.041 & 0.752 ± 0.024 & 0.983 ± 0.002 \\
 &  & \textbullet & \textbullet & \textbullet & \textbullet & 0.973 ± 0.007 & 0.757 ± 0.012 & 0.731 ± 0.011 & 0.737 ± 0.024 & 0.757 ± 0.012 & 0.984 ± 0.001 \\
 & \textbullet &  &  &  & \textbullet & 0.913 ± 0.002 & 0.575 ± 0.029 & 0.553 ± 0.030 & 0.591 ± 0.045 & 0.575 ± 0.029 & 0.971 ± 0.002 \\
 & \textbullet &  &  & \textbullet &  & 0.927 ± 0.014 & 0.628 ± 0.061 & 0.606 ± 0.077 & 0.613 ± 0.090 & 0.628 ± 0.061 & 0.974 ± 0.004 \\
 & \textbullet &  &  & \textbullet & \textbullet & 0.939 ± 0.006 & 0.657 ± 0.035 & 0.643 ± 0.049 & 0.659 ± 0.064 & 0.657 ± 0.035 & 0.977 ± 0.003 \\
 & \textbullet &  & \textbullet &  &  & 0.982 ± 0.006 & 0.793 ± 0.012 & 0.790 ± 0.016 & 0.816 ± 0.030 & 0.793 ± 0.012 & 0.987 ± 0.001 \\
 & \textbullet &  & \textbullet &  & \textbullet & 0.982 ± 0.004 & 0.777 ± 0.044 & 0.775 ± 0.042 & 0.795 ± 0.043 & 0.777 ± 0.044 & 0.986 ± 0.003 \\
 & \textbullet &  & \textbullet & \textbullet &  & 0.984 ± 0.003 & 0.794 ± 0.017 & 0.787 ± 0.015 & 0.810 ± 0.016 & 0.794 ± 0.017 & 0.987 ± 0.001 \\
 & \textbullet &  & \textbullet & \textbullet & \textbullet & 0.985 ± 0.002 & 0.790 ± 0.016 & 0.786 ± 0.016 & 0.808 ± 0.013 & 0.790 ± 0.016 & 0.987 ± 0.001 \\
 & \textbullet & \textbullet &  &  &  & 0.918 ± 0.011 & 0.563 ± 0.049 & 0.541 ± 0.055 & 0.551 ± 0.068 & 0.563 ± 0.049 & 0.971 ± 0.004 \\
 & \textbullet & \textbullet &  &  & \textbullet & 0.921 ± 0.013 & 0.584 ± 0.032 & 0.562 ± 0.028 & 0.576 ± 0.020 & 0.584 ± 0.032 & 0.972 ± 0.002 \\
 & \textbullet & \textbullet &  & \textbullet &  & 0.923 ± 0.009 & 0.579 ± 0.025 & 0.552 ± 0.021 & 0.551 ± 0.022 & 0.579 ± 0.025 & 0.971 ± 0.001 \\
 & \textbullet & \textbullet &  & \textbullet & \textbullet & 0.927 ± 0.007 & 0.562 ± 0.019 & 0.528 ± 0.019 & 0.531 ± 0.029 & 0.562 ± 0.019 & 0.970 ± 0.002 \\
 & \textbullet & \textbullet & \textbullet &  &  & 0.970 ± 0.003 & 0.759 ± 0.046 & 0.736 ± 0.062 & 0.764 ± 0.062 & 0.759 ± 0.046 & 0.984 ± 0.003 \\
 & \textbullet & \textbullet & \textbullet &  & \textbullet & 0.968 ± 0.007 & 0.737 ± 0.023 & 0.714 ± 0.020 & 0.737 ± 0.034 & 0.737 ± 0.023 & 0.982 ± 0.002 \\
 & \textbullet & \textbullet & \textbullet & \textbullet &  & 0.972 ± 0.005 & 0.754 ± 0.025 & 0.733 ± 0.031 & 0.751 ± 0.044 & 0.754 ± 0.025 & 0.983 ± 0.001 \\
 & \textbullet & \textbullet & \textbullet & \textbullet & \textbullet & 0.970 ± 0.002 & 0.753 ± 0.027 & 0.736 ± 0.035 & 0.747 ± 0.042 & 0.753 ± 0.027 & 0.983 ± 0.002 \\
\textbullet &  &  &  &  & \textbullet & 0.967 ± 0.007 & 0.764 ± 0.033 & 0.759 ± 0.035 & 0.792 ± 0.045 & 0.764 ± 0.033 & 0.984 ± 0.002 \\
\textbullet &  &  &  & \textbullet &  & 0.969 ± 0.004 & 0.751 ± 0.027 & 0.739 ± 0.030 & 0.764 ± 0.050 & 0.751 ± 0.027 & 0.983 ± 0.002 \\
\textbullet &  &  &  & \textbullet & \textbullet & 0.969 ± 0.004 & 0.761 ± 0.014 & 0.755 ± 0.014 & 0.801 ± 0.024 & 0.761 ± 0.014 & 0.984 ± 0.001 \\
\textbullet &  &  & \textbullet &  &  & 0.984 ± 0.006 & 0.818 ± 0.032 & 0.818 ± 0.028 & 0.855 ± 0.023 & 0.818 ± 0.032 & 0.988 ± 0.002 \\
\textbullet &  &  & \textbullet &  & \textbullet & 0.984 ± 0.005 & 0.820 ± 0.013 & 0.825 ± 0.016 & 0.853 ± 0.027 & 0.820 ± 0.013 & 0.989 ± 0.001 \\
\textbullet &  &  & \textbullet & \textbullet &  & 0.984 ± 0.006 & 0.806 ± 0.012 & 0.806 ± 0.012 & 0.826 ± 0.012 & 0.806 ± 0.012 & 0.987 ± 0.001 \\
\textbullet &  &  & \textbullet & \textbullet & \textbullet & 0.988 ± 0.002 & 0.811 ± 0.036 & 0.808 ± 0.036 & 0.830 ± 0.040 & 0.811 ± 0.036 & 0.988 ± 0.002 \\
\textbullet &  & \textbullet &  &  &  & 0.946 ± 0.010 & 0.691 ± 0.031 & 0.659 ± 0.025 & 0.650 ± 0.029 & 0.691 ± 0.031 & 0.979 ± 0.002 \\
\textbullet &  & \textbullet &  &  & \textbullet & 0.955 ± 0.005 & 0.695 ± 0.025 & 0.670 ± 0.026 & 0.672 ± 0.039 & 0.695 ± 0.025 & 0.979 ± 0.001 \\
\textbullet &  & \textbullet &  & \textbullet &  & 0.956 ± 0.007 & 0.709 ± 0.014 & 0.685 ± 0.024 & 0.681 ± 0.033 & 0.709 ± 0.014 & 0.980 ± 0.001 \\
\textbullet &  & \textbullet &  & \textbullet & \textbullet & 0.958 ± 0.005 & 0.707 ± 0.021 & 0.682 ± 0.030 & 0.680 ± 0.038 & 0.707 ± 0.021 & 0.980 ± 0.002 \\
\textbullet &  & \textbullet & \textbullet &  &  & 0.977 ± 0.005 & 0.751 ± 0.029 & 0.736 ± 0.033 & 0.761 ± 0.034 & 0.751 ± 0.029 & 0.983 ± 0.002 \\
\textbullet &  & \textbullet & \textbullet &  & \textbullet & 0.972 ± 0.006 & 0.765 ± 0.037 & 0.749 ± 0.038 & 0.769 ± 0.040 & 0.765 ± 0.037 & 0.985 ± 0.002 \\
\textbullet &  & \textbullet & \textbullet & \textbullet &  & 0.975 ± 0.007 & 0.774 ± 0.038 & 0.757 ± 0.041 & 0.776 ± 0.031 & 0.774 ± 0.038 & 0.985 ± 0.002 \\
\textbullet &  & \textbullet & \textbullet & \textbullet & \textbullet & 0.972 ± 0.003 & 0.767 ± 0.023 & 0.755 ± 0.029 & 0.787 ± 0.046 & 0.767 ± 0.023 & 0.985 ± 0.002 \\
\textbullet & \textbullet &  &  &  &  & 0.964 ± 0.010 & 0.764 ± 0.020 & 0.761 ± 0.020 & 0.794 ± 0.026 & 0.764 ± 0.020 & 0.984 ± 0.001 \\
\textbullet & \textbullet &  &  &  & \textbullet & 0.973 ± 0.003 & 0.773 ± 0.011 & 0.771 ± 0.012 & 0.796 ± 0.010 & 0.773 ± 0.011 & 0.985 ± 0.001 \\
\textbullet & \textbullet &  &  & \textbullet &  & 0.967 ± 0.007 & 0.783 ± 0.017 & 0.776 ± 0.020 & 0.800 ± 0.034 & 0.783 ± 0.017 & 0.985 ± 0.002 \\
\textbullet & \textbullet &  &  & \textbullet & \textbullet & 0.965 ± 0.005 & 0.759 ± 0.023 & 0.754 ± 0.019 & 0.776 ± 0.029 & 0.759 ± 0.023 & 0.984 ± 0.001 \\
\textbullet & \textbullet &  & \textbullet &  &  & 0.986 ± 0.004 & 0.800 ± 0.037 & 0.800 ± 0.036 & 0.824 ± 0.030 & 0.800 ± 0.037 & 0.987 ± 0.002 \\
\textbullet & \textbullet &  & \textbullet &  & \textbullet & 0.987 ± 0.005 & 0.817 ± 0.020 & 0.813 ± 0.021 & 0.834 ± 0.027 & 0.817 ± 0.020 & 0.988 ± 0.001 \\
\textbullet & \textbullet &  & \textbullet & \textbullet &  & 0.985 ± 0.002 & 0.795 ± 0.016 & 0.791 ± 0.017 & 0.814 ± 0.015 & 0.795 ± 0.016 & 0.987 ± 0.001 \\
\textbullet & \textbullet &  & \textbullet & \textbullet & \textbullet & 0.987 ± 0.002 & 0.814 ± 0.021 & 0.812 ± 0.022 & 0.834 ± 0.022 & 0.814 ± 0.021 & 0.988 ± 0.002 \\
\textbullet & \textbullet & \textbullet &  &  &  & 0.950 ± 0.013 & 0.668 ± 0.041 & 0.637 ± 0.048 & 0.646 ± 0.054 & 0.668 ± 0.041 & 0.977 ± 0.004 \\
\textbullet & \textbullet & \textbullet &  &  & \textbullet & 0.952 ± 0.010 & 0.662 ± 0.027 & 0.631 ± 0.038 & 0.635 ± 0.028 & 0.662 ± 0.027 & 0.977 ± 0.003 \\
\textbullet & \textbullet & \textbullet &  & \textbullet &  & 0.956 ± 0.006 & 0.701 ± 0.028 & 0.680 ± 0.031 & 0.675 ± 0.037 & 0.701 ± 0.028 & 0.980 ± 0.002 \\
\textbullet & \textbullet & \textbullet &  & \textbullet & \textbullet & 0.960 ± 0.006 & 0.719 ± 0.024 & 0.686 ± 0.028 & 0.668 ± 0.033 & 0.719 ± 0.024 & 0.981 ± 0.002 \\
\textbullet & \textbullet & \textbullet & \textbullet &  &  & 0.977 ± 0.003 & 0.764 ± 0.026 & 0.753 ± 0.030 & 0.764 ± 0.038 & 0.764 ± 0.026 & 0.985 ± 0.002 \\
\textbullet & \textbullet & \textbullet & \textbullet &  & \textbullet & 0.975 ± 0.007 & 0.786 ± 0.029 & 0.765 ± 0.033 & 0.789 ± 0.035 & 0.786 ± 0.029 & 0.986 ± 0.002 \\
\textbullet & \textbullet & \textbullet & \textbullet & \textbullet &  & 0.977 ± 0.001 & 0.770 ± 0.016 & 0.763 ± 0.017 & 0.779 ± 0.029 & 0.770 ± 0.016 & 0.985 ± 0.001 \\
\textbullet & \textbullet & \textbullet & \textbullet & \textbullet & \textbullet & 0.976 ± 0.003 & 0.774 ± 0.036 & 0.752 ± 0.039 & 0.765 ± 0.036 & 0.774 ± 0.036 & 0.985 ± 0.002 \\
\end{longtable}

          \begin{longtable}{ccccccrrrrrr}
\caption{Classification metrics for AttOmics IF with different omics combination on CCLE dataset.}\label{tab:perf_comb_AttOmics IF_CCLE} \\
\toprule
mRNA & miRNA & CNV & DNAm & Protein & Metabolomics & AUROC & Accuracy & F1 & Precision & Recall & Specificity \\
\midrule
\endfirsthead
\caption[]{Classification metrics for AttOmics IF with different omics combination on CCLE dataset.} \\
\toprule
mRNA & miRNA & CNV & DNAm & Protein & Metabolomics & AUROC & Accuracy & F1 & Precision & Recall & Specificity \\
\midrule
\endhead
\midrule
\multicolumn{12}{r}{Continued on next page} \\
\midrule
\endfoot
\bottomrule
\endlastfoot
 &  &  &  & \textbullet & \textbullet & 0.956 ± 0.011 & 0.714 ± 0.046 & 0.701 ± 0.054 & 0.716 ± 0.070 & 0.714 ± 0.046 & 0.980 ± 0.003 \\
 &  &  & \textbullet &  & \textbullet & 0.984 ± 0.004 & 0.813 ± 0.022 & 0.804 ± 0.025 & 0.819 ± 0.030 & 0.813 ± 0.022 & 0.988 ± 0.002 \\
 &  &  & \textbullet & \textbullet &  & 0.984 ± 0.004 & 0.826 ± 0.030 & 0.820 ± 0.034 & 0.847 ± 0.032 & 0.826 ± 0.030 & 0.989 ± 0.002 \\
 &  &  & \textbullet & \textbullet & \textbullet & 0.985 ± 0.008 & 0.823 ± 0.031 & 0.817 ± 0.034 & 0.840 ± 0.033 & 0.823 ± 0.031 & 0.989 ± 0.002 \\
 &  & \textbullet &  &  & \textbullet & 0.929 ± 0.017 & 0.621 ± 0.026 & 0.579 ± 0.033 & 0.582 ± 0.039 & 0.621 ± 0.026 & 0.974 ± 0.003 \\
 &  & \textbullet &  & \textbullet &  & 0.955 ± 0.013 & 0.698 ± 0.023 & 0.667 ± 0.035 & 0.693 ± 0.052 & 0.698 ± 0.023 & 0.979 ± 0.002 \\
 &  & \textbullet &  & \textbullet & \textbullet & 0.952 ± 0.007 & 0.721 ± 0.037 & 0.686 ± 0.045 & 0.692 ± 0.032 & 0.721 ± 0.037 & 0.981 ± 0.003 \\
 &  & \textbullet & \textbullet &  &  & 0.985 ± 0.007 & 0.844 ± 0.026 & 0.836 ± 0.031 & 0.865 ± 0.027 & 0.844 ± 0.026 & 0.990 ± 0.002 \\
 &  & \textbullet & \textbullet &  & \textbullet & 0.988 ± 0.005 & 0.847 ± 0.038 & 0.829 ± 0.049 & 0.848 ± 0.051 & 0.847 ± 0.038 & 0.990 ± 0.003 \\
 &  & \textbullet & \textbullet & \textbullet &  & 0.986 ± 0.006 & 0.851 ± 0.018 & 0.839 ± 0.025 & 0.859 ± 0.040 & 0.851 ± 0.018 & 0.991 ± 0.001 \\
 &  & \textbullet & \textbullet & \textbullet & \textbullet & 0.986 ± 0.004 & 0.854 ± 0.034 & 0.844 ± 0.038 & 0.867 ± 0.040 & 0.854 ± 0.034 & 0.991 ± 0.002 \\
 & \textbullet &  &  &  & \textbullet & 0.939 ± 0.007 & 0.700 ± 0.025 & 0.688 ± 0.050 & 0.720 ± 0.079 & 0.700 ± 0.025 & 0.979 ± 0.002 \\
 & \textbullet &  &  & \textbullet &  & 0.949 ± 0.009 & 0.705 ± 0.028 & 0.686 ± 0.032 & 0.695 ± 0.038 & 0.705 ± 0.028 & 0.980 ± 0.002 \\
 & \textbullet &  &  & \textbullet & \textbullet & 0.957 ± 0.013 & 0.715 ± 0.016 & 0.706 ± 0.011 & 0.726 ± 0.029 & 0.715 ± 0.016 & 0.980 ± 0.001 \\
 & \textbullet &  & \textbullet &  &  & 0.985 ± 0.004 & 0.822 ± 0.012 & 0.814 ± 0.016 & 0.839 ± 0.023 & 0.822 ± 0.012 & 0.989 ± 0.000 \\
 & \textbullet &  & \textbullet &  & \textbullet & 0.986 ± 0.003 & 0.836 ± 0.016 & 0.833 ± 0.019 & 0.856 ± 0.022 & 0.836 ± 0.016 & 0.990 ± 0.001 \\
 & \textbullet &  & \textbullet & \textbullet &  & 0.983 ± 0.007 & 0.825 ± 0.013 & 0.820 ± 0.014 & 0.843 ± 0.011 & 0.825 ± 0.013 & 0.989 ± 0.001 \\
 & \textbullet &  & \textbullet & \textbullet & \textbullet & 0.984 ± 0.008 & 0.811 ± 0.035 & 0.798 ± 0.035 & 0.826 ± 0.043 & 0.811 ± 0.035 & 0.988 ± 0.002 \\
 & \textbullet & \textbullet &  &  &  & 0.952 ± 0.013 & 0.707 ± 0.049 & 0.677 ± 0.061 & 0.680 ± 0.066 & 0.707 ± 0.049 & 0.980 ± 0.004 \\
 & \textbullet & \textbullet &  &  & \textbullet & 0.957 ± 0.013 & 0.741 ± 0.035 & 0.707 ± 0.035 & 0.703 ± 0.040 & 0.741 ± 0.035 & 0.982 ± 0.002 \\
 & \textbullet & \textbullet &  & \textbullet &  & 0.962 ± 0.011 & 0.751 ± 0.028 & 0.732 ± 0.031 & 0.751 ± 0.055 & 0.751 ± 0.028 & 0.984 ± 0.002 \\
 & \textbullet & \textbullet &  & \textbullet & \textbullet & 0.967 ± 0.006 & 0.753 ± 0.017 & 0.727 ± 0.013 & 0.716 ± 0.013 & 0.753 ± 0.017 & 0.983 ± 0.001 \\
 & \textbullet & \textbullet & \textbullet &  &  & 0.989 ± 0.003 & 0.854 ± 0.027 & 0.844 ± 0.034 & 0.875 ± 0.028 & 0.854 ± 0.027 & 0.991 ± 0.002 \\
 & \textbullet & \textbullet & \textbullet &  & \textbullet & 0.987 ± 0.009 & 0.849 ± 0.030 & 0.842 ± 0.031 & 0.870 ± 0.030 & 0.849 ± 0.030 & 0.991 ± 0.002 \\
 & \textbullet & \textbullet & \textbullet & \textbullet &  & 0.990 ± 0.004 & 0.847 ± 0.029 & 0.834 ± 0.033 & 0.873 ± 0.024 & 0.847 ± 0.029 & 0.991 ± 0.002 \\
 & \textbullet & \textbullet & \textbullet & \textbullet & \textbullet & 0.988 ± 0.004 & 0.866 ± 0.013 & 0.856 ± 0.019 & 0.885 ± 0.015 & 0.866 ± 0.013 & 0.992 ± 0.001 \\
\textbullet &  &  &  &  & \textbullet & 0.977 ± 0.009 & 0.785 ± 0.021 & 0.783 ± 0.023 & 0.814 ± 0.032 & 0.785 ± 0.021 & 0.985 ± 0.002 \\
\textbullet &  &  &  & \textbullet &  & 0.975 ± 0.006 & 0.779 ± 0.032 & 0.782 ± 0.037 & 0.813 ± 0.045 & 0.779 ± 0.032 & 0.985 ± 0.002 \\
\textbullet &  &  &  & \textbullet & \textbullet & 0.981 ± 0.004 & 0.808 ± 0.031 & 0.810 ± 0.030 & 0.839 ± 0.036 & 0.808 ± 0.031 & 0.987 ± 0.002 \\
\textbullet &  &  & \textbullet &  &  & 0.992 ± 0.004 & 0.868 ± 0.024 & 0.859 ± 0.029 & 0.888 ± 0.028 & 0.868 ± 0.024 & 0.991 ± 0.002 \\
\textbullet &  &  & \textbullet &  & \textbullet & 0.991 ± 0.005 & 0.870 ± 0.036 & 0.870 ± 0.038 & 0.892 ± 0.046 & 0.870 ± 0.036 & 0.992 ± 0.003 \\
\textbullet &  &  & \textbullet & \textbullet &  & 0.993 ± 0.007 & 0.883 ± 0.038 & 0.880 ± 0.042 & 0.891 ± 0.044 & 0.883 ± 0.038 & 0.992 ± 0.003 \\
\textbullet &  &  & \textbullet & \textbullet & \textbullet & 0.991 ± 0.005 & 0.853 ± 0.030 & 0.849 ± 0.032 & 0.876 ± 0.028 & 0.853 ± 0.030 & 0.991 ± 0.002 \\
\textbullet &  & \textbullet &  &  &  & 0.980 ± 0.006 & 0.809 ± 0.033 & 0.796 ± 0.036 & 0.816 ± 0.058 & 0.809 ± 0.033 & 0.987 ± 0.002 \\
\textbullet &  & \textbullet &  &  & \textbullet & 0.981 ± 0.008 & 0.821 ± 0.009 & 0.810 ± 0.012 & 0.826 ± 0.024 & 0.821 ± 0.009 & 0.988 ± 0.001 \\
\textbullet &  & \textbullet &  & \textbullet &  & 0.980 ± 0.006 & 0.819 ± 0.021 & 0.807 ± 0.023 & 0.830 ± 0.039 & 0.819 ± 0.021 & 0.988 ± 0.001 \\
\textbullet &  & \textbullet &  & \textbullet & \textbullet & 0.981 ± 0.011 & 0.801 ± 0.066 & 0.785 ± 0.089 & 0.809 ± 0.101 & 0.801 ± 0.066 & 0.987 ± 0.005 \\
\textbullet &  & \textbullet & \textbullet &  &  & 0.991 ± 0.004 & 0.858 ± 0.016 & 0.855 ± 0.018 & 0.889 ± 0.027 & 0.858 ± 0.016 & 0.991 ± 0.001 \\
\textbullet &  & \textbullet & \textbullet &  & \textbullet & 0.990 ± 0.003 & 0.862 ± 0.026 & 0.855 ± 0.031 & 0.880 ± 0.026 & 0.862 ± 0.026 & 0.991 ± 0.002 \\
\textbullet &  & \textbullet & \textbullet & \textbullet &  & 0.994 ± 0.003 & 0.883 ± 0.015 & 0.877 ± 0.021 & 0.899 ± 0.020 & 0.883 ± 0.015 & 0.993 ± 0.001 \\
\textbullet &  & \textbullet & \textbullet & \textbullet & \textbullet & 0.995 ± 0.004 & 0.888 ± 0.031 & 0.880 ± 0.034 & 0.899 ± 0.030 & 0.888 ± 0.031 & 0.993 ± 0.002 \\
\textbullet & \textbullet &  &  &  &  & 0.977 ± 0.006 & 0.781 ± 0.031 & 0.783 ± 0.030 & 0.816 ± 0.038 & 0.781 ± 0.031 & 0.986 ± 0.002 \\
\textbullet & \textbullet &  &  &  & \textbullet & 0.976 ± 0.008 & 0.768 ± 0.020 & 0.763 ± 0.022 & 0.792 ± 0.034 & 0.768 ± 0.020 & 0.985 ± 0.001 \\
\textbullet & \textbullet &  &  & \textbullet &  & 0.978 ± 0.006 & 0.796 ± 0.031 & 0.791 ± 0.037 & 0.806 ± 0.040 & 0.796 ± 0.031 & 0.987 ± 0.002 \\
\textbullet & \textbullet &  &  & \textbullet & \textbullet & 0.983 ± 0.003 & 0.811 ± 0.035 & 0.810 ± 0.038 & 0.847 ± 0.034 & 0.811 ± 0.035 & 0.987 ± 0.003 \\
\textbullet & \textbullet &  & \textbullet &  &  & 0.992 ± 0.004 & 0.859 ± 0.015 & 0.853 ± 0.019 & 0.877 ± 0.016 & 0.859 ± 0.015 & 0.991 ± 0.001 \\
\textbullet & \textbullet &  & \textbullet &  & \textbullet & 0.991 ± 0.002 & 0.865 ± 0.008 & 0.862 ± 0.009 & 0.883 ± 0.018 & 0.865 ± 0.008 & 0.991 ± 0.001 \\
\textbullet & \textbullet &  & \textbullet & \textbullet &  & 0.991 ± 0.003 & 0.869 ± 0.033 & 0.865 ± 0.038 & 0.888 ± 0.027 & 0.869 ± 0.033 & 0.992 ± 0.002 \\
\textbullet & \textbullet &  & \textbullet & \textbullet & \textbullet & 0.993 ± 0.004 & 0.879 ± 0.022 & 0.874 ± 0.028 & 0.896 ± 0.027 & 0.879 ± 0.022 & 0.992 ± 0.002 \\
\textbullet & \textbullet & \textbullet &  &  &  & 0.974 ± 0.006 & 0.804 ± 0.037 & 0.794 ± 0.041 & 0.818 ± 0.053 & 0.804 ± 0.037 & 0.987 ± 0.002 \\
\textbullet & \textbullet & \textbullet &  &  & \textbullet & 0.982 ± 0.005 & 0.803 ± 0.045 & 0.798 ± 0.050 & 0.827 ± 0.059 & 0.803 ± 0.045 & 0.987 ± 0.003 \\
\textbullet & \textbullet & \textbullet &  & \textbullet &  & 0.987 ± 0.004 & 0.843 ± 0.028 & 0.840 ± 0.030 & 0.876 ± 0.020 & 0.843 ± 0.028 & 0.990 ± 0.002 \\
\textbullet & \textbullet & \textbullet &  & \textbullet & \textbullet & 0.981 ± 0.003 & 0.823 ± 0.029 & 0.812 ± 0.031 & 0.832 ± 0.038 & 0.823 ± 0.029 & 0.988 ± 0.002 \\
\textbullet & \textbullet & \textbullet & \textbullet &  &  & 0.991 ± 0.003 & 0.875 ± 0.023 & 0.870 ± 0.026 & 0.895 ± 0.028 & 0.875 ± 0.023 & 0.992 ± 0.002 \\
\textbullet & \textbullet & \textbullet & \textbullet &  & \textbullet & 0.990 ± 0.006 & 0.853 ± 0.019 & 0.847 ± 0.025 & 0.870 ± 0.021 & 0.853 ± 0.019 & 0.991 ± 0.001 \\
\textbullet & \textbullet & \textbullet & \textbullet & \textbullet &  & 0.994 ± 0.004 & 0.882 ± 0.040 & 0.875 ± 0.043 & 0.898 ± 0.039 & 0.882 ± 0.040 & 0.992 ± 0.002 \\
\textbullet & \textbullet & \textbullet & \textbullet & \textbullet & \textbullet & 0.990 ± 0.006 & 0.865 ± 0.019 & 0.856 ± 0.020 & 0.880 ± 0.010 & 0.865 ± 0.019 & 0.991 ± 0.001 \\
\end{longtable}

          \begin{longtable}{ccccccrrrrrr}
\caption{Classification metrics for CrossAttOmics with different omics combination on CCLE dataset.} \label{tab:perf_comb_CrossAttOmics_CCLE} \\
\toprule
mRNA & miRNA & CNV & DNAm & Protein & Metabolomics & AUROC & Accuracy & F1 & Precision & Recall & Specificity \\
\midrule
\endfirsthead
\caption[]{Classification metrics for CrossAttOmics with different omics combination on CCLE dataset.} \\
\toprule
mRNA & miRNA & CNV & DNAm & Protein & Metabolomics & AUROC & Accuracy & F1 & Precision & Recall & Specificity \\
\midrule
\endhead
\midrule
\multicolumn{12}{r}{Continued on next page} \\
\midrule
\endfoot
\bottomrule
\endlastfoot
 & \textbullet &  & \textbullet &  &  & 0.982 ± 0.007 & 0.790 ± 0.022 & 0.780 ± 0.026 & 0.801 ± 0.030 & 0.790 ± 0.022 & 0.987 ± 0.002 \\
 & \textbullet &  & \textbullet &  & \textbullet & 0.975 ± 0.014 & 0.809 ± 0.032 & 0.801 ± 0.031 & 0.810 ± 0.032 & 0.809 ± 0.032 & 0.988 ± 0.003 \\
 & \textbullet &  & \textbullet & \textbullet &  & 0.978 ± 0.008 & 0.780 ± 0.037 & 0.765 ± 0.045 & 0.787 ± 0.058 & 0.780 ± 0.037 & 0.986 ± 0.003 \\
 & \textbullet &  & \textbullet & \textbullet & \textbullet & 0.975 ± 0.004 & 0.789 ± 0.018 & 0.781 ± 0.014 & 0.801 ± 0.010 & 0.789 ± 0.018 & 0.987 ± 0.002 \\
 & \textbullet & \textbullet &  &  &  & 0.924 ± 0.007 & 0.656 ± 0.032 & 0.612 ± 0.034 & 0.618 ± 0.059 & 0.656 ± 0.032 & 0.976 ± 0.002 \\
 & \textbullet & \textbullet &  &  & \textbullet & 0.934 ± 0.013 & 0.669 ± 0.059 & 0.635 ± 0.061 & 0.645 ± 0.066 & 0.669 ± 0.059 & 0.977 ± 0.004 \\
 & \textbullet & \textbullet &  & \textbullet &  & 0.953 ± 0.008 & 0.716 ± 0.023 & 0.688 ± 0.031 & 0.697 ± 0.033 & 0.716 ± 0.023 & 0.980 ± 0.001 \\
 & \textbullet & \textbullet &  & \textbullet & \textbullet & 0.953 ± 0.009 & 0.722 ± 0.044 & 0.688 ± 0.044 & 0.683 ± 0.048 & 0.722 ± 0.044 & 0.981 ± 0.003 \\
 & \textbullet & \textbullet & \textbullet &  &  & 0.983 ± 0.005 & 0.833 ± 0.022 & 0.822 ± 0.030 & 0.848 ± 0.030 & 0.833 ± 0.022 & 0.989 ± 0.002 \\
 & \textbullet & \textbullet & \textbullet &  & \textbullet & 0.978 ± 0.005 & 0.831 ± 0.017 & 0.825 ± 0.020 & 0.845 ± 0.015 & 0.831 ± 0.017 & 0.989 ± 0.001 \\
 & \textbullet & \textbullet & \textbullet & \textbullet &  & 0.984 ± 0.005 & 0.850 ± 0.029 & 0.841 ± 0.030 & 0.873 ± 0.024 & 0.850 ± 0.029 & 0.991 ± 0.002 \\
 & \textbullet & \textbullet & \textbullet & \textbullet & \textbullet & 0.981 ± 0.006 & 0.834 ± 0.013 & 0.825 ± 0.016 & 0.851 ± 0.019 & 0.834 ± 0.013 & 0.989 ± 0.001 \\
\textbullet &  &  &  &  & \textbullet & 0.965 ± 0.007 & 0.767 ± 0.031 & 0.755 ± 0.032 & 0.776 ± 0.048 & 0.767 ± 0.031 & 0.984 ± 0.002 \\
\textbullet &  &  &  & \textbullet &  & 0.966 ± 0.011 & 0.749 ± 0.018 & 0.737 ± 0.022 & 0.759 ± 0.023 & 0.749 ± 0.018 & 0.983 ± 0.002 \\
\textbullet &  &  &  & \textbullet & \textbullet & 0.967 ± 0.005 & 0.763 ± 0.055 & 0.746 ± 0.062 & 0.771 ± 0.055 & 0.763 ± 0.055 & 0.983 ± 0.004 \\
\textbullet &  &  & \textbullet &  &  & 0.985 ± 0.006 & 0.834 ± 0.037 & 0.829 ± 0.043 & 0.862 ± 0.046 & 0.834 ± 0.037 & 0.989 ± 0.003 \\
\textbullet &  &  & \textbullet &  & \textbullet & 0.981 ± 0.007 & 0.849 ± 0.022 & 0.843 ± 0.032 & 0.865 ± 0.025 & 0.849 ± 0.022 & 0.990 ± 0.002 \\
\textbullet &  &  & \textbullet & \textbullet &  & 0.989 ± 0.003 & 0.842 ± 0.028 & 0.831 ± 0.031 & 0.855 ± 0.035 & 0.842 ± 0.028 & 0.990 ± 0.001 \\
\textbullet &  &  & \textbullet & \textbullet & \textbullet & 0.985 ± 0.005 & 0.839 ± 0.028 & 0.831 ± 0.033 & 0.855 ± 0.039 & 0.839 ± 0.028 & 0.989 ± 0.002 \\
\textbullet &  & \textbullet &  &  &  & 0.967 ± 0.015 & 0.799 ± 0.025 & 0.782 ± 0.032 & 0.798 ± 0.043 & 0.799 ± 0.025 & 0.986 ± 0.002 \\
\textbullet &  & \textbullet &  &  & \textbullet & 0.973 ± 0.009 & 0.770 ± 0.039 & 0.756 ± 0.046 & 0.778 ± 0.038 & 0.770 ± 0.039 & 0.984 ± 0.003 \\
\textbullet &  & \textbullet &  & \textbullet &  & 0.973 ± 0.004 & 0.786 ± 0.025 & 0.768 ± 0.034 & 0.783 ± 0.047 & 0.786 ± 0.025 & 0.985 ± 0.001 \\
\textbullet &  & \textbullet &  & \textbullet & \textbullet & 0.966 ± 0.007 & 0.776 ± 0.029 & 0.743 ± 0.037 & 0.766 ± 0.042 & 0.776 ± 0.029 & 0.984 ± 0.002 \\
\textbullet &  & \textbullet & \textbullet &  &  & 0.983 ± 0.007 & 0.828 ± 0.029 & 0.810 ± 0.032 & 0.834 ± 0.037 & 0.828 ± 0.029 & 0.988 ± 0.002 \\
\textbullet &  & \textbullet & \textbullet &  & \textbullet & 0.985 ± 0.013 & 0.841 ± 0.017 & 0.832 ± 0.025 & 0.845 ± 0.038 & 0.841 ± 0.017 & 0.989 ± 0.001 \\
\textbullet &  & \textbullet & \textbullet & \textbullet &  & 0.986 ± 0.005 & 0.846 ± 0.036 & 0.837 ± 0.045 & 0.856 ± 0.063 & 0.846 ± 0.036 & 0.990 ± 0.002 \\
\textbullet &  & \textbullet & \textbullet & \textbullet & \textbullet & 0.988 ± 0.006 & 0.847 ± 0.017 & 0.838 ± 0.023 & 0.845 ± 0.038 & 0.847 ± 0.017 & 0.990 ± 0.002 \\
\textbullet & \textbullet &  &  &  &  & 0.966 ± 0.007 & 0.764 ± 0.024 & 0.748 ± 0.025 & 0.759 ± 0.030 & 0.764 ± 0.024 & 0.983 ± 0.002 \\
\textbullet & \textbullet &  &  &  & \textbullet & 0.972 ± 0.002 & 0.781 ± 0.029 & 0.773 ± 0.031 & 0.793 ± 0.035 & 0.781 ± 0.029 & 0.985 ± 0.001 \\
\textbullet & \textbullet &  &  & \textbullet &  & 0.968 ± 0.003 & 0.770 ± 0.011 & 0.759 ± 0.013 & 0.785 ± 0.012 & 0.770 ± 0.011 & 0.984 ± 0.001 \\
\textbullet & \textbullet &  &  & \textbullet & \textbullet & 0.968 ± 0.005 & 0.795 ± 0.017 & 0.785 ± 0.018 & 0.802 ± 0.022 & 0.795 ± 0.017 & 0.985 ± 0.001 \\
\textbullet & \textbullet &  & \textbullet &  &  & 0.987 ± 0.003 & 0.842 ± 0.025 & 0.835 ± 0.028 & 0.856 ± 0.033 & 0.842 ± 0.025 & 0.989 ± 0.002 \\
\textbullet & \textbullet &  & \textbullet &  & \textbullet & 0.988 ± 0.003 & 0.834 ± 0.018 & 0.832 ± 0.016 & 0.862 ± 0.013 & 0.834 ± 0.018 & 0.989 ± 0.001 \\
\textbullet & \textbullet &  & \textbullet & \textbullet &  & 0.990 ± 0.002 & 0.844 ± 0.021 & 0.837 ± 0.025 & 0.858 ± 0.031 & 0.844 ± 0.021 & 0.990 ± 0.002 \\
\textbullet & \textbullet &  & \textbullet & \textbullet & \textbullet & 0.987 ± 0.002 & 0.878 ± 0.030 & 0.873 ± 0.028 & 0.893 ± 0.028 & 0.878 ± 0.030 & 0.992 ± 0.002 \\
\textbullet & \textbullet & \textbullet &  &  &  & 0.974 ± 0.005 & 0.821 ± 0.043 & 0.808 ± 0.046 & 0.827 ± 0.044 & 0.821 ± 0.043 & 0.987 ± 0.002 \\
\textbullet & \textbullet & \textbullet &  &  & \textbullet & 0.969 ± 0.015 & 0.787 ± 0.027 & 0.767 ± 0.050 & 0.788 ± 0.055 & 0.787 ± 0.027 & 0.985 ± 0.003 \\
\textbullet & \textbullet & \textbullet &  & \textbullet &  & 0.975 ± 0.011 & 0.781 ± 0.046 & 0.768 ± 0.048 & 0.807 ± 0.037 & 0.781 ± 0.046 & 0.985 ± 0.003 \\
\textbullet & \textbullet & \textbullet &  & \textbullet & \textbullet & 0.969 ± 0.005 & 0.767 ± 0.015 & 0.755 ± 0.019 & 0.777 ± 0.045 & 0.767 ± 0.015 & 0.985 ± 0.001 \\
\textbullet & \textbullet & \textbullet & \textbullet &  &  & 0.989 ± 0.002 & 0.857 ± 0.008 & 0.851 ± 0.008 & 0.875 ± 0.014 & 0.857 ± 0.008 & 0.991 ± 0.001 \\
\textbullet & \textbullet & \textbullet & \textbullet &  & \textbullet & 0.991 ± 0.002 & 0.854 ± 0.034 & 0.842 ± 0.042 & 0.865 ± 0.038 & 0.854 ± 0.034 & 0.990 ± 0.002 \\
\textbullet & \textbullet & \textbullet & \textbullet & \textbullet &  & 0.987 ± 0.006 & 0.866 ± 0.020 & 0.860 ± 0.019 & 0.883 ± 0.020 & 0.866 ± 0.020 & 0.991 ± 0.001 \\
\textbullet & \textbullet & \textbullet & \textbullet & \textbullet & \textbullet & 0.990 ± 0.002 & 0.866 ± 0.029 & 0.863 ± 0.033 & 0.891 ± 0.026 & 0.866 ± 0.029 & 0.991 ± 0.002 \\
\end{longtable}

          \begin{longtable}{ccccccrrrrrr}
\caption{Classification metrics for MLP with different omics combination on CCLE dataset.}\label{tab:perf_comb_MLP_CCLE} \\
\toprule
mRNA & miRNA & CNV & DNAm & Protein & Metabolomics & AUROC & Accuracy & F1 & Precision & Recall & Specificity \\
\midrule
\endfirsthead
\caption[]{Classification metrics for MLP with different omics combination on CCLE dataset.} \\
\toprule
mRNA & miRNA & CNV & DNAm & Protein & Metabolomics & AUROC & Accuracy & F1 & Precision & Recall & Specificity \\
\midrule
\endhead
\midrule
\multicolumn{12}{r}{Continued on next page} \\
\midrule
\endfoot
\bottomrule
\endlastfoot
 &  &  &  &  & \textbullet & 0.910 ± 0.015 & 0.570 ± 0.055 & 0.541 ± 0.066 & 0.565 ± 0.073 & 0.570 ± 0.055 & 0.970 ± 0.004 \\
 &  &  &  & \textbullet &  & 0.951 ± 0.008 & 0.715 ± 0.025 & 0.698 ± 0.029 & 0.698 ± 0.027 & 0.715 ± 0.025 & 0.980 ± 0.002 \\
 &  &  & \textbullet &  &  & 0.982 ± 0.003 & 0.791 ± 0.014 & 0.784 ± 0.016 & 0.818 ± 0.013 & 0.791 ± 0.014 & 0.987 ± 0.001 \\
 &  & \textbullet &  &  &  & 0.891 ± 0.007 & 0.462 ± 0.023 & 0.447 ± 0.017 & 0.495 ± 0.023 & 0.462 ± 0.023 & 0.964 ± 0.001 \\
 & \textbullet &  &  &  &  & 0.934 ± 0.006 & 0.731 ± 0.023 & 0.723 ± 0.036 & 0.736 ± 0.045 & 0.731 ± 0.023 & 0.981 ± 0.002 \\
\textbullet &  &  &  &  &  & 0.913 ± 0.020 & 0.711 ± 0.025 & 0.705 ± 0.017 & 0.727 ± 0.021 & 0.711 ± 0.025 & 0.980 ± 0.002 \\
\end{longtable}

          \begin{longtable}{ccccccrrrrrr}
\caption{Classification metrics for MLP EF with different omics combination on CCLE dataset.}\label{tab:perf_comb_MLP EF_CCLE} \\
\toprule
mRNA & miRNA & CNV & DNAm & Protein & Metabolomics & AUROC & Accuracy & F1 & Precision & Recall & Specificity \\
\midrule
\endfirsthead
\caption[]{Classification metrics for MLP EF with different omics combination on CCLE dataset.} \\
\toprule
mRNA & miRNA & CNV & DNAm & Protein & Metabolomics & AUROC & Accuracy & F1 & Precision & Recall & Specificity \\
\midrule
\endhead
\midrule
\multicolumn{12}{r}{Continued on next page} \\
\midrule
\endfoot
\bottomrule
\endlastfoot
 &  &  & \textbullet &  & \textbullet & 0.978 ± 0.009 & 0.784 ± 0.021 & 0.773 ± 0.023 & 0.793 ± 0.022 & 0.784 ± 0.021 & 0.987 ± 0.001 \\
 &  &  & \textbullet & \textbullet &  & 0.980 ± 0.006 & 0.799 ± 0.008 & 0.792 ± 0.008 & 0.824 ± 0.014 & 0.799 ± 0.008 & 0.987 ± 0.001 \\
 &  &  & \textbullet & \textbullet & \textbullet & 0.984 ± 0.005 & 0.796 ± 0.018 & 0.788 ± 0.020 & 0.817 ± 0.023 & 0.796 ± 0.018 & 0.988 ± 0.001 \\
 &  & \textbullet &  &  & \textbullet & 0.890 ± 0.005 & 0.502 ± 0.036 & 0.487 ± 0.028 & 0.520 ± 0.024 & 0.502 ± 0.036 & 0.967 ± 0.002 \\
 &  & \textbullet &  & \textbullet &  & 0.904 ± 0.007 & 0.502 ± 0.034 & 0.480 ± 0.036 & 0.511 ± 0.046 & 0.502 ± 0.034 & 0.968 ± 0.002 \\
 &  & \textbullet &  & \textbullet & \textbullet & 0.903 ± 0.015 & 0.512 ± 0.033 & 0.489 ± 0.042 & 0.524 ± 0.035 & 0.512 ± 0.033 & 0.967 ± 0.003 \\
 &  & \textbullet & \textbullet &  &  & 0.974 ± 0.004 & 0.769 ± 0.018 & 0.749 ± 0.024 & 0.765 ± 0.042 & 0.769 ± 0.018 & 0.985 ± 0.001 \\
 &  & \textbullet & \textbullet &  & \textbullet & 0.979 ± 0.001 & 0.759 ± 0.021 & 0.736 ± 0.019 & 0.751 ± 0.037 & 0.759 ± 0.021 & 0.984 ± 0.001 \\
 &  & \textbullet & \textbullet & \textbullet &  & 0.974 ± 0.004 & 0.768 ± 0.028 & 0.759 ± 0.032 & 0.780 ± 0.037 & 0.768 ± 0.028 & 0.985 ± 0.002 \\
 &  & \textbullet & \textbullet & \textbullet & \textbullet & 0.977 ± 0.005 & 0.776 ± 0.015 & 0.761 ± 0.017 & 0.774 ± 0.034 & 0.776 ± 0.015 & 0.985 ± 0.001 \\
 & \textbullet &  &  &  & \textbullet & 0.946 ± 0.004 & 0.770 ± 0.013 & 0.768 ± 0.013 & 0.813 ± 0.009 & 0.770 ± 0.013 & 0.984 ± 0.001 \\
 & \textbullet &  &  & \textbullet &  & 0.956 ± 0.004 & 0.774 ± 0.010 & 0.766 ± 0.012 & 0.772 ± 0.017 & 0.774 ± 0.010 & 0.985 ± 0.001 \\
 & \textbullet &  &  & \textbullet & \textbullet & 0.961 ± 0.006 & 0.776 ± 0.031 & 0.758 ± 0.031 & 0.755 ± 0.030 & 0.776 ± 0.031 & 0.985 ± 0.002 \\
 & \textbullet &  & \textbullet &  &  & 0.982 ± 0.005 & 0.821 ± 0.018 & 0.814 ± 0.019 & 0.839 ± 0.029 & 0.821 ± 0.018 & 0.989 ± 0.001 \\
 & \textbullet &  & \textbullet &  & \textbullet & 0.981 ± 0.004 & 0.809 ± 0.024 & 0.800 ± 0.023 & 0.822 ± 0.024 & 0.809 ± 0.024 & 0.988 ± 0.002 \\
 & \textbullet &  & \textbullet & \textbullet &  & 0.983 ± 0.006 & 0.809 ± 0.031 & 0.803 ± 0.029 & 0.822 ± 0.032 & 0.809 ± 0.031 & 0.988 ± 0.002 \\
 & \textbullet &  & \textbullet & \textbullet & \textbullet & 0.980 ± 0.006 & 0.817 ± 0.010 & 0.808 ± 0.007 & 0.835 ± 0.017 & 0.817 ± 0.010 & 0.989 ± 0.001 \\
 & \textbullet & \textbullet &  &  &  & 0.887 ± 0.017 & 0.481 ± 0.047 & 0.465 ± 0.036 & 0.515 ± 0.038 & 0.481 ± 0.047 & 0.965 ± 0.003 \\
 & \textbullet & \textbullet &  &  & \textbullet & 0.889 ± 0.024 & 0.492 ± 0.041 & 0.470 ± 0.065 & 0.500 ± 0.099 & 0.492 ± 0.041 & 0.966 ± 0.004 \\
 & \textbullet & \textbullet &  & \textbullet &  & 0.906 ± 0.008 & 0.554 ± 0.032 & 0.525 ± 0.038 & 0.565 ± 0.037 & 0.554 ± 0.032 & 0.970 ± 0.003 \\
 & \textbullet & \textbullet &  & \textbullet & \textbullet & 0.910 ± 0.005 & 0.539 ± 0.013 & 0.526 ± 0.014 & 0.562 ± 0.026 & 0.539 ± 0.013 & 0.969 ± 0.001 \\
 & \textbullet & \textbullet & \textbullet &  &  & 0.977 ± 0.004 & 0.764 ± 0.017 & 0.747 ± 0.021 & 0.760 ± 0.042 & 0.764 ± 0.017 & 0.985 ± 0.001 \\
 & \textbullet & \textbullet & \textbullet &  & \textbullet & 0.978 ± 0.002 & 0.772 ± 0.023 & 0.751 ± 0.023 & 0.767 ± 0.035 & 0.772 ± 0.023 & 0.985 ± 0.001 \\
 & \textbullet & \textbullet & \textbullet & \textbullet &  & 0.979 ± 0.003 & 0.774 ± 0.022 & 0.758 ± 0.013 & 0.768 ± 0.001 & 0.774 ± 0.022 & 0.985 ± 0.002 \\
 & \textbullet & \textbullet & \textbullet & \textbullet & \textbullet & 0.979 ± 0.004 & 0.770 ± 0.017 & 0.755 ± 0.019 & 0.783 ± 0.043 & 0.770 ± 0.017 & 0.985 ± 0.001 \\
\textbullet &  &  &  &  & \textbullet & 0.914 ± 0.014 & 0.735 ± 0.020 & 0.728 ± 0.024 & 0.760 ± 0.030 & 0.735 ± 0.020 & 0.982 ± 0.002 \\
\textbullet &  &  &  & \textbullet &  & 0.905 ± 0.020 & 0.717 ± 0.008 & 0.716 ± 0.005 & 0.769 ± 0.028 & 0.717 ± 0.008 & 0.981 ± 0.001 \\
\textbullet &  &  &  & \textbullet & \textbullet & 0.905 ± 0.011 & 0.707 ± 0.026 & 0.703 ± 0.025 & 0.751 ± 0.017 & 0.707 ± 0.026 & 0.980 ± 0.001 \\
\textbullet &  &  & \textbullet &  &  & 0.937 ± 0.014 & 0.737 ± 0.030 & 0.733 ± 0.030 & 0.774 ± 0.033 & 0.737 ± 0.030 & 0.983 ± 0.002 \\
\textbullet &  &  & \textbullet &  & \textbullet & 0.925 ± 0.014 & 0.735 ± 0.036 & 0.725 ± 0.034 & 0.775 ± 0.042 & 0.735 ± 0.036 & 0.982 ± 0.002 \\
\textbullet &  &  & \textbullet & \textbullet &  & 0.931 ± 0.011 & 0.764 ± 0.019 & 0.753 ± 0.018 & 0.769 ± 0.012 & 0.764 ± 0.019 & 0.984 ± 0.002 \\
\textbullet &  &  & \textbullet & \textbullet & \textbullet & 0.946 ± 0.014 & 0.756 ± 0.019 & 0.741 ± 0.012 & 0.766 ± 0.018 & 0.756 ± 0.019 & 0.984 ± 0.001 \\
\textbullet &  & \textbullet &  &  &  & 0.899 ± 0.004 & 0.606 ± 0.013 & 0.567 ± 0.003 & 0.582 ± 0.051 & 0.606 ± 0.013 & 0.973 ± 0.000 \\
\textbullet &  & \textbullet &  &  & \textbullet & 0.895 ± 0.023 & 0.583 ± 0.039 & 0.556 ± 0.061 & 0.578 ± 0.088 & 0.583 ± 0.039 & 0.972 ± 0.003 \\
\textbullet &  & \textbullet &  & \textbullet &  & 0.893 ± 0.014 & 0.586 ± 0.074 & 0.548 ± 0.071 & 0.549 ± 0.061 & 0.586 ± 0.074 & 0.972 ± 0.005 \\
\textbullet &  & \textbullet &  & \textbullet & \textbullet & 0.886 ± 0.016 & 0.571 ± 0.027 & 0.538 ± 0.032 & 0.547 ± 0.045 & 0.571 ± 0.027 & 0.970 ± 0.002 \\
\textbullet &  & \textbullet & \textbullet &  &  & 0.945 ± 0.007 & 0.727 ± 0.036 & 0.713 ± 0.055 & 0.736 ± 0.059 & 0.727 ± 0.036 & 0.982 ± 0.003 \\
\textbullet &  & \textbullet & \textbullet &  & \textbullet & 0.956 ± 0.005 & 0.754 ± 0.010 & 0.747 ± 0.014 & 0.776 ± 0.043 & 0.754 ± 0.010 & 0.984 ± 0.001 \\
\textbullet &  & \textbullet & \textbullet & \textbullet &  & 0.939 ± 0.004 & 0.712 ± 0.030 & 0.704 ± 0.038 & 0.737 ± 0.058 & 0.712 ± 0.030 & 0.981 ± 0.003 \\
\textbullet &  & \textbullet & \textbullet & \textbullet & \textbullet & 0.937 ± 0.025 & 0.694 ± 0.036 & 0.677 ± 0.034 & 0.717 ± 0.029 & 0.694 ± 0.036 & 0.980 ± 0.003 \\
\textbullet & \textbullet &  &  &  &  & 0.911 ± 0.021 & 0.699 ± 0.017 & 0.687 ± 0.020 & 0.725 ± 0.035 & 0.699 ± 0.017 & 0.979 ± 0.001 \\
\textbullet & \textbullet &  &  &  & \textbullet & 0.912 ± 0.012 & 0.703 ± 0.037 & 0.687 ± 0.044 & 0.723 ± 0.068 & 0.703 ± 0.037 & 0.980 ± 0.003 \\
\textbullet & \textbullet &  &  & \textbullet &  & 0.899 ± 0.025 & 0.703 ± 0.028 & 0.705 ± 0.033 & 0.753 ± 0.034 & 0.703 ± 0.028 & 0.980 ± 0.002 \\
\textbullet & \textbullet &  &  & \textbullet & \textbullet & 0.921 ± 0.016 & 0.709 ± 0.024 & 0.697 ± 0.026 & 0.733 ± 0.044 & 0.709 ± 0.024 & 0.980 ± 0.002 \\
\textbullet & \textbullet &  & \textbullet &  &  & 0.934 ± 0.011 & 0.747 ± 0.015 & 0.736 ± 0.012 & 0.760 ± 0.007 & 0.747 ± 0.015 & 0.983 ± 0.001 \\
\textbullet & \textbullet &  & \textbullet &  & \textbullet & 0.926 ± 0.015 & 0.740 ± 0.035 & 0.733 ± 0.036 & 0.762 ± 0.031 & 0.740 ± 0.035 & 0.983 ± 0.003 \\
\textbullet & \textbullet &  & \textbullet & \textbullet &  & 0.932 ± 0.013 & 0.737 ± 0.025 & 0.726 ± 0.027 & 0.747 ± 0.027 & 0.737 ± 0.025 & 0.982 ± 0.002 \\
\textbullet & \textbullet &  & \textbullet & \textbullet & \textbullet & 0.934 ± 0.018 & 0.755 ± 0.024 & 0.749 ± 0.019 & 0.773 ± 0.011 & 0.755 ± 0.024 & 0.984 ± 0.002 \\
\textbullet & \textbullet & \textbullet &  &  &  & 0.904 ± 0.018 & 0.613 ± 0.054 & 0.584 ± 0.070 & 0.594 ± 0.075 & 0.613 ± 0.054 & 0.974 ± 0.004 \\
\textbullet & \textbullet & \textbullet &  &  & \textbullet & 0.887 ± 0.017 & 0.597 ± 0.064 & 0.555 ± 0.070 & 0.558 ± 0.093 & 0.597 ± 0.064 & 0.972 ± 0.005 \\
\textbullet & \textbullet & \textbullet &  & \textbullet &  & 0.902 ± 0.021 & 0.598 ± 0.026 & 0.552 ± 0.049 & 0.551 ± 0.076 & 0.598 ± 0.026 & 0.973 ± 0.003 \\
\textbullet & \textbullet & \textbullet &  & \textbullet & \textbullet & 0.898 ± 0.026 & 0.596 ± 0.039 & 0.574 ± 0.035 & 0.584 ± 0.042 & 0.596 ± 0.039 & 0.973 ± 0.002 \\
\textbullet & \textbullet & \textbullet & \textbullet &  &  & 0.947 ± 0.012 & 0.728 ± 0.021 & 0.722 ± 0.023 & 0.744 ± 0.027 & 0.728 ± 0.021 & 0.982 ± 0.001 \\
\textbullet & \textbullet & \textbullet & \textbullet &  & \textbullet & 0.943 ± 0.007 & 0.713 ± 0.020 & 0.695 ± 0.032 & 0.719 ± 0.037 & 0.713 ± 0.020 & 0.981 ± 0.002 \\
\textbullet & \textbullet & \textbullet & \textbullet & \textbullet &  & 0.936 ± 0.017 & 0.696 ± 0.057 & 0.673 ± 0.059 & 0.690 ± 0.048 & 0.696 ± 0.057 & 0.979 ± 0.004 \\
\textbullet & \textbullet & \textbullet & \textbullet & \textbullet & \textbullet & 0.945 ± 0.005 & 0.744 ± 0.025 & 0.735 ± 0.022 & 0.756 ± 0.037 & 0.744 ± 0.025 & 0.983 ± 0.001 \\
\end{longtable}

          \begin{longtable}{ccccccrrrrrr}
\caption{Classification metrics for MLP IF with different omics combination on CCLE dataset.} \label{tab:perf_comb_MLP IF_CCLE} \\
\toprule
mRNA & miRNA & CNV & DNAm & Protein & Metabolomics & AUROC & Accuracy & F1 & Precision & Recall & Specificity \\
\midrule
\endfirsthead
\caption[]{Classification metrics for MLP IF with different omics combination on CCLE dataset.} \\
\toprule
mRNA & miRNA & CNV & DNAm & Protein & Metabolomics & AUROC & Accuracy & F1 & Precision & Recall & Specificity \\
\midrule
\endhead
\midrule
\multicolumn{12}{r}{Continued on next page} \\
\midrule
\endfoot
\bottomrule
\endlastfoot
 &  &  &  & \textbullet & \textbullet & 0.997 ± 0.003 & 0.954 ± 0.010 & 0.954 ± 0.012 & 0.963 ± 0.012 & 0.954 ± 0.010 & 0.997 ± 0.001 \\
 &  &  & \textbullet &  & \textbullet & 0.998 ± 0.002 & 0.964 ± 0.013 & 0.964 ± 0.014 & 0.969 ± 0.011 & 0.964 ± 0.013 & 0.997 ± 0.001 \\
 &  &  & \textbullet & \textbullet &  & 0.996 ± 0.003 & 0.955 ± 0.011 & 0.955 ± 0.012 & 0.961 ± 0.013 & 0.955 ± 0.011 & 0.997 ± 0.001 \\
 &  &  & \textbullet & \textbullet & \textbullet & 0.997 ± 0.002 & 0.945 ± 0.028 & 0.946 ± 0.028 & 0.955 ± 0.025 & 0.945 ± 0.028 & 0.996 ± 0.002 \\
 &  & \textbullet &  &  & \textbullet & 0.995 ± 0.002 & 0.919 ± 0.028 & 0.922 ± 0.028 & 0.932 ± 0.028 & 0.919 ± 0.028 & 0.994 ± 0.002 \\
 &  & \textbullet &  & \textbullet &  & 0.988 ± 0.002 & 0.876 ± 0.023 & 0.873 ± 0.021 & 0.898 ± 0.024 & 0.876 ± 0.023 & 0.992 ± 0.001 \\
 &  & \textbullet &  & \textbullet & \textbullet & 0.999 ± 0.000 & 0.950 ± 0.018 & 0.950 ± 0.018 & 0.964 ± 0.012 & 0.950 ± 0.018 & 0.996 ± 0.001 \\
 &  & \textbullet & \textbullet &  &  & 0.991 ± 0.007 & 0.926 ± 0.013 & 0.925 ± 0.015 & 0.935 ± 0.014 & 0.926 ± 0.013 & 0.995 ± 0.001 \\
 &  & \textbullet & \textbullet &  & \textbullet & 0.996 ± 0.002 & 0.938 ± 0.013 & 0.935 ± 0.014 & 0.941 ± 0.014 & 0.938 ± 0.013 & 0.995 ± 0.001 \\
 &  & \textbullet & \textbullet & \textbullet &  & 0.995 ± 0.003 & 0.948 ± 0.011 & 0.948 ± 0.013 & 0.958 ± 0.011 & 0.948 ± 0.011 & 0.996 ± 0.001 \\
 &  & \textbullet & \textbullet & \textbullet & \textbullet & 0.994 ± 0.005 & 0.946 ± 0.016 & 0.944 ± 0.017 & 0.954 ± 0.013 & 0.946 ± 0.016 & 0.996 ± 0.001 \\
 & \textbullet &  &  &  & \textbullet & 0.995 ± 0.001 & 0.918 ± 0.013 & 0.923 ± 0.012 & 0.937 ± 0.011 & 0.918 ± 0.013 & 0.994 ± 0.001 \\
 & \textbullet &  &  & \textbullet &  & 0.993 ± 0.001 & 0.883 ± 0.013 & 0.883 ± 0.012 & 0.909 ± 0.015 & 0.883 ± 0.013 & 0.992 ± 0.001 \\
 & \textbullet &  &  & \textbullet & \textbullet & 0.998 ± 0.001 & 0.958 ± 0.005 & 0.958 ± 0.003 & 0.964 ± 0.005 & 0.958 ± 0.005 & 0.997 ± 0.000 \\
 & \textbullet &  & \textbullet &  &  & 0.994 ± 0.006 & 0.940 ± 0.019 & 0.939 ± 0.021 & 0.947 ± 0.020 & 0.940 ± 0.019 & 0.996 ± 0.001 \\
 & \textbullet &  & \textbullet &  & \textbullet & 0.995 ± 0.003 & 0.945 ± 0.017 & 0.943 ± 0.018 & 0.950 ± 0.016 & 0.945 ± 0.017 & 0.996 ± 0.001 \\
 & \textbullet &  & \textbullet & \textbullet &  & 0.997 ± 0.002 & 0.961 ± 0.017 & 0.962 ± 0.017 & 0.970 ± 0.012 & 0.961 ± 0.017 & 0.997 ± 0.001 \\
 & \textbullet &  & \textbullet & \textbullet & \textbullet & 0.998 ± 0.002 & 0.973 ± 0.015 & 0.972 ± 0.018 & 0.974 ± 0.017 & 0.973 ± 0.015 & 0.998 ± 0.001 \\
 & \textbullet & \textbullet &  &  &  & 0.988 ± 0.003 & 0.872 ± 0.015 & 0.875 ± 0.016 & 0.905 ± 0.013 & 0.872 ± 0.015 & 0.992 ± 0.001 \\
 & \textbullet & \textbullet &  &  & \textbullet & 0.996 ± 0.003 & 0.926 ± 0.019 & 0.930 ± 0.020 & 0.949 ± 0.017 & 0.926 ± 0.019 & 0.995 ± 0.001 \\
 & \textbullet & \textbullet &  & \textbullet &  & 0.995 ± 0.003 & 0.886 ± 0.025 & 0.889 ± 0.027 & 0.917 ± 0.019 & 0.886 ± 0.025 & 0.993 ± 0.002 \\
 & \textbullet & \textbullet &  & \textbullet & \textbullet & 0.999 ± 0.001 & 0.938 ± 0.016 & 0.943 ± 0.016 & 0.958 ± 0.012 & 0.938 ± 0.016 & 0.996 ± 0.001 \\
 & \textbullet & \textbullet & \textbullet &  &  & 0.994 ± 0.003 & 0.929 ± 0.017 & 0.926 ± 0.020 & 0.940 ± 0.021 & 0.929 ± 0.017 & 0.995 ± 0.001 \\
 & \textbullet & \textbullet & \textbullet &  & \textbullet & 0.998 ± 0.002 & 0.949 ± 0.014 & 0.948 ± 0.017 & 0.959 ± 0.014 & 0.949 ± 0.014 & 0.996 ± 0.001 \\
 & \textbullet & \textbullet & \textbullet & \textbullet &  & 0.996 ± 0.003 & 0.956 ± 0.015 & 0.955 ± 0.016 & 0.964 ± 0.012 & 0.956 ± 0.015 & 0.997 ± 0.001 \\
 & \textbullet & \textbullet & \textbullet & \textbullet & \textbullet & 0.997 ± 0.002 & 0.956 ± 0.008 & 0.955 ± 0.008 & 0.963 ± 0.007 & 0.956 ± 0.008 & 0.997 ± 0.001 \\
\textbullet &  &  &  &  & \textbullet & 0.912 ± 0.014 & 0.764 ± 0.040 & 0.757 ± 0.041 & 0.810 ± 0.023 & 0.764 ± 0.040 & 0.984 ± 0.003 \\
\textbullet &  &  &  & \textbullet &  & 0.908 ± 0.028 & 0.755 ± 0.059 & 0.749 ± 0.065 & 0.812 ± 0.044 & 0.755 ± 0.059 & 0.983 ± 0.004 \\
\textbullet &  &  &  & \textbullet & \textbullet & 0.914 ± 0.023 & 0.747 ± 0.026 & 0.740 ± 0.023 & 0.797 ± 0.009 & 0.747 ± 0.026 & 0.984 ± 0.001 \\
\textbullet &  &  & \textbullet &  &  & 0.942 ± 0.016 & 0.856 ± 0.031 & 0.853 ± 0.026 & 0.887 ± 0.008 & 0.856 ± 0.031 & 0.991 ± 0.002 \\
\textbullet &  &  & \textbullet &  & \textbullet & 0.937 ± 0.020 & 0.868 ± 0.030 & 0.868 ± 0.022 & 0.900 ± 0.017 & 0.868 ± 0.030 & 0.991 ± 0.002 \\
\textbullet &  &  & \textbullet & \textbullet &  & 0.927 ± 0.009 & 0.834 ± 0.014 & 0.831 ± 0.015 & 0.875 ± 0.014 & 0.834 ± 0.014 & 0.989 ± 0.001 \\
\textbullet &  &  & \textbullet & \textbullet & \textbullet & 0.940 ± 0.011 & 0.880 ± 0.038 & 0.875 ± 0.029 & 0.895 ± 0.020 & 0.880 ± 0.038 & 0.992 ± 0.002 \\
\textbullet &  & \textbullet &  &  &  & 0.910 ± 0.013 & 0.754 ± 0.029 & 0.746 ± 0.034 & 0.791 ± 0.045 & 0.754 ± 0.029 & 0.983 ± 0.002 \\
\textbullet &  & \textbullet &  &  & \textbullet & 0.920 ± 0.015 & 0.769 ± 0.023 & 0.757 ± 0.025 & 0.799 ± 0.037 & 0.769 ± 0.023 & 0.985 ± 0.001 \\
\textbullet &  & \textbullet &  & \textbullet &  & 0.922 ± 0.022 & 0.780 ± 0.026 & 0.772 ± 0.024 & 0.813 ± 0.026 & 0.780 ± 0.026 & 0.986 ± 0.001 \\
\textbullet &  & \textbullet &  & \textbullet & \textbullet & 0.917 ± 0.017 & 0.809 ± 0.022 & 0.807 ± 0.028 & 0.848 ± 0.035 & 0.809 ± 0.022 & 0.987 ± 0.002 \\
\textbullet &  & \textbullet & \textbullet &  &  & 0.939 ± 0.015 & 0.859 ± 0.031 & 0.856 ± 0.029 & 0.882 ± 0.026 & 0.859 ± 0.031 & 0.991 ± 0.002 \\
\textbullet &  & \textbullet & \textbullet &  & \textbullet & 0.939 ± 0.007 & 0.872 ± 0.029 & 0.867 ± 0.023 & 0.895 ± 0.012 & 0.872 ± 0.029 & 0.991 ± 0.001 \\
\textbullet &  & \textbullet & \textbullet & \textbullet &  & 0.958 ± 0.011 & 0.894 ± 0.032 & 0.887 ± 0.031 & 0.907 ± 0.022 & 0.894 ± 0.032 & 0.993 ± 0.002 \\
\textbullet &  & \textbullet & \textbullet & \textbullet & \textbullet & 0.955 ± 0.011 & 0.896 ± 0.026 & 0.892 ± 0.024 & 0.912 ± 0.010 & 0.896 ± 0.026 & 0.993 ± 0.001 \\
\textbullet & \textbullet &  &  &  &  & 0.908 ± 0.015 & 0.761 ± 0.012 & 0.756 ± 0.007 & 0.806 ± 0.008 & 0.761 ± 0.012 & 0.984 ± 0.001 \\
\textbullet & \textbullet &  &  &  & \textbullet & 0.919 ± 0.026 & 0.794 ± 0.041 & 0.787 ± 0.040 & 0.820 ± 0.032 & 0.794 ± 0.041 & 0.986 ± 0.003 \\
\textbullet & \textbullet &  &  & \textbullet &  & 0.915 ± 0.015 & 0.771 ± 0.024 & 0.752 ± 0.025 & 0.806 ± 0.021 & 0.771 ± 0.024 & 0.984 ± 0.001 \\
\textbullet & \textbullet &  &  & \textbullet & \textbullet & 0.922 ± 0.018 & 0.801 ± 0.009 & 0.798 ± 0.010 & 0.837 ± 0.012 & 0.801 ± 0.009 & 0.987 ± 0.001 \\
\textbullet & \textbullet &  & \textbullet &  &  & 0.931 ± 0.013 & 0.864 ± 0.025 & 0.867 ± 0.022 & 0.901 ± 0.022 & 0.864 ± 0.025 & 0.991 ± 0.002 \\
\textbullet & \textbullet &  & \textbullet &  & \textbullet & 0.943 ± 0.014 & 0.869 ± 0.039 & 0.864 ± 0.038 & 0.893 ± 0.036 & 0.869 ± 0.039 & 0.991 ± 0.003 \\
\textbullet & \textbullet &  & \textbullet & \textbullet &  & 0.940 ± 0.005 & 0.858 ± 0.025 & 0.853 ± 0.021 & 0.878 ± 0.012 & 0.858 ± 0.025 & 0.991 ± 0.001 \\
\textbullet & \textbullet &  & \textbullet & \textbullet & \textbullet & 0.948 ± 0.007 & 0.883 ± 0.028 & 0.875 ± 0.021 & 0.897 ± 0.015 & 0.883 ± 0.028 & 0.992 ± 0.002 \\
\textbullet & \textbullet & \textbullet &  &  &  & 0.917 ± 0.011 & 0.785 ± 0.017 & 0.777 ± 0.015 & 0.818 ± 0.025 & 0.785 ± 0.017 & 0.986 ± 0.001 \\
\textbullet & \textbullet & \textbullet &  &  & \textbullet & 0.922 ± 0.021 & 0.794 ± 0.025 & 0.787 ± 0.021 & 0.829 ± 0.026 & 0.794 ± 0.025 & 0.986 ± 0.002 \\
\textbullet & \textbullet & \textbullet &  & \textbullet &  & 0.918 ± 0.016 & 0.805 ± 0.043 & 0.791 ± 0.041 & 0.821 ± 0.030 & 0.805 ± 0.043 & 0.987 ± 0.003 \\
\textbullet & \textbullet & \textbullet &  & \textbullet & \textbullet & 0.933 ± 0.020 & 0.823 ± 0.065 & 0.819 ± 0.062 & 0.860 ± 0.037 & 0.823 ± 0.065 & 0.988 ± 0.004 \\
\textbullet & \textbullet & \textbullet & \textbullet &  &  & 0.955 ± 0.012 & 0.872 ± 0.030 & 0.865 ± 0.029 & 0.884 ± 0.029 & 0.872 ± 0.030 & 0.991 ± 0.002 \\
\textbullet & \textbullet & \textbullet & \textbullet &  & \textbullet & 0.948 ± 0.008 & 0.890 ± 0.022 & 0.889 ± 0.020 & 0.916 ± 0.009 & 0.890 ± 0.022 & 0.993 ± 0.001 \\
\textbullet & \textbullet & \textbullet & \textbullet & \textbullet &  & 0.942 ± 0.016 & 0.885 ± 0.029 & 0.884 ± 0.022 & 0.912 ± 0.015 & 0.885 ± 0.029 & 0.992 ± 0.002 \\
\textbullet & \textbullet & \textbullet & \textbullet & \textbullet & \textbullet & 0.956 ± 0.012 & 0.900 ± 0.020 & 0.898 ± 0.022 & 0.916 ± 0.020 & 0.900 ± 0.020 & 0.994 ± 0.001 \\
\end{longtable}

          \begin{longtable}{ccccccrrrrrr}
\caption{Classification metrics for MOGONET with different omics combination on CCLE dataset.} \label{tab:perf_comb_MOGONET_CCLE} \\
\toprule
mRNA & miRNA & CNV & DNAm & Protein & Metabolomics & AUROC & Accuracy & F1 & Precision & Recall & Specificity \\
\midrule
\endfirsthead
\caption[]{Classification metrics for MOGONET with different omics combination on CCLE dataset.} \\
\toprule
mRNA & miRNA & CNV & DNAm & Protein & Metabolomics & AUROC & Accuracy & F1 & Precision & Recall & Specificity \\
\midrule
\endhead
\midrule
\multicolumn{12}{r}{Continued on next page} \\
\midrule
\endfoot
\bottomrule
\endlastfoot
 &  &  &  & \textbullet & \textbullet & 0.878 ± 0.012 & 0.483 ± 0.036 & 0.470 ± 0.035 & 0.491 ± 0.049 & 0.483 ± 0.036 & 0.967 ± 0.003 \\
 &  &  & \textbullet &  & \textbullet & 0.913 ± 0.009 & 0.558 ± 0.019 & 0.549 ± 0.023 & 0.586 ± 0.029 & 0.558 ± 0.019 & 0.972 ± 0.001 \\
 &  &  & \textbullet & \textbullet &  & 0.935 ± 0.007 & 0.598 ± 0.015 & 0.597 ± 0.025 & 0.667 ± 0.054 & 0.598 ± 0.015 & 0.974 ± 0.001 \\
 &  &  & \textbullet & \textbullet & \textbullet & 0.901 ± 0.012 & 0.633 ± 0.046 & 0.616 ± 0.040 & 0.657 ± 0.032 & 0.633 ± 0.046 & 0.977 ± 0.003 \\
 &  & \textbullet &  &  & \textbullet & 0.842 ± 0.008 & 0.351 ± 0.048 & 0.345 ± 0.044 & 0.381 ± 0.060 & 0.351 ± 0.048 & 0.959 ± 0.002 \\
 &  & \textbullet &  & \textbullet &  & 0.889 ± 0.004 & 0.451 ± 0.059 & 0.449 ± 0.059 & 0.514 ± 0.082 & 0.451 ± 0.059 & 0.964 ± 0.003 \\
 &  & \textbullet &  & \textbullet & \textbullet & 0.876 ± 0.015 & 0.550 ± 0.033 & 0.531 ± 0.037 & 0.556 ± 0.049 & 0.550 ± 0.033 & 0.971 ± 0.001 \\
 &  & \textbullet & \textbullet &  &  & 0.945 ± 0.003 & 0.666 ± 0.043 & 0.648 ± 0.046 & 0.680 ± 0.049 & 0.666 ± 0.043 & 0.977 ± 0.002 \\
 &  & \textbullet & \textbullet &  & \textbullet & 0.920 ± 0.017 & 0.620 ± 0.056 & 0.598 ± 0.052 & 0.623 ± 0.041 & 0.620 ± 0.056 & 0.975 ± 0.003 \\
 &  & \textbullet & \textbullet & \textbullet &  & 0.938 ± 0.009 & 0.664 ± 0.022 & 0.653 ± 0.031 & 0.698 ± 0.048 & 0.664 ± 0.022 & 0.978 ± 0.002 \\
 & \textbullet &  &  &  & \textbullet & 0.860 ± 0.013 & 0.454 ± 0.028 & 0.443 ± 0.021 & 0.480 ± 0.025 & 0.454 ± 0.028 & 0.963 ± 0.002 \\
 & \textbullet &  &  & \textbullet &  & 0.865 ± 0.005 & 0.513 ± 0.047 & 0.501 ± 0.048 & 0.531 ± 0.066 & 0.513 ± 0.047 & 0.968 ± 0.002 \\
 & \textbullet &  &  & \textbullet & \textbullet & 0.862 ± 0.022 & 0.514 ± 0.033 & 0.491 ± 0.030 & 0.508 ± 0.043 & 0.514 ± 0.033 & 0.968 ± 0.002 \\
 & \textbullet &  & \textbullet &  &  & 0.918 ± 0.005 & 0.636 ± 0.051 & 0.621 ± 0.052 & 0.636 ± 0.048 & 0.636 ± 0.051 & 0.974 ± 0.003 \\
 & \textbullet &  & \textbullet &  & \textbullet & 0.903 ± 0.007 & 0.604 ± 0.037 & 0.590 ± 0.035 & 0.632 ± 0.040 & 0.604 ± 0.037 & 0.972 ± 0.003 \\
 & \textbullet &  & \textbullet & \textbullet &  & 0.905 ± 0.024 & 0.613 ± 0.009 & 0.589 ± 0.009 & 0.621 ± 0.048 & 0.613 ± 0.009 & 0.974 ± 0.001 \\
 & \textbullet & \textbullet &  &  &  & 0.867 ± 0.003 & 0.481 ± 0.024 & 0.455 ± 0.011 & 0.497 ± 0.037 & 0.481 ± 0.024 & 0.964 ± 0.002 \\
 & \textbullet & \textbullet &  &  & \textbullet & 0.837 ± 0.013 & 0.497 ± 0.017 & 0.468 ± 0.031 & 0.478 ± 0.034 & 0.497 ± 0.017 & 0.965 ± 0.001 \\
 & \textbullet & \textbullet &  & \textbullet &  & 0.872 ± 0.013 & 0.547 ± 0.018 & 0.536 ± 0.018 & 0.591 ± 0.020 & 0.547 ± 0.018 & 0.968 ± 0.001 \\
 & \textbullet & \textbullet & \textbullet &  &  & 0.915 ± 0.015 & 0.638 ± 0.036 & 0.619 ± 0.041 & 0.649 ± 0.039 & 0.638 ± 0.036 & 0.974 ± 0.002 \\
\textbullet &  &  &  &  & \textbullet & 0.884 ± 0.011 & 0.514 ± 0.045 & 0.518 ± 0.041 & 0.577 ± 0.022 & 0.514 ± 0.045 & 0.969 ± 0.002 \\
\textbullet &  &  &  & \textbullet &  & 0.900 ± 0.009 & 0.539 ± 0.025 & 0.538 ± 0.021 & 0.599 ± 0.019 & 0.539 ± 0.025 & 0.970 ± 0.001 \\
\textbullet &  &  &  & \textbullet & \textbullet & 0.881 ± 0.027 & 0.594 ± 0.047 & 0.581 ± 0.040 & 0.620 ± 0.026 & 0.594 ± 0.047 & 0.974 ± 0.003 \\
\textbullet &  &  & \textbullet &  &  & 0.932 ± 0.024 & 0.712 ± 0.058 & 0.688 ± 0.057 & 0.720 ± 0.056 & 0.712 ± 0.058 & 0.980 ± 0.003 \\
\textbullet &  &  & \textbullet &  & \textbullet & 0.918 ± 0.009 & 0.657 ± 0.021 & 0.647 ± 0.010 & 0.689 ± 0.005 & 0.657 ± 0.021 & 0.977 ± 0.001 \\
\textbullet &  &  & \textbullet & \textbullet &  & 0.919 ± 0.012 & 0.663 ± 0.030 & 0.649 ± 0.023 & 0.722 ± 0.045 & 0.663 ± 0.030 & 0.977 ± 0.002 \\
\textbullet &  & \textbullet &  &  &  & 0.898 ± 0.012 & 0.566 ± 0.025 & 0.555 ± 0.015 & 0.618 ± 0.038 & 0.566 ± 0.025 & 0.971 ± 0.001 \\
\textbullet &  & \textbullet &  &  & \textbullet & 0.877 ± 0.021 & 0.565 ± 0.069 & 0.557 ± 0.061 & 0.595 ± 0.040 & 0.565 ± 0.069 & 0.972 ± 0.004 \\
\textbullet &  & \textbullet &  & \textbullet &  & 0.895 ± 0.019 & 0.602 ± 0.040 & 0.588 ± 0.033 & 0.622 ± 0.030 & 0.602 ± 0.040 & 0.973 ± 0.002 \\
\textbullet &  & \textbullet & \textbullet &  &  & 0.938 ± 0.009 & 0.688 ± 0.054 & 0.664 ± 0.053 & 0.691 ± 0.052 & 0.688 ± 0.054 & 0.979 ± 0.003 \\
\textbullet & \textbullet &  &  &  &  & 0.889 ± 0.009 & 0.572 ± 0.041 & 0.567 ± 0.045 & 0.598 ± 0.056 & 0.572 ± 0.041 & 0.971 ± 0.003 \\
\textbullet & \textbullet &  &  &  & \textbullet & 0.871 ± 0.030 & 0.563 ± 0.048 & 0.564 ± 0.045 & 0.607 ± 0.044 & 0.563 ± 0.048 & 0.971 ± 0.003 \\
\textbullet & \textbullet &  &  & \textbullet &  & 0.872 ± 0.028 & 0.600 ± 0.033 & 0.589 ± 0.033 & 0.638 ± 0.031 & 0.600 ± 0.033 & 0.973 ± 0.002 \\
\textbullet & \textbullet &  & \textbullet &  &  & 0.904 ± 0.007 & 0.660 ± 0.037 & 0.648 ± 0.039 & 0.672 ± 0.033 & 0.660 ± 0.037 & 0.976 ± 0.003 \\
\textbullet & \textbullet & \textbullet &  &  &  & 0.880 ± 0.009 & 0.596 ± 0.031 & 0.579 ± 0.038 & 0.609 ± 0.045 & 0.596 ± 0.031 & 0.972 ± 0.002 \\
\end{longtable}

 \end{landscape}